\section{行列式 I}

\subsection{作业}

\begin{problem}
	习题 1.1.1
	\begin{solution}
		\begin{enumerate}
			\item[(1)] $\tau = 3$,是奇排列;
			\item[(2)] 排列的左半部分和右半部分内部递增,因此无逆序对。而排列中每个奇数都能与每个小于它的偶数形成逆序对,因此 $\tau = \sum_{i=1}^n (i-1) = n(n-1)/2$。因此当 $n=4k+2 \text{ 或 } 4k+3 (k \in \mathbb{N})$ 时排列是奇排列,否则是偶排列。
		\end{enumerate}
	\end{solution}
\end{problem}

\begin{problem}
	习题 1.1.2
	\begin{solution}
		\begin{enumerate}
			\item[(1)] $a_{31} a_{25} a_{13} a_{54} a_{42} = a_{13} a_{25} a_{31} a_{42} a_{54}$,可知系数为 $-1$;
			\item[(2)] $a_{14} a_{23} a_{51} a_{32} a_{45} = a_{14} a_{23} a_{32} a_{45} a_{51}$,可知系数为 $-1$。
		\end{enumerate}
	\end{solution}
\end{problem}

\begin{problem}
	习题 1.1.3
	\begin{proof}
		设长度为 $n$ 的奇排列集合为 $\mathscr{P}_{n,0}$,偶排列集合为 $\mathscr{P}_{n,1}$。那么定义操作 $F$:对于某个排列 $p$,交换 $p_1,p_2$ 可得到一个新排列 $F(p)=p'$,显然 $p,p'$ 奇偶性不同。那么存在单射 $f: \mathscr{P}_{n,0} \to \mathscr{P}_{n,1} \quad p \mapsto F(p)$ 和单射 $g: \mathscr{P}_{n,1} \to \mathscr{P}_{n,0} \quad p \mapsto F(p)$。可得 $\mathscr{P}_{n,0}$ 和 $\mathscr{P}_{n,1}$ 等势,即奇排列个数与偶排列相等。
	\end{proof}
\end{problem}

\begin{problem}
	习题 1.1.4
	\begin{solution}
		\begin{enumerate}
			\item[(2)] 由题可知,行列式中唯一非零的项为 $a_{14} a_{21} a_{33} a_{42}$,其系数为 $1$,因此原行列式值为 $a_{14} a_{21} a_{33} a_{42}$。
			\item[(4)] 行列式的每一项中,每行每列均占且只占一个元素。因此原行列式的 $3,4,5$ 行中需要选出三个不同列的元素,其中必定有 $0$。因此原行列式值为 $0$。
		\end{enumerate}
	\end{solution}
\end{problem}

\begin{problem}
	习题 1.1.5
	\begin{proof}
		零元素数量 $> n^2 - n$ 等价于非零元素数量 $< n$。于是无法选出 $5$ 个非零元素。因此行列式必定为 $0$。
	\end{proof}
\end{problem}

\begin{problem}
	习题 1.1.6
	\begin{proof}
		记 $[x^k] F(x)$ 表示多项式 $F(x)$ 的 $x^k$ 项系数。

		行列式每个元素均是次数不超过 $1$ 的多项式,因此行列式的值是次数不超过 $n$ 的多项式。而除了 $\prod_{i=1}^n (x - a_{i,i})$ 这一项外,其余项均不含 $n$ 个 $x$,因此得到 $[x^n] f(x) = [x^n] \prod_{i=1}^n (x - a_{i,i}) = 1$,非零,于是 $f(x)$ 是 $n$ 次多项式。

		有 $[x^n] f(x) = 1,\ [x^{n-1}] f(x) = 0$。其中后者是因为有且仅有主对角线元素含 $x$,因此在选出 $n-1$ 个含 $x$ 的元素后,剩下的一个元素也只能在主对角线上,不可能得到含 $n-1$ 个 $x$ 的项。
	\end{proof}
\end{problem}
