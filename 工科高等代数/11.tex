\section{二次型 I}

\subsection{二次型的定义}

\begin{definition}[二次型]
	设 $P$ 是一个数域,一个系数在数域 $P$ 中的 $n$ 元齐次多项式
	$$
	f(x) = \sum_{i=1}^n \sum_{j=1}^n a_{i,j} x_i x_j
	$$
	满足 $a_{i,j}=a_{j,i}$,那么它称为\textbf{数域 $P$ 上的 $n$ 元二次型}。
\end{definition}

显然,二次型可以用矩阵表示。我们把 $\matr{A} = [a_{i,j}]$ 称为\textbf{二次型的矩阵},那么有 $\matr{A} = \matr{A}^\textrm{T}$,即二次型矩阵一定是对称的。若我们设
$$
\vect{x} = \cvec{x_1, x_2, \vdots, x_n}
$$
那么二次型可以表示为:
$$
f(\vect{x}) = \transpose{\vect{x}} \matr{A} \vect{x}
$$
这里认为 $1 \times 1$ 矩阵和数域 $P$ 中的数\sout{可隐式转化}同等。

\begin{definition}[线性替换]
	设 $x_1, x_2, \dots, x_n, y_1, y_2, \dots, y_n$ 是两组变元,那么称
	$$
	x_i = \sum_{j=1}^n c_{i,j} y_j
	$$
	为一个\textbf{线性替换}。线性替换可以表示称矩阵形式:
	$$
	\cvec{x_1, x_2, \vdots, x_n} = \begin{bmatrix}
		c_{1,1} & c_{1,2} & \cdots & c_{1,n} \\
		c_{2,1} & c_{2,2} & \cdots & c_{2,n} \\
		\vdots & \vdots & \ddots & \vdots \\
		c_{n,1} & c_{n,2} & \cdots & c_{n,n}
	\end{bmatrix} \cvec{y_1, y_2, \vdots, y_n}
	$$
	记 $\matr{C} = [c_{i,j}]$ 为\textbf{线性替换的矩阵}。若 $\det \matr{C} \neq 0$,则称线性替换是\textbf{非退化}的,否则称线性替换是\textbf{退化}的。
\end{definition}

\begin{property}
	线性替换总是把二次型变成二次型。

	\begin{proof}
		设有二次型 $f(\vect{x}) = \transpose{\vect{x}} \matr{A} \vect{x}$ 和线性替换 $\vect{x} = \matr{C} \vect{y}$,那么:
		$$
		f(\vect{x}) = \transpose{(\matr{C} \vect{y})} \matr{A} (\matr{C} \vect{y}) = \transpose{\vect{y}} (\transpose{\matr{C}} \matr{A} \matr{C}) \vect{y}
		$$
	\end{proof}
\end{property}

从线性替换对二次型矩阵的影响,我们可以引入一种等价关系:

\begin{definition}[合同变换]
	数域 $P$ 上的 $n$ 阶方阵 $\matr{A}, \matr{B}$ 称为\textbf{合同的},当且仅当存在 $n$ 阶可逆方阵 $\matr{C}$ 使得:
	$$
	\matr{B} = \transpose{\matr{C}} \matr{A} \matr{C}
	$$
	那么变换 $\matr{A} \mapsto \transpose{\matr{C}} \matr{A} \matr{C}$ 称为矩阵的一个\textbf{合同变换}。
\end{definition}

合同关系具有等价关系的性质:

\begin{itemize}
	\item 自反性:$\matr{A} = \transpose{\matr{E}} \matr{A} \matr{E}$;
	\item 对称性:$\matr{B} = \transpose{\matr{C}} \matr{A} \matr{C} \Leftrightarrow \matr{A} = \transpose{(\matr{C}^{-1})} \matr{B} \matr{C}^{-1}$;
	\item 传递性:$\matr{A}_1 = \transpose{\matr{C}_1} \matr{A} \matr{C}_1 \land \matr{A}_2 = \transpose{\matr{C}_2} \matr{A}_1 \matr{C}_2 \Rightarrow \matr{A}_2 = \transpose{(\matr{C}_1 \matr{C}_2)} \matr{A} (\matr{C}_1 \matr{C}_2)$。
\end{itemize}
