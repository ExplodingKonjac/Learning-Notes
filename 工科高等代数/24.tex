\section{欧几里德空间 II}

\subsection{正交矩阵}

\begin{definition}[正交矩阵]
	若 $n$ 阶实矩阵 $\matr{A}$ 满足:
	$$
	\matr{A} \transpose{\matr{A}} = \transpose{\matr{A}} \matr{A} = \matr{E}
	$$
	那么称 $\matr{A}$ 为正交矩阵。
\end{definition}

\begin{property}
	\begin{itemize}
		\item 正交矩阵可逆;
		\item 正交矩阵的乘积还是正交矩阵;
		\item 正交矩阵的逆还是正交矩阵。
		\item 正交矩阵的行列式为 $\pm 1$。
	\end{itemize}
\end{property}

\begin{property}
	标准正交基到标准正交基的过渡矩阵为正交矩阵。
\end{property}

\subsection{同构}

\begin{definition}[欧几里德空间的同构]
	$\mathbb{R}$ 上的欧几里德空间 $V, W$ 是\textbf{同构}的,当且仅当存在映射 $\sigma: V \to W$ 满足:
	\begin{itemize}
		\item $\sigma$ 是双射;
		\item $\forall\,\vect{\alpha}, \vect{\beta} \in V: \sigma(\vect{\alpha}, \vect{\alpha}) = \sigma(\vect{\alpha}) + \sigma(\vect{\beta})$;
		\item $\forall\,\vect{\alpha} \in V, k \in \mathbb{R}: \sigma(k \vect{\alpha}) = k \sigma(\vect{\alpha})$;
		\item $\forall\,\vect{\alpha}, \vect{\beta} \in V: (\vect{\alpha}, \vect{\beta}) = (\sigma(\vect{\alpha}), \sigma(\vect{\beta}))$。
	\end{itemize}
	$\sigma$ 称为欧几里德空间的一个\textbf{同构映射}。
\end{definition}

可见,欧几里德空间的同构也是线性空间的同构。其也是一种等价关系,具有自反性、对称性和传递性。

\subsection{正交变换}

\begin{definition}[正交变换]
	若欧几里德空间 $V$ 上的线性变换 $\mathscr{A}$ 满足:
	$$
	\forall\,\vect{\alpha}, \vect{\beta} \in V: (\vect{\alpha}, \vect{\beta}) = (\mathscr{A}(\vect{\alpha}), \mathscr{A}(\vect{\beta}))
	$$
	那么称 $\mathscr{A}$ 为\textbf{正交变换}。
\end{definition}

\begin{theorem}
	设 $\mathscr{A}$ 是 $n$ 维欧几里德空间 $V$ 上的线性变换,那么下面条件等价:
	\begin{enumerate}
		\item $\mathscr{A}$ 是正交变换;
		\item $\forall\,\vect{\alpha} \in V: |\vect{\alpha}| = |\mathscr{A}(\vect{\alpha})|$;
		\item 若 $\vect{\varepsilon}_1, \vect{\varepsilon}_2, \dots, \vect{\varepsilon}_n$ 是标准正交基,那么 $\mathscr{A}(\vect{\varepsilon}_1), \mathscr{A}(\vect{\varepsilon}_2), \dots, \mathscr{A}(\vect{\varepsilon}_n)$ 也是标准正交基;
		\item $\mathscr{A}$ 在任意一组标准正交基下的矩阵都是正交矩阵。
	\end{enumerate}

	\begin{proof}
		\ 
		\begin{itemize}
			\item $1 \Rightarrow 2$:$|\mathscr{A}(\vect{\alpha})|^2 = (\mathscr{A}(\vect{\alpha}), \mathscr{A}(\vect{\alpha})) = (\vect{\alpha}, \vect{\alpha}) = |\vect{\alpha}|^2$。
			
			\item $2 \Rightarrow 1$:取任意一组标准正交基 $\vect{\varepsilon}_1, \vect{\varepsilon}_2, \dots, \vect{\varepsilon}_n$,那么:
			$$
			\begin{aligned}
				|\vect{\varepsilon}_i + \vect{\varepsilon}_j|^2 & = |\mathscr{A}(\vect{\varepsilon}_i + \mathscr{A}(\vect{\varepsilon}_j)| \\
				(\vect{\varepsilon}_i + \vect{\varepsilon}_j, \vect{\varepsilon}_i + \vect{\varepsilon}_j) & = (\mathscr{A}(\vect{\varepsilon}_i) + \mathscr{A}(\vect{\varepsilon}_j), \mathscr{A}(\vect{\varepsilon}_i) + \mathscr{A}(\vect{\varepsilon}_j)) \\
				|\vect{\varepsilon}_i|^2 + |\vect{\varepsilon}_j|^2 + 2 (\vect{\varepsilon}_i, \vect{\varepsilon}_j) & = |\mathscr{A}(\vect{\varepsilon}_i)|^2 + |\mathscr{A}(\vect{\varepsilon}_j)|^2 + 2 (\mathscr{A}(\vect{\varepsilon}_i), \mathscr{A}(\vect{\varepsilon}_j)) \\
			\end{aligned}
			$$
			而 $|\vect{\varepsilon}_i| = |\mathscr{A}(\vect{\varepsilon}_i)|, |\vect{\varepsilon}_j| = |\mathscr{A}(\vect{\varepsilon}_j)|$,因此 $(\vect{\varepsilon}_i, \vect{\varepsilon}_j) = (\mathscr{A}(\vect{\varepsilon}_i), \mathscr{A}(\vect{\varepsilon}_j))$。于是得证。
			
			\item $1 \Rightarrow 3$:$(\vect{\varepsilon}_i \vect{\varepsilon}_j) = (\mathscr{A}(\vect{\varepsilon}_i), \mathscr{A}(\vect{\varepsilon}_j)) = 0$。
			
			\item $3 \Rightarrow 4$:根据线性变换的矩阵的定义,若 $\vect{\varepsilon}_1, \vect{\varepsilon}_2, \dots, \vect{\varepsilon}_n$ 是标准正交基,那么
			$$
			[\mathscr{A}(\vect{\varepsilon}_1), \mathscr{A}(\vect{\varepsilon}_2), \dots, \mathscr{A}(\vect{\varepsilon}_n)] = [\vect{\varepsilon}_1, \vect{\varepsilon}_2, \dots, \vect{\varepsilon}_n] \matr{A}
			$$
			而 $\mathscr{A}(\vect{\varepsilon}_1), \mathscr{A}(\vect{\varepsilon}_2), \dots, \mathscr{A}(\vect{\varepsilon}_n)$ 也是标准正交基,那么 $\matr{A}$ 也是正交矩阵。

			\item $4 \Rightarrow 1$:取一组标准正交基 $\vect{\varepsilon}_1, \vect{\varepsilon}_2, \dots, \vect{\varepsilon}_n$,设 $\matr{A}$ 是 $\mathscr{A}$ 在这组基下的矩阵。设:
			$$
			\vect{\alpha} = [\vect{\varepsilon}_1, \vect{\varepsilon}_2, \dots, \vect{\varepsilon}_n] \vect{x},\ \vect{\beta} = [\vect{\varepsilon}_1, \vect{\varepsilon}_2), \dots, \vect{\varepsilon}_n] \vect{y}
			$$
			其中 $\vect{x}, \vect{y} \in \mathbb{R}^n$。因此
			$$
			(\mathscr{A}(\vect{\alpha}), \mathscr{A}(\vect{\beta})) = \transpose{(\matr{A} \vect{x})} (\matr{A} \vect{y}) = \transpose{\vect{x}} \transpose{\matr{A}} \matr{A} \vect{y} = \transpose{\vect{x}} \vect{y} = (\vect{\alpha}, \vect{\beta})
			$$
		\end{itemize}
	\end{proof}
\end{theorem}

\subsection{子空间}

\begin{definition}[正交子空间]
	对于欧几里德空间的两个子空间 $V, W$,若
	$$
	\forall\,\vect{\alpha} \in V, \vect{\beta} \in W: (\vect{\alpha}, \vect{\beta}) = 0
	$$
	则称 $V$ 与 $W$ 为\textbf{正交子空间}。
\end{definition}

\begin{theorem}
	若 $V_1, V_2, \dots, V_n$ 两两正交,那么 $V_1 + V_2 + \cdots + V_n$ 是直和。

	\begin{proof}
		只需证明 $\vect{0}$ 表示唯一。设
		$$
		\vect{0} = \sum_{i=1}^n \vect{\alpha}_i \quad (\vect{\alpha}_i \in V_i)
		$$
		那么:
		$$
		0 = (\vect{0}, \vect{\alpha}_j) = \sum_{i=1}^n (\vect{\alpha}_i, \vect{\alpha}_j) = (\vect{\alpha}_j, \vect{\alpha}_j)
		$$
		因此 $\vect{\alpha}_j = \vect{0}$。
	\end{proof}
\end{theorem}

\begin{definition}[正交补]
	对于欧几里德空间 $V$ 的子空间 $W$,定义 $W$ 的正交补为:
	$$
	W^\perp = \{\vect{v}: \vect{v} \in V \land \forall\,\vect{w} \in W: (\vect{v}, \vect{w}) = 0\}
	$$
\end{definition}

\begin{property}
	$$
	W \oplus W^\perp = V
	$$
\end{property}

可见,一个空间的补空间不是唯一的,但是它的正交补是唯一的。

\subsection{作业}

\begin{problem}
	第九章习题 1

	\begin{proof}
		记 $\matr{A} = [c_{i,j}]$,那么:
		$$
		(\vect{\alpha}, \vect{\beta}) = \sum_{i=1}^n \sum_{j=1}^n x_i y_j c_{i,j}
		$$
		\begin{enumerate}
			\item[\textbf{1)}]
			\begin{itemize}
				\item 可交换:因为 $\matr{A}$ 是对称矩阵,易得 $(\vect{\alpha}, \vect{\beta}) = (\vect{\beta}, \vect{\alpha})$。
				\item 双线性:$(k_1 \vect{\alpha}_1 + k_2 \vect{\alpha}_2, \vect{\beta}) = \transpose{(k_1 \vect{\alpha}_1 + k_2 \vect{\alpha}_2)} \matr{A} \vect{\beta} = k_1 \transpose{\vect{\alpha}_1} \matr{A} \vect{\beta} + k_2 \transpose{\vect{\alpha}_2} \matr{A} \vect{\beta} = k_1 (\vect{\alpha}_1, \vect{\beta}) + k_2 (\vect{\alpha}_2, \vect{\beta})$。
				\item 正定性:因为 $\matr{A}$ 是正定矩阵,因此 $\transpose{\vect{x}} \matr{A} \vect{x} \ge 0$。
			\end{itemize}
			
			故这是一个合法的内积,在此定义下 $\mathbb{R}^n$ 是欧几里德空间。

			\item[\textbf{2)}] $(\vect{\varepsilon}_i, \vect{\varepsilon}_j) = c_{i,j}$,因此度量矩阵就是 $\matr{A}$。
			\item[\textbf{3)}]
			$$
			\ab(\sum_{i=1}^n \sum_{j=1}^n x_i y_j c_{i,j})^2 \le \ab(\sum_{i=1}^n \sum_{j=1}^n x_i x_j c_{i,j}) \ab(\sum_{i=1}^n \sum_{j=1}^n y_i y_j c_{i,j})
			$$
		\end{enumerate}
	\end{proof}
\end{problem}

\begin{problem}
	第九章习题 2

	\begin{solution}
		\begin{enumerate}
			\item[\textbf{1)}]
			$$
			\langle \vect{\alpha}, \vect{\beta} \rangle = \arccos \frac{0}{2 \sqrt{3} \cdot \sqrt{10}} = \frac{\pi}{2}
			$$
		\end{enumerate}
	\end{solution}
\end{problem}

\begin{problem}
	第九章习题 3

	\begin{proof}
		只需证 $\abs{\vect{\alpha} + \vect{\beta}} \le \abs{\vect{\alpha}} + \abs{\vect{\beta}}$。而:
		$$
		\begin{aligned}
			\abs{\vect{\alpha} + \vect{\beta}}^2 & = (\vect{\alpha} + \vect{\beta}, \vect{\alpha} + \vect{\beta}) \\
			& = \abs{\vect{\alpha}}^2 + 2 (\vect{\alpha}, \vect{\beta}) + \abs{\vect{\beta}}^2 \\
			& \le \abs{\vect{\alpha}}^2 + 2 \abs{\vect{\alpha}} \abs{\vect{\beta}} + \abs{\vect{\beta}}^2 \\
			& = (\abs{\vect{\alpha}} + \abs{\vect{\beta}})^2
		\end{aligned}
		$$
		故得证。
	\end{proof}
\end{problem}

\begin{problem}
	第九章习题 4

	\begin{solution}
		$$
		\vect{\alpha} = \transpose{\rvec{-\frac{4}{\sqrt{26}}, 0, -\frac{1}{\sqrt{26}}, \frac{3}{\sqrt{26}}}}
		$$
	\end{solution}
\end{problem}

\begin{problem}
	第九章习题 5

	\begin{proof}
		\begin{enumerate}
			\item[\textbf{1)}] 若 $\vect{\gamma} \neq \vect{0}$,那么取 $V$ 的一组正交基 $\vect{\varepsilon}_1', \vect{\varepsilon}_2', \dots, \vect{\varepsilon}_n'$,则
			$$
			\vect{\varepsilon}_1', \vect{\varepsilon}_2', \dots, \vect{\varepsilon}_n', \vect{\gamma}
			$$
			是正交向量组,因此也是线性无关组,而这是不可能的。因此 $\vect{\gamma} = \vect{0}$。

			\item[\textbf{2)}]
			$$
			\begin{gathered}
				\forall\,\vect{\alpha} \in V: (\vect{\gamma}_1, \vect{\alpha}) = (\vect{\gamma}_2, \vect{\alpha}) \\
				\Rightarrow \forall\,\vect{\alpha} \in V: (\vect{\gamma}_1 - \vect{\gamma}_2, \vect{\alpha}) = 0 \\
				\Rightarrow (\vect{\gamma}_1 - \vect{\gamma}_2, \vect{\gamma}_1 - \vect{\gamma}_2) = 0 \\
				\Rightarrow \vect{\gamma}_1 - \vect{\gamma}_2 = \vect{0} \\
			\end{gathered}
			$$
		\end{enumerate}
	\end{proof}
\end{problem}

\begin{problem}
	第九章习题 7

	\begin{solution}
		使用 Schmidt 正交化方法,得到:
		$$
		\begin{gathered}
			\vect{\eta}_1 = \frac{1}{\sqrt{2}} \ab(\vect{\varepsilon}_1 + \vect{\varepsilon}_5) \\
			\vect{\eta}_2 = \frac{\sqrt{2}}{\sqrt{5}} \ab(-\frac{1}{2} \vect{\varepsilon}_1 + \vect{\varepsilon}_2 - \vect{\varepsilon}_4 + \frac{1}{2} \vect{\varepsilon}_5) \\
			\vect{\eta}_3 = \frac{1}{2} \ab(\vect{\varepsilon}_1 + \vect{\varepsilon}_2 + \vect{\varepsilon}_3 - \vect{\varepsilon}_5)
		\end{gathered}
		$$
	\end{solution}
\end{problem}

\begin{problem}
	第九章习题 8
	
	\begin{solution}
		解得一组基础解系:
		$$
		\vect{\varepsilon}_1 = \cvec{4, -5, 0, 0, 1},\ \vect{\varepsilon}_2 = \cvec{-1, 1, 0, 1, 0},\ \vect{\varepsilon}_3 = \cvec{0, 1, 1, 0, 0}
		$$
		运用 Schmidt 正交化方法:
		$$
		\vect{\eta}_1 = \frac{1}{\sqrt{42}} \cvec{4, -5, 0, 0, 1},\ \vect{\eta}_2 = \frac{1}{\sqrt{210}} \cvec{-2, -1, 0, 14, 3},\ \vect{\eta}_3 = \frac{1}{3 \sqrt{35}} \cvec{7, 6, 15, 1, 2}
		$$
	\end{solution}
\end{problem}

\begin{problem}
	第九章习题 11

	\begin{proof}
		\begin{enumerate}
			\item[\textbf{1)}] 设有两组基
			\begin{gather}
				\vect{\varepsilon}_1, \vect{\varepsilon}_2, \dots, \vect{\varepsilon}_n \tag{1} \\
				\vect{\eta}_1, \vect{\eta}_2, \dots, \vect{\eta}_n \tag{2}
			\end{gather}

			设 $(1)$ 的度量矩阵为 $\matr{A}$,$(2)$ 的度量矩阵为 $\matr{B}$,$(1)$ 到 $(2)$ 的过渡矩阵为 $\matr{X}$。
			
			对于任意 $\vect{\alpha}, \vect{\beta} \in V$,设它们在基 $(1)$ 下坐标分别为 $\vect{x}, \vect{y}$,在基 $(2)$ 下坐标分别为 $\vect{x}', \vect{y}'$,那么:
			$$
			(\vect{\alpha}, \vect{\beta}) = \transpose{\vect{x}} \matr{A} \vect{y} = \transpose{\vect{x}'} \matr{B} \vect{y}'
			$$

			而根据基变换,有:
			$$
			\vect{x} = \matr{X} \vect{x}',\ \vect{y} = \matr{X} \vect{y}'
			$$
			于是得到
			$$
			(\vect{\alpha}, \vect{\beta}) = \transpose{(\matr{X} \vect{x}')} \matr{A} (\matr{X} \vect{y}') = \transpose{\vect{x}'} \ab(\transpose{\matr{X}} \matr{A} \matr{X}) \vect{y}'
			$$
			因此
			$$
			\matr{B} = \transpose{\matr{X}} \matr{A} \matr{X}
			$$
			也就是说 $\matr{A}, \matr{B}$ 合同。
		\end{enumerate}
	\end{proof}
\end{problem}

\begin{problem}
	第九章习题 12

	\begin{proof}
		可以发现,当对矩阵
		$$
		\begin{bmatrix}
			\vect{\alpha}_1 & \vect{\alpha}_2 & \cdots & \vect{\alpha}_n
		\end{bmatrix}
		$$
		执行一次矩阵为 $P$ 的初等列变换,$\matr{\Delta}$ 就会相应地进行变换:
		$$
		\matr{\Delta} \mapsto \transpose{\matr{P}} \matr{\Delta} \matr{P}
		$$
		这样的变换不改变 $\vect{\alpha}_1, \vect{\alpha}_2, \dots, \vect{\alpha}_n$ 的线性相关性,$\matr{\Delta}$ 也一直保持合同,$\det \matr{\Delta}$ 不变。

		因此我们可以执行变换直到将 $\matr{\Delta}$ 对角化:
		$$
		\matr{\Delta} = \diag(\lambda_1, \lambda_2, \dots, \lambda_n)
		$$
		也就是说,$\vect{\alpha}_1, \vect{\alpha}_2, \dots, \vect{\alpha}_n$ 两两正交。在此条件下,我们有:
		$$
		\det \matr{\Delta} \neq 0 \Leftrightarrow \prod_{i=1}^n \lambda_i \neq 0 \Leftrightarrow \forall\,i \in \{1, 2, \dots, n\}: \vect{\alpha}_i \neq 0 \Leftrightarrow \vect{\alpha}_1, \vect{\alpha}_2, \dots, \vect{\alpha}_n \text{ 线性无关}
		$$
		于是得证。
	\end{proof}
\end{problem}

\begin{problem}
	第九章习题 13

	\begin{proof}
		使用数学归纳法。首先 $n = 1$ 时命题显然成立。下面假设 $n - 1$ 时命题成立,那么 $n$ 时:

		对于一个 $n$ 阶上三角正交矩阵 $\matr{A}$,前 $n - 1$ 个列向量两两正交,第 $n$ 行的前 $n - 1$ 个元素均为 $0$,因此可以推得前 $n - 1$ 行、前 $n - 1$ 列组成的子矩阵是正交矩阵,根据归纳假设,它也是对角矩阵。因此 $\matr{A}$ 可以写为:
		$$
		\matr{A} = \begin{bmatrix}
			\lambda_1 & & & & a_1 \\
			& \lambda_2 & & & a_2 \\
			& & \ddots & & \vdots \\
			& & & \lambda_{n-1} & a_{n-1} \\
			0 & 0 & \cdots & 0 & a_n
		\end{bmatrix}
		$$
		又因为 $\matr{A}$ 的行向量两两正交,容易得到:
		$$
		\forall\, i, j \in \{1, 2, \dots, n\}: a_i a_j = 0
		$$
		而正交矩阵不能出现全 $0$ 的行,因此 $a_n \neq 0$,于是 $a_1 = a_2 = \cdots = a_{n-1} = 0$。因此 $\matr{A}$ 是对角矩阵。再根据 $\det \matr{A} = \pm 1$ 易证 $a_n = \pm 1$。
		
		至此命题得证。
	\end{proof}
\end{problem}

\begin{problem}
	第九章习题 14

	\begin{proof}
		\begin{itemize}
			\item[\textbf{1)}] 先证明存在性:设
			$$
			\matr{A} = \begin{bmatrix}
				\vect{\alpha}_1 & \vect{\alpha}_2 & \cdots & \vect{\alpha}_n
			\end{bmatrix}
			$$
			那么对 $\vect{\alpha}_1, \vect{\alpha}_2, \dots, \vect{\alpha}_n$ 做 Schmidt 正交化:
			$$
			\vect{\eta}_k = \vect{\alpha}_k - \sum_{j=1}^{k-1} \frac{(\vect{\alpha}_k, \vect{\alpha}_j)}{(\vect{\alpha}_j, \vect{\alpha}_j)} \vect{\eta}_j
			$$
			那么很容易得到 $\vect{\eta}_k$ 可以由 $\vect{\alpha}_1, \vect{\alpha}_2, \dots, \vect{\alpha}_k$ 线性表示,也就是说:
			$$
			\matr{A} = \begin{bmatrix}
				\vect{\eta}_1 & \vect{\eta}_2 & \cdots & \vect{\eta}_n
			\end{bmatrix} \begin{bmatrix}
				t_{1,1} & t_{1,2} & \cdots & t_{1,n} \\
				0 & t_{2,2} & \cdots & t_{2,n} \\
				\vdots & \vdots & \ddots & \vdots \\
				0 & 0 & \cdots & t_{n,n}
			\end{bmatrix}
			$$
			那么就得到 $\matr{A} = \matr{Q} \matr{T}$。

			再证明唯一性:若存在
			$$
			\matr{A} = \matr{Q}_1 \matr{T}_1 = \matr{Q}_2 \matr{T}_2
			$$
			那么:
			$$
			\matr{Q}_2^{-1} \matr{Q}_1 = \matr{T}_2 \matr{T}_1^{-1} \
			$$
			等式左侧是正交矩阵,右侧是上三角矩阵,根据习题 13 的结论,它们一定是对角矩阵,且对角元为 $\pm 1$。

			而 $\matr{T}_1, \matr{T}_2^{-1}$ 对角元均为正,因此:
			$$
			\matr{Q}_2^{-1} \matr{Q}_1 = \matr{T}_2 \matr{T}_1^{-1} = \matr{E}
			$$
			于是自然得到 $\matr{Q}_1 = \matr{Q}_2,\ \matr{T}_1 = \matr{T}_2$。

			\item[\textbf{2)}] 正定矩阵可以化为标准型 $\matr{E}$。下面构造一种算法将其化为 $\matr{E}$:
			
			首先,对于任何正定矩阵,角元必定为正,否则容易构造一组数代入得到非正数。

			假设已经将矩阵化为如下形式:
			$$
			\matr{A} = \begin{bmatrix}
				1 & & & & & \\
				& 1 & & & & \\
				& & \ddots & & & \\
				& & & 1 & & \\
				& & & & a_{m,m} & a_{m,m+1} & \cdots & a_{m,n} \\
				& & & & a_{m+1,m} & a_{m+1,m+1} & \cdots & a_{m+1,n} \\
				& & & & \vdots & \vdots & \ddots & \vdots \\
				& & & & a_{n,m} & a_{n,m+1} & \cdots & a_{n,n}
			\end{bmatrix}
			$$
			那么构造合同变换矩阵:
			$$
			\matr{C} = \begin{bmatrix}
				1 & & & & & \\
				& 1 & & & & \\
				& & \ddots & & & \\
				& & & 1 & & \\
				& & & & \frac{1}{a_{m,m}} & -\frac{a_{m,m+1}}{a_{m,m}} & \cdots & -\frac{a_{m,n}}{a_{m,m}} \\
				& & & & & 1 & & \\
				& & & & & & \ddots & \\
				& & & & & & & 1
			\end{bmatrix}
			$$
			那么 $\transpose{\matr{C}} \matr{A} \matr{C}$ 的变换效果就是用 $a_{m,m}$ 来将第 $m$ 行、第 $m$ 列都消成 $0$,再把 $a_{m,m}$ 归一。

			将上述过程进行 $n$ 次,就成功使用合同变换将 $\matr{A}$ 化为 $\matr{E}$。而每次变换矩阵都是上三角矩阵,因此总的矩阵也是上三角矩阵。即 $\matr{E} = \transpose{\matr{T}} \matr{A} \matr{T} \Rightarrow \matr{A} = \transpose{\ab(\matr{T}^{-1})} \matr{T}^{-1}$。
		\end{itemize}
	\end{proof}
\end{problem}

\begin{problem}
	第九章习题 15

	\begin{proof}
		\begin{enumerate}
			\item[\textbf{1)}]
			$$
			\begin{aligned}
				(\mathscr{A}(\vect{\alpha}), \mathscr{A}(\vect{\beta})) & = (\vect{\alpha} - 2 (\vect{\eta}, \vect{\alpha}) \vect{\eta}, \vect{\beta} - 2 (\vect{\eta}, \vect{\beta}) \vect{\eta}) \\
				& = (\vect{\alpha}, \vect{\beta}) - 2 (\vect{\eta}, \vect{\alpha}) (\vect{\eta}, \vect{\beta}) - 2 (\vect{\eta}, \vect{\beta}) (\vect{\eta}, \vect{\alpha}) + 4 (\vect{\eta}, \vect{\alpha}) (\vect{\eta}, \vect{\beta}) (\vect{\eta}, \vect{\eta}) \\
				& = (\vect{\alpha}, \vect{\beta})
			\end{aligned}
			$$
			其中因为 $\vect{\eta}$ 是单位向量,所以 $(\vect{\eta}, \vect{\eta}) = 1$。
		
			\item[\textbf{2)}] 设 $\mathscr{A}$ 在某组标准正交基下的矩阵为 $\matr{A}$,$\eta$ 在这组基下坐标为 $\vect{\xi}$,那么:
			$$
			\matr{A} \vect{x} = \vect{x} - 2 \vect{\xi} (\transpose{\vect{\xi}} \vect{x}) \Rightarrow \matr{A} = \matr{E} - 2 \vect{\xi} \transpose{\vect{\xi}}
			$$
			因为基变换不影响 $\det \matr{A}$,不妨取一个巧妙的基使得 $\vect{\xi} = \transpose{[1, 0, \dots, 0]}$。那么 $\det \matr{A} = \det \diag(-1, 1, \dots, 1) = -1$。
			
			故 $\mathscr{A}$ 是第二类的。

			\item[\textbf{3)}] 取 $V_1$ 的一组标准正交基
			$$
			\vect{\varepsilon}_1, \vect{\varepsilon}_2, \dots, \vect{\varepsilon}_{n-1}
			$$
			再扩充一个向量,使其成为 $V$ 的一组标准正交基
			$$
			\vect{\varepsilon}_1, \vect{\varepsilon}_2, \dots, \vect{\varepsilon}_{n-1}, \vect{\varepsilon}_n
			$$
			那么可得 $\mathscr{A}$ 在这组基下的矩阵为
			$$
			\matr{A} = \begin{bmatrix}
				1 & & & & a_1 \\
				& 1 & & & a_2 \\
				& & \ddots & & \vdots \\
				& & & 1 & a_{n-1} \\
				& & & & x
			\end{bmatrix}
			$$
			因为 $\matr{A}$ 是正交矩阵,根据行向量两两正交、列向量两两正交,可得 $a_1 = a_2 = \cdots = a_{n-1} = 0$。再根据 $\det \matr{A} = \pm 1$,可得 $x = \pm 1$。若 $x = 1$ 那么 $\vect{\varepsilon}_n$ 的特征值也为 $1$,这与 $\dim V_1 = n - 1$ 矛盾。因此 $x = -1$。

			那么,对于任意向量 $\vect{\alpha} = \sum_{i=1}^n a_i \vect{\varepsilon}_i$,我们有:
			$$
			\mathscr{A}(\vect{\alpha}) = \sum_{i=1}^n a_i \mathscr{A}(\vect{\varepsilon}_i) = \sum_{i=1}^{n-1} a_i \vect{\varepsilon}_i - a_n \vect{\varepsilon}_n = \vect{\alpha} - 2 (\vect{\varepsilon}_n, \vect{\alpha}) \vect{\varepsilon}_n
			$$
			也就是说 $\vect{\varepsilon}_n$ 就是镜面反射对应的 $\vect{\eta}$,$\mathscr{A}$ 就是一个镜面反射。可喜可贺,可喜可贺。
		\end{enumerate}
	\end{proof}
\end{problem}