\section{线性方程组 V \& 矩阵 I}

\subsection{线性方程组解的结构}

线性方程组 $\matr{A} \vect{x} = \vect{b}$ 的一个解是一个向量,集合 $V = \{\vect{\gamma} \in P^n: \matr{A} \vect{\gamma} = \vect{b}\}$ 称为方程组的解集。

对于齐次线性方程组而言,解的线性组合仍然是方程组的解。

\begin{definition}
	齐次线性方程组的一组解 $\vect{\eta}_1, \vect{\eta}_2, \cdots, \vect{\eta}_r$ 称为一个\textbf{基础解系},当且仅当:

	\begin{itemize}
		\item 方程组的所有解都可以被 $\vect{\eta}_1, \vect{\eta}_2, \cdots, \vect{\eta}_r$ 线性表出;
		\item $\vect{\eta}_1, \vect{\eta}_2, \cdots, \vect{\eta}_r$ 是线性无关的;
	\end{itemize}

	\sout{其实就是解空间的基}
\end{definition}

\begin{theorem} \label{theorem:齐次线性方程组解的结构}
	齐次线性方程组 $\matr{A} \vect{x} = \vect{0}$ 存在基础解系,且其中包含 $n - \rank \matr{A}$ 个解。

	\sout{其实就是说解集是一个线性空间}

	\begin{proof}
		记 $r = \rank \matr{A}$。消元后得到 $n - r$ 个自由元,那么自由元的一组取值和线性方程组的一组解一一对应。于是用
		$$
		\cvec{1, 0, \vdots, 0}, \cvec{0, 1, \vdots, 0}, \cdots, \cvec{0, 0, \vdots, 1}
		$$
		带入自由元即可得到一组基础解系:
		$$
		\vect{\eta}_1 = \cvec{c_{1,1}, c_{1,2}, \vdots, c_{1,r}, 1, 0, \vdots, 0},
		\vect{\eta}_2 = \cvec{c_{2,1}, c_{2,2}, \vdots, c_{2,r}, 0, 1, \vdots, 0},
		\cdots,
		\vect{\eta}_r = \cvec{c_{r,1}, c_{r,2}, \vdots, c_{r,r}, 0, 0, \vdots, 1},
		$$
		显然这些向量线性无关。而方程组的每组解 $\rvec{t_1, \dots, t_r, t_{r+1}, \dots, t_n}$ 都可以表示为
		$$
		\sum_{i=r+1}^n t_i \vect{\eta}_i
		$$
		因此 $\vect{\eta}_1, \vect{\eta}_2, \cdots, \vect{\eta}_r$ 是一组基础解系。
	\end{proof}
\end{theorem}

有基础解系的定义可以看出,任何与基础解系等价的线性无关向量组都是一个基础解系。

再看一般的线性方程组 $\matr{A} \vect{x} = \vect{b}$。我们称其对应的齐次线性方程组 $\matr{A} \vect{x} = \vect{b}$ 是它的\textbf{导出组}。那么可以看出:

\begin{itemize}
	\item 原方程组的两个解之差是导出组的解;
	\item 原方程组的一个解加上导出组的一个解是原方程组的解。
\end{itemize}

总结起来就是:

\begin{theorem}
	设 $\matr{A} \vect{x} = \vect{b}$ 有一个特解 $\vect{\gamma}_0$,那么该方程组的所有解可以表示为
	$$
	\vect{\gamma} = \vect{\gamma}_0 + \vect{\eta}
	$$
	其中 $\vect{\eta}$ 是导出组的一个解。
\end{theorem}

再根据定理 \ref{theorem:齐次线性方程组解的结构},设 $\vect{\eta}_1, \vect{\eta}_2, \dots, \vect{\eta}_r$ 是导出组的基础解系,那么方程组的所有解都能表示为
$$
\vect{\gamma} = \vect{\gamma}_0 + \sum_{i=1}^r k_i \vect{\eta}_i
$$
\sout{其实就是说解集是导出组解空间的一个仿射。}

\begin{corollary}
	若线性方程组有解,那么有唯一解的充要条件是导出组只有零解。
\end{corollary}

\subsection{矩阵的概念和运算}

用数域 $P$ 中的数排列成的 $n \times m$ 的数表称为 $P$ 上的 $n \times m$ 矩阵。这些矩阵全体组成的集合记作 $P^{n \times m}$。

一个矩阵可以有下面的记法:
$$
\begin{bmatrix}
	a_{1,1} & a_{1,2} & \cdots & a_{1,m} \\
	a_{2,1} & a_{2,2} & \cdots & a_{2,m} \\
	\vdots & \vdots & \ddots & \vdots \\
	a_{n,1} & a_{n,2} & \cdots & a_{n,m}
\end{bmatrix} = [a_{i,j}] = \matr{A}
$$

\begin{definition}[矩阵的加法]
	若 $\matr{A} = [a_{i,j}]_{n \times m},\ \matr{B} = [b_{i,j}]_{n \times m}$,那么矩阵 $\matr{C} = [a_{i,j} + b_{i,j}]_{n \times m}$ 称为 $\matr{A}$ 和 $\matr{B}$ 的\textbf{和}。记作:
	$$
	\matr{C} = \matr{A} + \matr{B}
	$$
\end{definition}

\begin{definition}[零矩阵]
	记元素全为 $0$ 的矩阵为零矩阵,记作 $\matr{O}$。	
\end{definition}

\begin{definition}[负矩阵]
	记矩阵 $[-a_{i,j}]_{n \times m}$ 为 $\matr{A}$ 的负矩阵,记作 $-\matr{A}$。
\end{definition}

\begin{definition}[矩阵的减法]
	$$
	\matr{A} - \matr{B} = \matr{A} + (-\matr{B})
	$$
\end{definition}

根据矩阵加法的定义可以得到下面的性质:

\begin{property}[矩阵的加法的性质]
	\ 
	\begin{itemize}
		\item 矩阵的加法满足交换律和结合律;
		\item $\matr{A} + \matr{O} = \matr{O} + \matr{A} = \matr{A}$;
		\item $\matr{A} + (-\matr{A}) = \matr{O}$;
		\item $\rank(\matr{A} + \matr{B}) \le \rank\matr{A} + \rank\matr{B}$。
	\end{itemize}
\end{property}

\begin{definition}[矩阵的乘法]
	若 $\matr{A} = [a_{i,k}]_{s \times n},\ \matr{B} = [b_{k,j}]_{n \times m}$,那么记
	$$
	c_{i,j} = \sum_{k=1}^n a_{i,k} b_{k,j}
	$$
	则 $\matr{C} = [c_{i,j}]_{s \times m}$ 称为 $\matr{A}$ 和 $\matr{B}$ 的\textbf{乘积},记作
	$$
	\matr{C} = \matr{A} \matr{B} \text{ 或 } \matr{C} = \matr{A} \times \matr{B}
	$$
\end{definition}

\begin{definition}[单位矩阵]
	矩阵 $[[i=j]]_{n \times n}$ 称为 $n$ 阶单位矩阵,记作 $\matr{I}_n$。
\end{definition}

显然单位矩阵是矩阵乘法的幺元,有 $\matr{I}_n \times \matr{A}_{n \times m} = \matr{A}_{n \times m},\ \matr{B}_{n \times m} \times \matr{I}_m = \matr{B}_{n \times m}$。

矩阵的乘法满足以下性质:

\begin{property}[矩阵的乘法的性质]
	\ 
	\begin{itemize}
		\item 矩阵的乘法满足结合律;
		\item 矩阵的乘法\textbf{不}满足交换律;
		\item 矩阵乘法和矩阵加法满足左/右分配律。
	\end{itemize}
\end{property}