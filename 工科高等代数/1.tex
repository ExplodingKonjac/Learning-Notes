\section{数域}

\subsection{定义}

\begin{definition}[域的定义]
	如果一个集合 $D$ 和其上的运算 $+,\times$ 满足如下条件:

	\begin{enumerate}
		\item $(D,+)$ 构成阿贝尔群;
		\item $(D,\times)$ 构成阿贝尔群;
		\item $+,\times$ 有分配律。
	\end{enumerate}

	那么 $(D,+,\times)$ 是一个域。
\end{definition}

数域的定义可以在此基础上进一步限制:

\begin{definition}[数域的定义]
	若集合 $P$ 满足 $\{0,1\} \subseteq P \subseteq \mathbb{C}$,且 $(P,+,\times)$ 是数域。

	课本定义:若 $\mathbb{C}$ 的子集 $P$ 包含 $0,1$ 且其上的加减乘除封闭,那么 $P$ 是数域。
\end{definition}

要证明一个集合是数域,只需要验证加减乘除的封闭性即可。

\subsection{性质}

\begin{conclusion}
	对于域 $P$,若 $a,b \in P \land ab=0$,那么 $a=0 \lor b=0$。

	\begin{proof}
		先证明 $0 \times x=0$:随便找一个 $y \neq 0$,那么等价于证明 $(y-y) \times x=0$,也就是证明 $y \times x-y \times x=0$,这是根据定义显然的。

		回到原问题。不妨设有 $a \neq 0$,我们希望证明 $b=0$。那么找到 $a$ 的乘法逆元 $a^{-1}$,则有 $a^{-1} \times a \times b=a^{-1} \times 0=0$。根据结合律得到 $1 \times b=0$,再由定义 $b=0$。
	\end{proof}
\end{conclusion}

\begin{conclusion}
	若集合 $P$ 是数域,那么 $\mathbb{Q} \subseteq P$。

	\begin{proof}
		首先根据定义,$0 \in P,\ 1 \in P$。根据加减法封闭,推出 $\mathbb{Z} \subseteq P$,再由除法封闭,推出 $\mathbb{Q} \subseteq P$。
	\end{proof}
\end{conclusion}

\subsection{拓展}

\begin{theorem}[林德曼定理]
	$\pi$ 是超越数。即对于任意复数域上的非零多项式 $f(x)$,有 $f(\pi) \neq 0$。

	\begin{proof}
		我不会。
	\end{proof}
\end{theorem}