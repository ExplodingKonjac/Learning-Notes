\section{欧几里德空间}

\subsection{欧几里德空间}

\begin{definition}[内积与欧几里德空间]
	设 $V$ 是 $\mathbb{R}$ 上的线性空间,若能在 $V$ 上定义一个二元实函数,称为\textbf{内积},记作 $(\vect{\alpha}, \vect{\beta})$,其满足:
	\begin{itemize}
		\item $(\vect{\alpha}, \vect{\beta}) = (\vect{\beta}, \vect{\alpha})$;
		\item $(k \vect{\alpha}, \vect{\beta}) = k (\vect{\alpha}, \vect{\beta})$;
		\item $(\vect{\alpha} + \vect{\gamma}, \vect{\beta}) = (\vect{\alpha}, \vect{\beta}) + (\vect{\gamma}, \vect{\beta})$;
		\item $(\vect{\alpha}, \vect{\alpha}) \ge 0$。
	\end{itemize}
	那么就称 $V$ 为一个\textbf{欧几里得空间}。
\end{definition}

\begin{example}
	如在线性空间 $\mathbb{R}^n$ 中,我们可以定义内积:
	$$
	\ab(\transpose{[x_1, x_2, \dots, x_n]}, \transpose{[y_1, y_2, \dots, y_n]}) = \sum_{i=1}^n x_i y_i
	$$
	那么 $\mathbb{R}^n$ 就成一个欧几里得空间。

	这种内积被称为标准内积。
\end{example}

\begin{example}
	在 $[a, b]$ 上的所有实连续函数构成的空间 $C[a, b]$ 中,定义内积:
	$$
	(f, g) = \int_a^b f(x) g(x) \,\dif{x}
	$$
	那么 $C[a, b]$ 就成一个欧几里得空间。
\end{example}

由于 $(\vect{\alpha}, \vect{\alpha}) \ge 0$,我们可以对其开平方根,于是可以定义欧几里德空间中向量的长度:

\begin{definition}
	对于欧几里德空间 $V$ 中的向量 $\vect{\alpha}$,定义其长度为:
	$$
	\abs*{\vect{\alpha}} = \sqrt{(\vect{\alpha}, \vect{\alpha})}
	$$
\end{definition}

\begin{property}
	设 $V$ 是一个欧几里德空间,那么: 
	$$
	\forall\,k \in \mathbb{R}, \vect{\alpha} \in V: \abs*{k \vect{\alpha}} = |k| |\vect{\alpha}|
	$$
\end{property}

\begin{property}[柯西-布涅科夫斯基不等式]
	设 $V$ 是一个欧几里德空间,那么:
	$$
	\forall\,\vect{\alpha}, \vect{\beta} \in V: |(\vect{\alpha}, \vect{\beta})| \le \abs*{\vect{\alpha}} \abs*{\vect{\beta}}
	$$
	等号在 $\vect{\alpha}, \vect{\beta}$ 线性相关时取等。
	\begin{proof}
		不妨先设 $\vect{\beta} \neq \vect{0}$。那么我们要证:
		$$
		(\vect{\alpha}, \vect{\beta})^2 \le (\vect{\alpha}, \vect{\alpha})(\vect{\beta}, \vect{\beta})
		$$
		令 $\vect{\gamma} = \vect{\alpha} + t \vect{\beta}$,那么:
		$$
		(\vect{\gamma}, \vect{\gamma}) = (\vect{\alpha}, \vect{\alpha})^2 + 2 (\vect{\alpha}, \vect{\beta}) t + (\vect{\beta}, \vect{\beta}) t^2 \ge 0
		$$
		而 $(\vect{\beta}, \vect{\beta}) > 0$,那么:
		$$
		\Delta = \ab(2(\vect{\alpha}, \vect{\beta}))^2 - 4(\vect{\alpha}, \vect{\alpha})(\vect{\beta}, \vect{\beta}) \le 0
		$$
		即证。
	\end{proof}
\end{property}

在空间 $\mathbb{R}^n$ 中,上述不等式就具有下面的形式:
$$
\ab(\sum_{i=1}^n a_i b_i)^2 \le \ab(\sum_{i=1}^n a_i^2) \ab(\sum_{i=1}^n b_i^2)
$$

此外,从柯西不等式还能推导得到三角不等式:
$$
|\vect{\alpha} + \vect{\beta}| \le |\vect{\alpha}| + |\vect{\beta}|
$$

\begin{definition}[单位向量]
	在欧几里德空间中,长度为 $1$ 的向量称为单位向量。

	对于任意非零向量 $\vect{\alpha}$,可知 $\frac{\vect{\alpha}}{|\vect{\alpha}|}$,这个过程称为向量的单位化。
\end{definition}

\begin{definition}[夹角]
	在欧几里德空间中,定义两向量 $\vect{\alpha}, \vect{\beta}$ 的夹角:
	$$
	\ab\langle \vect{\alpha}, \vect{\beta} \rangle = \arccos \frac{|(\vect{\alpha}, \vect{\beta})|}{|\vect{\alpha}| |\vect{\beta}|}
	$$
\end{definition}

\subsection{正交向量组}

\begin{definition}[正交]
	若 $V$ 是欧几里德空间,$\vect{\alpha}, \vect{\beta} \in V$ 满足
	$$
	(\vect{\alpha}, \vect{\beta}) = 0
	$$
	或等价地满足
	$$
	\ab\langle \vect{\alpha}, \vect{\beta} \rangle = \frac{\pi}{2}
	$$
	那么称 $\vect{\alpha}, \vect{\beta}$ 正交。记作 $\vect{\alpha} \bot \vect{\beta}$。
\end{definition}

\begin{property}[勾股定理]
	若欧几里德空间 $V$ 中的向量 $\vect{\alpha}_1, \vect{\alpha}_2, \dots, \vect{\alpha}_n$ 两两正交,那么:
	$$
	\abs*{\sum_{i=1}^n \vect{\alpha}_i}^2 = \sum_{i=1}^n \abs*{\vect{\alpha}_i}^2
	$$
\end{property}

对于欧几里德空间 $V$ 的一组基 $\vect{\varepsilon}_1, \vect{\varepsilon}_2, \dots, \vect{\varepsilon}_n$,若:
$$
\vect{\alpha} = \sum_{i=1}^n x_i \vect{\varepsilon}_i,\ \vect{\beta} = \sum_{i=1}^n y_i \vect{\varepsilon}_i
$$
那么可以发现:
$$
(\vect{\alpha}, \vect{\beta}) = \sum_{i=1}^n \sum_{j=1}^n x_i y_j (\varepsilon_i, \varepsilon_j) = \transpose{\vect{x}} \matr{A} \vect{y}
$$
其中 $\matr{A} = [(\varepsilon_i, \varepsilon_j)]$。

\begin{definition}[度量矩阵]
	对于欧几里德空间 $V$ 下的一组基 $\vect{\varepsilon}_1, \vect{\varepsilon}_2, \dots, \vect{\varepsilon}_n$,定义
	$$
	\matr{A} = \cvec{\vect{\varepsilon}_1, \vect{\varepsilon}_2, \vdots, \vect{\varepsilon}_n} \cdot \rvec{\vect{\varepsilon}_1, \vect{\varepsilon}_2, \dots, \vect{\varepsilon}_n} = \begin{bmatrix}
		(\vect{\varepsilon}_1, \vect{\varepsilon}_1) & (\vect{\varepsilon}_1, \vect{\varepsilon}_2) & \cdots & (\vect{\varepsilon}_1, \vect{\varepsilon}_n) \\
		(\vect{\varepsilon}_2, \vect{\varepsilon}_1) & (\vect{\varepsilon}_2, \vect{\varepsilon}_2) & \cdots & (\vect{\varepsilon}_2, \vect{\varepsilon}_n) \\
		\vdots & \vdots & \ddots & \vdots \\
		(\vect{\varepsilon}_n, \vect{\varepsilon}_1) & (\vect{\varepsilon}_n, \vect{\varepsilon}_2) & \cdots & (\vect{\varepsilon}_n, \vect{\varepsilon}_n)
	\end{bmatrix}
	$$
	为 $V$ 在这组基下的\textbf{度量矩阵}。
\end{definition}

当知道度量矩阵后,任意两向量的内积都可以根据坐标进行计算。因而度量矩阵完全确定内积。

\begin{property}
	对于欧几里德空间中的两组基:
	$$
	\begin{gathered}
		\vect{\varepsilon}_1, \vect{\varepsilon}_2, \dots, \vect{\varepsilon}_n \\
		\vect{\eta}_1, \vect{\eta}_2, \dots, \vect{\eta}_n
	\end{gathered}
	$$
	若其满足:
	$$
	\rvec{\vect{\eta}_1, \vect{\eta}_2, \dots, \vect{\eta}_n} = \rvec{\vect{\varepsilon}_1, \vect{\varepsilon}_2, \dots, \vect{\varepsilon}_n} \matr{C}
	$$
	且 $\matr{A}$ 是 $\vect{\varepsilon}_1, \vect{\varepsilon}_2, \dots, \vect{\varepsilon}_n$ 的度量矩阵,$\matr{B}$ 是 $\vect{\eta}_1, \vect{\eta}_2, \dots, \vect{\eta}_n$ 的度量矩阵,那么:
	$$
	\matr{B} = \transpose{\matr{C}} \matr{A} \matr{C}
	$$
\end{property}

也就是说,欧几里德空间下的基变换对应于度量矩阵的合同变换。

\begin{definition}[正交向量组]
	若欧几里德空间 $V$ 下的一组非零向量 $\vect{\alpha}_1, \vect{\alpha}_2, \dots, \vect{\alpha}_n$ 满足其两两正交,那么称这组向量为\textbf{正交向量组}。
\end{definition}

\begin{property}
	正交向量组是线性无关的。
\end{property}

\begin{definition}[标准正交基]
	对于欧几里德空间 $V$ 下的一组基,若它的向量两两正交且长度为 $1$,那么称这组基为\textbf{标准正交基}。
\end{definition}

因为任意正定矩阵都合同于单位矩阵,因此可以断言标准正交基是存在的。

\begin{property}
	在标准正交基下,$\vect{\alpha}, \vect{\beta}$ 的坐标分别为 $\vect{x}, \vect{y}$,那么:
	$$
	(\vect{\alpha}, \vect{\beta}) = \sum_{i=1}^n x_i y_i = \transpose{\vect{x}} \vect{y}
	$$
\end{property}

\begin{theorem}
	对于 $V$ 的一组正交向量组 $\vect{\alpha}_1, \vect{\alpha}_2, \dots, \vect{\alpha}_m$,且 $|\vect{\alpha}_i| = 1$,其可以扩充成一组标准正交基。
\end{theorem}

\begin{theorem}
	对于欧几里德空间 $V$ 中一个线性无关向量组 $\vect{\varepsilon}_1, \vect{\varepsilon}_2, \dots, \vect{\varepsilon}_n$,存在一个正交向量组 $\vect{\eta}_1, \vect{\eta}_2, \dots, \vect{\eta}_n$ 使得
	$$
	L(\vect{\varepsilon}_1, \vect{\varepsilon}_2, \dots, \vect{\varepsilon}_n) = L(\vect{\eta}_1, \vect{\eta}_2, \dots, \vect{\eta}_n)
	$$

	\begin{proof}
		首先,令 $\vect{\eta}_1 = \vect{\varepsilon}_1$。由于 $\vect{\varepsilon}_1$ 是非零向量,因此 $|\vect{\eta}_1| \ne 0$。对于 $k=2, 3, \dots, n$,我们构造向量 $\vect{\eta}_k$ 如下:
		$$
		\vect{\eta}_k = \vect{\varepsilon}_k - \sum_{j=1}^{k-1} \frac{(\vect{\varepsilon}_k, \vect{\eta}_j)}{|\vect{\eta}_j|^2} \vect{\eta}_j
		$$
		其中 $\frac{(\vect{\varepsilon}_k, \vect{\eta}_j)}{|\vect{\eta}_j|^2}$ 是将 $\vect{\varepsilon}_k$ 在 $\vect{\eta}_j$ 方向上的投影系数。

		下面我们证明 $\vect{\eta}_k$ 与 $\vect{\eta}_1, \dots, \vect{\eta}_{k-1}$ 正交。我们采用归纳法。假设 $\vect{\eta}_1, \dots, \vect{\eta}_{k-1}$ 已经两两正交。对于任意 $i \in \{1, 2, \dots, k-1\}$,计算内积:
		$$
		\begin{aligned}
			(\vect{\eta}_k, \vect{\eta}_i) &= \left(\vect{\varepsilon}_k - \sum_{j=1}^{k-1} \frac{(\vect{\varepsilon}_k, \vect{\eta}_j)}{|\vect{\eta}_j|^2} \vect{\eta}_j, \vect{\eta}_i\right) \\
			&= (\vect{\varepsilon}_k, \vect{\eta}_i) - \sum_{j=1}^{k-1} \frac{(\vect{\varepsilon}_k, \vect{\eta}_j)}{|\vect{\eta}_j|^2} (\vect{\eta}_j, \vect{\eta}_i)
		\end{aligned}
		$$
		由于 $\vect{\eta}_1, \dots, \vect{\eta}_{k-1}$ 两两正交,所以当 $j \ne i$ 时,$(\vect{\eta}_j, \vect{\eta}_i) = 0$;当 $j=i$ 时,$(\vect{\eta}_j, \vect{\eta}_i) = |\vect{\eta}_i|^2$。
		因此,求和项中只有 $j=i$ 的那一项非零:
		$$
		\sum_{j=1}^{k-1} \frac{(\vect{\varepsilon}_k, \vect{\eta}_j)}{|\vect{\eta}_j|^2} (\vect{\eta}_j, \vect{\eta}_i) = \frac{(\vect{\varepsilon}_k, \vect{\eta}_i)}{|\vect{\eta}_i|^2} (\vect{\eta}_i, \vect{\eta}_i) = \frac{(\vect{\varepsilon}_k, \vect{\eta}_i)}{|\vect{\eta}_i|^2} |\vect{\eta}_i|^2 = (\vect{\varepsilon}_k, \vect{\eta}_i)
		$$
		代回原式,得到:
		$$
		(\vect{\eta}_k, \vect{\eta}_i) = (\vect{\varepsilon}_k, \vect{\eta}_i) - (\vect{\varepsilon}_k, \vect{\eta}_i) = 0
		$$
		这证明了 $\vect{\eta}_k$ 与所有 $\vect{\eta}_1, \dots, \vect{\eta}_{k-1}$ 都是正交的。
		由于 $\vect{\varepsilon}_1, \dots, \vect{\varepsilon}_n$ 线性无关,所以 $\vect{\eta}_k \ne \vect{0}$。

		于是我们就得到了一个正交向量组 $\vect{\eta}_1, \dots, \vect{\eta}_n$ 满足
		$$
		L(\vect{\varepsilon}_1, \dots, \vect{\varepsilon}_k) = L(\vect{\eta}_1, \dots, \vect{\eta}_k)
		$$

		这种构造称为 \textbf{Schmidt 正交化方法}。
	\end{proof}
\end{theorem}