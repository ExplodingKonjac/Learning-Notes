\section{欧几里德空间 III}

\subsection{对称变换}

\begin{definition}[对称变换]
	对于欧几里德空间 $V$ 上的线性变换 $\mathscr{A}$,若:
	$$
	\forall\,\vect{v}, \vect{w} \in V: (\mathscr{A}(\vect{v}), \vect{w}) = (\vect{v}, \mathscr{A}(\vect{w}))
	$$
	则称 $\mathscr{A}$ 为\textbf{对称变换}。
\end{definition}

\begin{property}
	对称变换都是线性变换。

	\begin{proof}
		对于任意 $\vect{\gamma} \in V$:
		$$
		\begin{aligned}
			(\mathscr{A}(k_1 \vect{\alpha} + k_2 \vect{\beta}), \vect{\gamma}) & = (k_1 \vect{\alpha} + k_2 \vect{\beta}, \mathscr{A}(\vect{\gamma})) \\
			& = k_1 (\vect{\alpha}, \mathscr{A}(\vect{\gamma})) + k_2 (\vect{\beta}, \mathscr{A}(\vect{\gamma})) \\
			& = k_1 (\mathscr{A}(\vect{\alpha}), \vect{\gamma}) + k_2 (\mathscr{A}(\vect{\beta}), \vect{\gamma}) \\
			& = (k_1 \mathscr{A}(\vect{\alpha}) + k_2 \mathscr{A}(\vect{\beta}), \vect{\gamma})
		\end{aligned}
		$$
		这说明 $\mathscr{A}(k_1 \vect{\alpha} + k_2 \vect{\beta}) = k_1 \mathscr{A}(\vect{\alpha}) + k_2 \mathscr{A}(\vect{\beta})$,
	\end{proof}
\end{property}

\begin{property}
	在标准正交基下,对称变换的矩阵是实对称矩阵。

	\begin{proof}
		设对称变换 $\mathscr{A}$ 在标准正交基 $\vect{\varepsilon}_1, \vect{\varepsilon}_2, \dots, \vect{\varepsilon}_n$ 下的矩阵为 $\matr{A} = [a_{i,j}]$,那么:
		$$
		\begin{aligned}
			(\mathscr{A}(\vect{\varepsilon}_j), \vect{\varepsilon}_i) & = \ab(\sum_{i=1}^n a_{k,j} \vect{\varepsilon}_j, \vect{\varepsilon}_i) \\
			& = \ab(a_{i,j} \vect{\varepsilon}_i, \vect{\varepsilon}_i) = a_{i,j}
		\end{aligned}
		$$
		而 $(\mathscr{A}(\vect{\varepsilon}_j), \vect{\varepsilon}_i) = (\vect{\varepsilon}_j, \mathscr{A}(\vect{\varepsilon}_i))$,因此 $a_{i,j} = a_{j,i}$。
	\end{proof}
\end{property}

\begin{lemma}
	实对称矩阵的特征值都是实数。

	\begin{proof}
		设 $\lambda \in \mathbb{C}$ 是实对称矩阵 $\matr{A}$ 的特征值,设其对应的一个特征向量 $\vect{z}$,那么:

		\begin{gather}
			\matr{A} \vect{z} = \lambda \vect{z} \tag{1} \\
			\overline{\matr{A}} \overline{\vect{z}} = \overline{\lambda} \overline{\vect{z}} \tag{2}
		\end{gather}

		而 $\matr{A}$ 是实矩阵,因此 $\overline{\matr{A}} = \matr{A}$,所以 $\overline{\lambda} \overline{\vect{z}} = \matr{A} \overline{\vect{z}} \quad (2.1)$。

		根据 (1) 可以得到:
		$$
		\transpose{\overline{\vect{z}}} \matr{A} \vect{z} = \lambda \transpose{\overline{\vect{z}}} \vect{z}
		$$
		再对 (2.1) 式取转置、右乘 $\vect{z}$ 得到:
		$$
		\transpose{\overline{\vect{z}}} \matr{A} \vect{z} = \overline{\lambda} \transpose{\overline{\vect{z}}} \vect{z}
		$$
		两式联立就得到
		$$
		\lambda \transpose{\overline{\vect{z}}} \vect{z} = \overline{\lambda} \transpose{\overline{\vect{z}}} \vect{z}
		$$
		而 $\vect{z} \neq 0$,因此 $\lambda = \overline{\lambda}$,即 $\lambda$ 是实数。
	\end{proof}
\end{lemma}

\begin{theorem}
	对称变换的不同特征值对应的特征子空间正交。

	\begin{proof}
		设 $\lambda_1, \lambda_2$ 是 $\mathscr{A}$ 的两个不同的特征值,$\vect{\alpha}_1, \vect{\alpha}_2$ 分别是属于它们的一个特征向量。那么有:
		$$
		\mathscr{A}(\vect{\alpha}_1) = \lambda_1 \vect{\alpha}_1, \quad \mathscr{A}(\vect{\alpha}_2) = \lambda_2 \vect{\alpha}_2
		$$
		由对称变换的定义:
		$$
		(\mathscr{A}(\vect{\alpha}_1), \vect{\alpha}_2) = (\vect{\alpha}_1, \mathscr{A}(\vect{\alpha}_2))
		$$
		将特征值的关系代入上式,得到:
		$$
		\lambda_1 (\vect{\alpha}_1, \vect{\alpha}_2) = \lambda_2 (\vect{\alpha}_1, \vect{\alpha}_2)
		$$
		由于 $\lambda_1 \neq \lambda_2$,所以 $(\vect{\alpha}_1, \vect{\alpha}_2) = 0$。即 $\vect{\alpha}_1 \perp \vect{\alpha}_2$。
	\end{proof}
\end{theorem}

\begin{theorem}
	对于任意 $n$ 阶实对称矩阵 $\matr{A}$,存在正交方阵 $\matr{T}$ 使得 $\transpose{\matr{T}} \matr{A} \matr{T}$ 是对角矩阵。

	\begin{proof}
		实对称矩阵可以对应于 $\mathbb{R}^n$ 上的对称变换。

		根据实对称矩阵特征子空间的性质,我们可以通过不断从特征子空间扩展正交基的方式,取出一组标准正交基 $\vect{\varepsilon}_1, \vect{\varepsilon}_2, \dots, \vect{\varepsilon}_n$,使得它们都是 $\matr{A}$ 的特征向量。也即:
		$$
		\matr{A} \rvec{\vect{\varepsilon}_1, \vect{\varepsilon}_2, \dots, \vect{\varepsilon}_n} = \rvec{\vect{\varepsilon}_1, \vect{\varepsilon}_2, \dots, \vect{\varepsilon}_n} \diag(\lambda_1, \lambda_2, \dots, \lambda_n)
		$$
		那么我们取
		$$
		\matr{T} = \rvec{\vect{\varepsilon}_1, \vect{\varepsilon}_2, \dots, \vect{\varepsilon}_n}
		$$
		因为 $\matr{T}$ 正交,$\matr{T}^{-1} = \transpose{\matr{T}}$,因此:
		$$
		\matr{T}^{-1} \matr{A} \matr{T} = \transpose{\matr{T}} \matr{A} \matr{T} = \diag(\lambda_1, \lambda_2, \dots, \lambda_n)
		$$
	\end{proof}
\end{theorem}