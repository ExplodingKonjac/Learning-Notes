\section{线性方程组 III}

\subsection{向量组的秩}

\begin{definition}[极大线性无关组]
	如果一个向量组的某个子集线性无关,并且加入任意一个原向量组中的向量后线性相关,那么这个子集称为原向量组的\textbf{极大线性无关组}。
\end{definition}

极大线性无关组具有下面性质:

\begin{property}
	\ 
	\begin{itemize}
		\item 一个向量组和它所有极大线性无关组等价;
		\item 一个向量组的所有极大线性无关组大小相等。
	\end{itemize}
\end{property}

由此得到向量组的秩的定义:

\begin{definition}[向量组的秩]
	向量组的秩定义为它的极大线性无关组的大小。
\end{definition}

根据定理 \ref{theorem:线性表出与向量组大小关系} 及其推论可知下面的性质:

\begin{property}
	\ 
	\begin{itemize}
		\item $\rank(\vect{\alpha}_1, \vect{\alpha}_2, \dots, \vect{\alpha}_s) \le s$,当且仅当向量组线性无关时取等;
		\item 等价的向量组拥有相同的秩;
		\item 若 $\vect{\alpha}_1, \vect{\alpha}_2, \dots, \vect{\alpha}_s$ 能被 $\vect{\beta}_1, \vect{\beta}_2, \dots, \vect{\beta}_t$ 线性表出,那么 $\rank \vect{\alpha} \le \rank \vect{\beta}$。
	\end{itemize}
\end{property}

\subsection{矩阵的秩}

\begin{definition}[行秩和列秩]
	矩阵的行向量组的秩称为它的行秩,矩阵的列向量组的秩称为它的列秩。
\end{definition}

\begin{theorem}
	矩阵的行秩等于列秩

	\begin{proof}
		设 $\matr{A}$ 的行秩为 $r$,列秩为 $r'$,那么只需要证明 $r \le r'$,由对称性即可得到 $r = r'$。
		
		取出 $\matr{A}$ 的行向量组的任意一个极大无关组 $\vect{\alpha}_1, \vect{\alpha}_2, \dots, \vect{\alpha}_r$,那么方程组
		$$
		\begin{bmatrix}
			\vect{\alpha}_1 & \vect{\alpha}_2 & \cdots & \vect{\alpha}_r
		\end{bmatrix} \vect{x} = \vect{0}
		$$
		只有零解。因此该方程组的系数矩阵(记做 $\matr{B}$)的行秩 $\ge r$(否则取出一个极大无关组,得到等价的方程组有非零解)。

		取出 $\matr{B}$ 的行向量组的一个极大无关组,这个向量组对应 $\matr{A}$ 的一个列向量组保留 $r$ 个分量。而显然线性无关的向量组添加若干个分量后仍然线性无关,因此 $\matr{A}$ 对应的这些列向量也是线性无关的。因此 $r' \ge r$。
	\end{proof}
\end{theorem}

\begin{theorem}
	设 $\matr{A} \in P^{n \times m}$,那么对 $A$ 进行有限次初等变换使其变为 $\matr{I}_r$,那么 $\matr{A}$ 的行秩和列秩都是 $r$。

	\begin{proof}
		经过初等行(列)变换后行(列)向量组仍然等价,因此行(列)秩不变。使用高斯-约旦消元法消成 $\matr{I}_r$ 后行秩和列秩也和原本相同。而显然 $I_r$ 的行秩和列秩都是 $r$。
	\end{proof}
\end{theorem}

于是可以简单给出矩阵的秩的定义:

\begin{definition}[矩阵的秩]
	矩阵的秩定义为它的行秩/列秩。
\end{definition}

显然一个 $\matr{A} \in P^{n \times n}$ 的秩应当 $\le n$。而对此有定理:

\begin{theorem}
	$\rank \matr{A} = n$ 当且仅当 $|A| \neq 0$。

	\begin{proof}
		将 $\matr{A}$ 进行高斯-约旦消元消成 $\matr{I}_r$,在此过程中 $\rank \matr{A}$ 不变,行列式是否为 $0$ 的性质也不变,因此每步都是等价推导。而显然 $\matr{I}_r$ 满足定理,于是得证。
	\end{proof}
\end{theorem}

矩阵的秩与行列式也有联系。我们记矩阵的某 $k$ 行某 $k$ 列的交点组成的矩阵的行列式为一个 \textbf{$k$ 阶子式},那么有:

\begin{theorem}
	矩阵的秩等于矩阵的最高阶非零子式的阶数。

	\begin{proof}
		设 $\rank \matr{A} = r$,最高阶非零子式阶数为 $k$。
		
		$\matr{A}$ 的任意 $r'\ (r' > r)$ 个行向量组都线性相关,那么 $\matr{A}$ 的任意一个 $r'$ 阶子式的行向量组都线性相关,行列式为 $0$。因此 $k \le r$。

		取出 $\matr{A}$ 的一个极大无关组 $\vect{\alpha}_1, \vect{\alpha}_2, \dots, \vect{\alpha}_r$,那么矩阵
		$$
		A' = \begin{bmatrix}
			\vect{\alpha}_1 \\ \vect{\alpha}_2 \\ \vdots \\ \vect{\alpha}_r
		\end{bmatrix}
		$$
		的秩也是 $r$。因此 $\matr{A'}$ 的列秩也是 $r$。取出 $\matr{A'}$ 的列向量组的一个极大无关组,它们构成一个 $\matr{A}$ 的 $r$ 阶非零子式。因此 $k \ge r$。

		综上 $k = r$。
	\end{proof}
\end{theorem}

这也是矩阵的秩的另一个等价定义。
