\section{二次型 II}

\subsection{标准型}

\begin{theorem}
	数域 $P$ 上的任何一个二次型都能经过非退化的线性替换变为标准型,即具有如下矩阵的形式:
	$$
	\matr{D} = \diag(\lambda_1, \lambda_2, \ldots, \lambda_n)
	$$

	\begin{proof}
		使用数学归纳。假设结论对于 $\le n - 1$ 阶二次型均成立,现在要将 $n$ 阶二次型 $\matr{A} = [a_{i,j}]$ 化为标准型,那么:
		\begin{itemize}
			\item 若至少存在一个 $a_{k,k} \neq 0$,那么把 $x_1, x_k$ 交换就得到 $a_{1,1} \neq 0$。于是有
			$$
			\begin{aligned}
				f(\vect{x}) & = a_{1,1} x_{1,1}^2 + 2 \sum_{i=2}^n a_{1,i} x_1 x_i + \sum_{i=2}^n \sum_{j=2}^n a_{i,j} x_i x_j \\
				& = a_{1,1} \ab(x_{1,1}^2 + 2 x_1 \sum_{i=2}^n \frac{a_{1,i} x_i}{a_{1,1}}) + \sum_{i=2}^n \sum_{j=2}^n a_{i,j} x_i x_j \\
				& = a_{1,1} \ab(x_{1,1} + \sum_{i=2}^n \frac{a_{1,i} x_i}{a_{1,1}})^2 - \frac{1}{a_{1,1}} \ab(\sum_{i=2}^n a_{1,i} x_i)^2 + \sum_{i=2}^n \sum_{j=2}^n a_{i,j} x_i x_j
			\end{aligned}
			$$
			于是我们可以进行线性替换
			$$
			\cvec{z_1, z_2, \vdots, z_n} = \begin{bmatrix}
				1 & \frac{a_{1,2}}{a_{1,1}} & \cdots & \frac{a_{1,n}}{a_{1,1}} \\
				0 & 1 & \cdots & 0 \\
				\vdots & \vdots & \ddots & \vdots \\
				0 & 0 & \cdots & 1
			\end{bmatrix} \cvec{x_1, x_2, \vdots, x_n}
			$$
			转换成 $n - 1$ 阶的情形。

			\item 若不存在 $a_{k,k} = 0$,那么设存在一个 $a_{p,q} \neq 0\ (p \neq q)$,那么进行线性替换:
			$$
			\begin{cases}
				z_p = \frac{x_p + x_q}{2} \\
				z_q = \frac{x_p - x_q}{2} \\
				z_i = x_i & , i \neq p \land i \neq q \\
			\end{cases}
			$$
			于是二次型就变为
			$$
			\begin{aligned}
				f(\vect{x}) & = 2 a_{p,q} x_p x_q + \dots \\
				& = 2 a_{p,q} (z_p + z_q) (z_p - z_q) + \dots \\
				& = 2 a_{p,q} (z_p^2 - z_q^2) + \dots
			\end{aligned}
			$$
			转化为第一种情况。
		\end{itemize}

		于是得证。
	\end{proof}
\end{theorem}

把二次型化为标准型的过程实际上就是通过线性替换来进行配方的过程。

\subsection{初等变换法配方}

也可以使用初等变换来将二次型化做标准型。根据配方法我们知道,任意对称矩阵 $\matr{A}$ 都存在可逆矩阵 $\matr{C}$ 使得:
$$
\transpose{\matr{C}} \matr{A} \matr{C} = \diag(\lambda_1, \lambda_2, \dots, \lambda_n)
$$
那么 $\matr{C}$ 可以表示为若干个初等矩阵的乘积:
$$
\matr{C} = \matr{P}_1 \matr{P}_2 \cdots \matr{P}_k
$$
回带到合同变换里就得到:
$$
\transpose{\matr{P}_k} \cdots \transpose{\matr{P}_2} \transpose{\matr{P}_1} \matr{A} \matr{P}_1 \matr{P}_2 \cdots \matr{P}_k = \diag(\lambda_1, \lambda_2, \dots, \lambda_n)
$$

这说明任何对称矩阵都能通过\textbf{相同类型的初等行变换和列变换}变成对角矩阵。具体的构造也很简单,只需要对于每一个 $k = 1, 2, \dots, n$,都用 $a_{k,k}$ 去消掉 $a_{k,k+1}, a_{k, k+2}, \dots, a_{k,n}, a_{k+1,k}, a_{k+2,k}, \dots, a_{n,k}$。类似于求逆的过程。我们可以对增广矩阵
$$
\ab[\begin{array}{c}
	\matr{A} \\
	\hdashline
	\matr{E}
\end{array}]
$$
做整个变换。因为
$$
\begin{bmatrix}
	\transpose{\matr{P}} & \matr{O} \\
	\matr{O} & \matr{E}
\end{bmatrix} \ab[\begin{array}{c}
	\matr{A} \\
	\hdashline
	\matr{E}
\end{array}] = \ab[\begin{array}{c}
	\transpose{\matr{P}} \matr{A} \\
	\hdashline
	\matr{E}
\end{array}],\ \ab[\begin{array}{c}
	\matr{A} \\
	\hdashline
	\matr{E}
\end{array}] \matr{P} = \ab[\begin{array}{c}
	\matr{A} \matr{P} \\
	\hdashline
	\matr{E} \matr{P}
\end{array}] 
$$
因此最后将 $\matr{A}$ 消成对角矩阵后,$\matr{E} = \matr{P}_1, \matr{P}_2, \dots, \matr{P}_n$。

\subsection{作业}

\begin{problem}
	第四章习题 23

	\begin{solution}
		\begin{enumerate}
			\item[\textbf{2)}] 化简得到
			$$
			\matr{X} = \begin{bmatrix}
				1 & 1 & -1 \\
				0 & 2 & 2 \\
				1 & -1 & 0
			\end{bmatrix}^{-1} \begin{bmatrix}
				1 & -1 & 1 \\
				1 & 1 & 0 \\
				2 & 1 & 1
			\end{bmatrix}
			$$
			使用初等变换法求解:
			$$
			\begin{aligned}
				\ab[\begin{array}{ccc:ccc}
					1 & 1 & -1 & 1 & -1 & 1 \\
					0 & 2 & 2 & 1 & 1 & 0 \\
					1 & -1 & 0 & 2 & 1 & 1
				\end{array}]
				\longrightarrow & \ab[\begin{array}{ccc:ccc}
					1 & 1 & -1 & 1 & -1 & 1 \\
					0 & 2 & 2 & 1 & 1 & 0 \\
					0 & -2 & 1 & 1 & 2 & 0
				\end{array}] \\
				\longrightarrow & \ab[\begin{array}{ccc:ccc}
					1 & 1 & -1 & 1 & -1 & 1 \\
					0 & 1 & 1 & \frac{1}{2} & \frac{1}{2} & 0 \\
					0 & -2 & 1 & 1 & 2 & 0
				\end{array}] \\
				\longrightarrow & \ab[\begin{array}{ccc:ccc}
					1 & 0 & -2 & \frac{1}{2} & -\frac{3}{2} & 1 \\
					0 & 1 & 1 & \frac{1}{2} & \frac{1}{2} & 0 \\
					0 & 0 & 3 & 2 & 3 & 0
				\end{array}] \\
				\longrightarrow & \ab[\begin{array}{ccc:ccc}
					1 & 0 & -2 & \frac{1}{2} & -\frac{3}{2} & 1 \\
					0 & 1 & 1 & \frac{1}{2} & \frac{1}{2} & 0 \\
					0 & 0 & 1 & \frac{2}{3} & 1 & 0
				\end{array}] \\
				\longrightarrow & \ab[\begin{array}{ccc:ccc}
					1 & 0 & 0 & \frac{11}{6} & \frac{1}{2} & 1 \\
					0 & 1 & 0 & -\frac{1}{6} & -\frac{1}{2} & 0 \\
					0 & 0 & 1 & \frac{2}{3} & 1 & 0
				\end{array}]
			\end{aligned}
			$$
			因此
			$$
			\matr{X} = \begin{bmatrix}
				\frac{11}{6} & \frac{1}{2} & 1 \\
				-\frac{1}{6} & -\frac{1}{2} & 0 \\
				\frac{2}{3} & 1 & 0
			\end{bmatrix}
			$$

			\item[\textbf{2)}] 化简得到
			$$
			\matr{X} = \begin{bmatrix}
				1 & -1 & 1 \\
				1 & 1 & 0 \\
				2 & 1 & 1
			\end{bmatrix} \begin{bmatrix}
				1 & 1 & -1 \\
				0 & 2 & 2 \\
				1 & -1 & 0
			\end{bmatrix}^{-1}
			$$
			由于是右乘,我们可以用初等列变换来解决。但是为了符合平时习惯,我们可以求解 $\transpose{\matr{X}}$:
			$$
			\begin{aligned}
				\ab[\begin{array}{ccc:ccc}
					1 & 0 & 1 & 1 & 1 & 2 \\
					1 & 2 & -1 & -1 & 1 & 1 \\
					-1 & 2 & 0 & 1 & 0 & 1
				\end{array}]
				\longrightarrow & \ab[\begin{array}{ccc:ccc}
					1 & 0 & 1 & 1 & 1 & 2 \\
					0 & 2 & -2 & -2 & 0 & -1 \\
					0 & 2 & 1 & 2 & 1 & 3
				\end{array}] \\
				\longrightarrow & \ab[\begin{array}{ccc:ccc}
					1 & 0 & 1 & 1 & 1 & 2 \\
					0 & 1 & -1 & -1 & 0 & -\frac{1}{2} \\
					0 & 2 & 1 & 2 & 1 & 3
				\end{array}] \\
				\longrightarrow & \ab[\begin{array}{ccc:ccc}
					1 & 0 & 1 & 1 & 1 & 2 \\
					0 & 1 & -1 & -1 & 0 & -\frac{1}{2} \\
					0 & 0 & 3 & 4 & 1 & 4
				\end{array}] \\
				\longrightarrow & \ab[\begin{array}{ccc:ccc}
					1 & 0 & 1 & 1 & 1 & 2 \\
					0 & 1 & -1 & -1 & 0 & -\frac{1}{2} \\
					0 & 0 & 1 & \frac{4}{3} & \frac{1}{3} & \frac{4}{3}
				\end{array}] \\
			\end{aligned}
			$$
			$$
			\begin{aligned}
				\longrightarrow & \ab[\begin{array}{ccc:ccc}
					1 & 0 & 0 & -\frac{1}{3} & \frac{2}{3} & \frac{2}{3} \\
					0 & 1 & 0 & \frac{1}{3} & \frac{1}{3} & \frac{5}{6} \\
					0 & 0 & 1 & \frac{4}{3} & \frac{1}{3} & \frac{4}{3}
				\end{array}]
			\end{aligned}
			$$
			因此
			$$
			\matr{X} = \transpose{\ab(\transpose{\matr{X}})} = \begin{bmatrix}
				-\frac{1}{3} & \frac{1}{3} & \frac{4}{3} \\
				\frac{2}{3} & \frac{1}{3} & \frac{1}{3} \\
				\frac{2}{3} & \frac{5}{6} & \frac{4}{3}
			\end{bmatrix}
			$$
		\end{enumerate}
	\end{solution}
\end{problem}

\begin{problem}
	第四章习题 29

	\begin{proof}
		第一行减去第二行左乘 $\matr{B}$ 得到:
		$$
		\begin{bmatrix}
			\matr{E}_m & \matr{B} \\
			\matr{A} & \matr{E}_n
		\end{bmatrix} \longrightarrow \begin{bmatrix}
			\matr{E}_m - \matr{B} \matr{A} & \matr{O} \\
			\matr{A} & \matr{E}_n
		\end{bmatrix}
		$$
		第二列见取第一列右乘 $\matr{B}$ 得到:
		$$
		\begin{bmatrix}
			\matr{E}_m & \matr{B} \\
			\matr{A} & \matr{E}_n
		\end{bmatrix} \longrightarrow \begin{bmatrix}
			\matr{E}_m & \matr{O} \\
			\matr{A} & \matr{E}_n - \matr{A} \matr{B}
		\end{bmatrix}
		$$
		由于过程是初等变换,行列式不变,再结合准对角矩阵的行列式,因此就有
		$$
		\det\ab(\begin{bmatrix}
			\matr{E}_m & \matr{B} \\
			\matr{A} & \matr{E}_n
		\end{bmatrix}) = \det\ab(\matr{E}_n - \matr{A} \matr{B}) = \det\ab(\matr{E}_m - \matr{B} \matr{A})
		$$
	\end{proof}
\end{problem}

\begin{problem}
	第四章习题 30

	\begin{proof}
		由题知:
		$$
		\det(\lambda \matr{E}_n - \matr{A} \matr{B}) = \begin{vmatrix}
			\matr{E}_m & \matr{B} \\
			\matr{A} & \lambda \matr{E}_n
		\end{vmatrix}
		$$
		用第一行减去第二行左乘 $\frac{1}{\lambda} \matr{B}$ 即得:
		$$
		\begin{aligned}
			\det(\lambda \matr{E}_n - \matr{A} \matr{B}) & = \begin{vmatrix}
				\matr{E}_m - \frac{1}{\lambda} \matr{B} \matr{A} & \matr{O} \\
				\matr{A} & \lambda \matr{E}_n
			\end{vmatrix} \\
			& = \lambda^n \det(\matr{E}_m - \frac{1}{\lambda} \matr{B} \matr{A}) \\
			& = \lambda^{n - m} \det(\lambda \matr{E}_m - \matr{B} - \matr{A})
		\end{aligned}
		$$
	\end{proof}
\end{problem}

\begin{problem}
	第五章习题 1 (I)

	\begin{solution}
		\begin{enumerate}
			\item[\textbf{2)}] 令
			$$
			\vect{z} = \begin{bmatrix}
				1 & 1 & 0 \\
				0 & 1 & 2 \\
				0 & 0 & 1
			\end{bmatrix} \vect{x}
			$$
			那么二次型化为 $z_1^2 + z_2^2$。使用矩阵检验:
			$$
			\transpose{\ab(\begin{bmatrix}
				1 & 1 & 0 \\
				0 & 1 & 2 \\
				0 & 0 & 1
			\end{bmatrix}^{-1})} \begin{bmatrix}
				1 & 1 & 0 \\
				1 & 2 & 2 \\
				0 & 2 & 4
			\end{bmatrix} \begin{bmatrix}
				1 & 1 & 0 \\
				0 & 1 & 2 \\
				0 & 0 & 1
			\end{bmatrix}^{-1} = \begin{bmatrix}
				1 & 0 & 0 \\
				0 & 1 & 0 \\
				0 & 0 & 0
			\end{bmatrix}
			$$

			\item[\textbf{6)}] 令
			$$
			\vect{z} = \begin{bmatrix}
				1 & 2 & 2 & 1 \\
				0 & 1 & \frac{3}{2} & \frac{1}{2} \\
				0 & 0 & 1 & 1 \\
				0 & 0 & 0 & 1
			\end{bmatrix} \vect{x}
			$$
			那么二次型化为 $z_1^2 - 2 z_2^2 + \frac{1}{2} z_3^2$。使用矩阵检验:
			$$
			\transpose{\ab(\begin{bmatrix}
				1 & 2 & 2 & 1 \\
				0 & 1 & \frac{3}{2} & \frac{1}{2} \\
				0 & 0 & 1 & 1 \\
				0 & 0 & 0 & 1
			\end{bmatrix}^{-1})} \begin{bmatrix}
				1 & 2 & 2 & 1 \\
				2 & 2 & 1 & 1 \\
				2 & 1 & 0 & 1 \\
				1 & 1 & 1 & 1
			\end{bmatrix} \begin{bmatrix}
				1 & 2 & 2 & 1 \\
				0 & 1 & \frac{3}{2} & \frac{1}{2} \\
				0 & 0 & 1 & 1 \\
				0 & 0 & 0 & 1
			\end{bmatrix}^{-1} = \begin{bmatrix}
				1 & 0 & 0 & 0 \\
				0 & -2 & 0 & 0 \\
				0 & 0 & \frac{1}{2} & 0 \\
				0 & 0 & 0 & 0
			\end{bmatrix}
			$$

			\item[\textbf{7)}] 令
			$$
			\vect{z} = \begin{bmatrix}
				1 & 1 & 0 & 0 \\
				0 & 1 & 1 & 1 \\
				0 & 1 & 0 & 1 \\
				0 & 0 & 0 & 1
			\end{bmatrix} \vect{x}
			$$
			那么二次型化为 $z_1^2 + z_2^2 - z_3^2 + x_4^2$。使用矩阵检验:
			$$
			\transpose{\ab(\begin{bmatrix}
				1 & 1 & 0 & 0 \\
				0 & 1 & 1 & 1 \\
				0 & 1 & 0 & 1 \\
				0 & 0 & 0 & 1
			\end{bmatrix}^{-1})} \begin{bmatrix}
				1 & 1 & 0 & 0 \\
				1 & 1 & 1 & 0 \\
				0 & 1 & 1 & 1 \\
				0 & 0 & 1 & 1
			\end{bmatrix} \begin{bmatrix}
				1 & 1 & 0 & 0 \\
				0 & 1 & 1 & 1 \\
				0 & 1 & 0 & 1 \\
				0 & 0 & 0 & 1
			\end{bmatrix}^{-1} = \begin{bmatrix}
				1 & 0 & 0 & 0 \\
				0 & 1 & 0 & 0 \\
				0 & 0 & -1 & 0 \\
				0 & 0 & 0 & 1
			\end{bmatrix}
			$$
		\end{enumerate}
	\end{solution}
\end{problem}

\begin{problem}
	第五章习题 3
	
	\begin{proof}
		只需证明合同变换能够实现交换对角上两个相邻元素即可。

		注意到存在矩阵
		$$
		\matr{C} = \bbordermatrix{
			~ & 1 & ~ & ~ & k & k+1 & ~ & ~ & n \cr
			1 & 1 & ~ & ~ & ~ & ~ & ~ & ~ & ~ \cr
			~ & ~ & \ddots & ~ & ~ & ~ & ~ & ~ & ~ \cr
			~ & ~ & ~ & 1 & ~ & ~ & ~ & ~ & ~ \cr
			k & ~ & ~ & ~ & 0 & 1 & ~ & ~ & ~ \cr
			k+1 & ~ & ~ & ~ & 1 & 0 & ~ & ~ & ~ \cr
			~ & ~ & ~ & ~ & ~ & ~ & 1 & ~ & ~ \cr
			~ & ~ & ~ & ~ & ~ & ~ & ~ & \ddots & ~ \cr
			n & ~ & ~ & ~ & ~ & ~ & ~ & ~ & 1 \cr
		}
		$$
		那么可以得到:
		$$
		\transpose{\matr{C}} \diag(\lambda_1, \dots, \lambda_k, \lambda_{k+1}, \dots, \lambda_n) \matr{C} = \diag(\lambda_1, \dots, \lambda_{k+1}, \lambda_k, \dots, \lambda_n)
		$$
		相当于先交换 $i, i+1$ 行再交换 $i, i+1$ 列。于是得证。
	\end{proof}
\end{problem}

\begin{problem}
	课后习题 4

	\begin{proof}
		\begin{enumerate}
			\item[\textbf{1)}] 令 $f(\vect{x}) = \sum_{i=1}^n \sum_{j=1}^n a_{i,j} x_i x_j$,那么:
			$$
			\matr{A} = -\transpose{\matr{A}} \Leftrightarrow f(\vect{x}) \equiv 0 \Leftrightarrow \forall \transpose{\vect{x}} \matr{A} \vect{x} \equiv 0
			$$

			\item[\textbf{2)}] 
			$$
			\forall \vect{x} \in P^n: \transpose{\vect{x}} \matr{A} \vect{x} = 0 \Rightarrow \matr{A} = -\transpose{\matr{A}}
			$$
			而又有 $\matr{A} = \transpose{\matr{A}}$,因此 $\matr{A} = \matr{O}$。
		\end{enumerate}
	\end{proof}
\end{problem}