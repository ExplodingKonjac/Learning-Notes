\section{二次型 IV \& 线性空间 I}

\subsection{实二次型的其它类型及判别法}

\begin{definition}
	对于 $\forall\,\vect{x} \in \mathbb{R}^n$,若

	\begin{itemize}
		\item $f(\vect{x}) < 0$,称二次型 $f(\vect{x})$ 是\textbf{负定的}。
		\item $f(\vect{x}) \ge 0$,称二次型 $f(\vect{x})$ 是\textbf{半正定的}。
		\item $f(\vect{x}) \le 0$,称二次型 $f(\vect{x})$ 是\textbf{半负定的}。
	\end{itemize}

	否则称二次型 $f(\vect{x})$ 是\textbf{不定的}。
\end{definition}

某个二次型 $f(\vect{x}) = \transpose{\matr{x}} \matr{A} \vect{x}$ 是正定的可以等价于下面的条件:

\begin{itemize}
	\item $\matr{A}$ 的正惯性指数 $>0$;
	\item $\matr{A}$ 合同于 $\matr{E}$。
\end{itemize}

而半正定可以等价于:

\begin{itemize}
	\item $\matr{A}$ 的正惯性指数 $=\rank \matr{A}$;
	\item $\matr{A}$ 合同于 $\matr{E}_r$;
	\item 存在 $\matr{C}$ 使得 $\matr{A} = \transpose{\matr{C}} \matr{C}$
	\item $\matr{A}$ 的所有顺序主子式 $\ge 0$。
\end{itemize}

同时,显然 $\matr{A}$ 正定等价于 $-\matr{A}$ 负定,$\matr{A}$ 半正定等价于 $-\matr{A}$ 半负定。

下面给出一个正定矩阵应用例子:

\begin{example}
	设 $\matr{A} \in \mathbb{R}^{n \times n}$ 正定,$b \in \mathbb{R}^n$,证明:
	
	$$
	p(\vect{x}) = \frac{\transpose{\vect{x}} \matr{A} \vect{x}}{2} - \transpose{\vect{x}} \vect{b}
	$$

	在 $\vect{x}_0 = \matr{A}^{-1} \vect{b}$ 取最小值且 $\min p = - \frac{1}{2} \transpose{\vect{b}} \matr{A}^{-1} \vect{b}$。

	\begin{proof}
		首先有:
		$$
		\begin{aligned}
			p(\vect{x}_0) & = \frac{\transpose{\ab(\matr{A}^{-1} \vect{b})} \matr{A} \ab(\matr{A}^{-1} \vect{b})}{2} - \transpose{\ab(\matr{A}^{-1} \vect{b})} \vect{b} \\
			& = \frac{\transpose{\vect{b}} \transpose{\ab(\matr{A}^{-1})} \matr{A} \matr{A}^{-1} \vect{b}}{2} - \transpose{\vect{b}} \transpose{\ab(\matr{A}^{-1})} \vect{b} \\
			& = -\frac{1}{2} \transpose{\vect{b}} \ab(\transpose{\matr{A}})^{-1} \vect{b} \\
			& = - \frac{1}{2} \transpose{\vect{b}} \matr{A}^{-1} \vect{b}
		\end{aligned}
		$$
		然后考虑 $p(\vect{x}) - p(\vect{x}_0)$。
		$$
		\begin{aligned}
			p(\vect{x}) - p(\vect{x}_0) & = \frac{\transpose{\vect{x}} \matr{A} \vect{x}}{2} - \transpose{\vect{x}} \vect{b} + \frac{1}{2} \transpose{\vect{b}} \matr{A}^{-1} \vect{b}
		\end{aligned}
		$$
	\end{proof}
\end{example}

\subsection{集合与映射 strikes back}

给出几个特殊的映射:

\begin{itemize}
	\item 恒等映射 $1_M: M \to M,\ x \mapsto x$。这个映射是映射复合运算的幺元;
	\item 笛卡尔积 $\times: (A, B) \mapsto \{(a, b): a \in A, b \in B\}$。
\end{itemize}

\subsection{线性空间}

\begin{definition}[线性空间]
	若对于数域 $P$ 和一个集合 $V$,满足:

	\begin{enumerate}
		\item 存在加法 $+: V \times V \to V$,满足:
		\begin{itemize}
			\item 交换律:$\vect{\alpha} + \vect{\beta} = \vect{\beta} + \vect{\alpha}$;
			\item 结合律:$\vect{\alpha} + (\vect{\beta} + \vect{\gamma}) = \vect{\alpha} + \vect{\beta} + \vect{\gamma}$;
			\item 存在零元:$\exists\,\vect{0} \in V: \vect{\alpha} + \vect{0} = \vect{\alpha}$;
			\item 存在负元素:$\forall\,\vect{\alpha} \in V: \exists\,\vect{\beta} \in V: \vect{\alpha} + \vect{\beta} = \vect{0}$。
		\end{itemize}

		\item 存在数量乘法 $\cdot: P \times V \to V$,满足:
		\begin{itemize}
			\item $1 \vect{\alpha} = \vect{\alpha}$;
			\item 数量乘法结合律:$k (l \vect{\gamma}) = (k l) \vect{\gamma}$。
		\end{itemize}

		\item 数量乘法和加法之间满足分配率:
		\begin{itemize}
			\item $(k + l) \vect{\alpha} = k \vect{\alpha} + l \vect{\alpha}$;
			\item $k (\vect{\alpha} + \vect{\beta}) = k \vect{\alpha} + k \vect{\beta}$。
		\end{itemize}
	\end{enumerate}
\end{definition}

根据定义可以得到:

\begin{itemize}
	\item $n$ 维向量 $P^n$ 对于数域 $P$ 构成线性空间;
	\item 多项式环 $P[x]$ 对于数域 $P$ 构成线性空间;
	\item 若数域 $F, P$ 满足 $P \subseteq F$,数域 $F$ 对于数域 $P$ 构成线性空间。
\end{itemize}

\subsection{作业}

\begin{problem}
	第五章习题 1. II)

	\begin{solution}
		\begin{enumerate}
			\item[\textbf{1)}] 原式可化为:
			$$
			-(x_1 + x_2 - x_3)^2 + (x_1 - x_2)^2 + x_3^2
			$$
			在实数范围内的规范型为:
			$$
			(x_1 - x_2)^2 + x_3^2 - (x_1 + x_2 - x_3)^2
			$$
			在复数范围内的规范型为:
			$$
			(\I x_1 + \I x_2 - \I x_3)^2 + (x_1 - x_2)^2 + x_3^2
			$$

			\item[\textbf{2)}] 原式可化为:
			$$
			(x_1 - x_2 + x_3)^2 - (2x_2 + x_3)^2
			$$
			在实数范围内的规范型为:
			$$
			(x_1 - x_2 + x_3)^2 - (2x_2 + x_3)^2
			$$
			在复数范围内的规范型为:
			$$
			(x_1 - x_2 + x_3)^2 + (2\I x_2 + \I x_3)^2
			$$
		\end{enumerate}
	\end{solution}
\end{problem}

\begin{problem}
	第五章习题 7
	
	\begin{solution}
		\begin{enumerate}
			\item[\textbf{2)}] 该二次型的矩阵为:
			$$
			\begin{bmatrix}
				99 & -6 & 24 \\
				-6 & 130 & -30 \\
				24 & -30 & 71
			\end{bmatrix}
			$$
			那么显然该矩阵的前两阶顺序主子式 $>0$,而:
			$$
			\begin{vmatrix}
				99 & -6 & 24 \\
				-6 & 130 & -30 \\
				24 & -30 & 71
			\end{vmatrix} = 755874 > 0
			$$
			因此该矩阵所有顺序主子式均 $>0$,二次型正定。

			\item[\textbf{3)}]
			$$
			\begin{aligned}
				\text{原式} & = \sum_{i=1}^n x_i + \frac{1}{2} \ab(\ab(\sum_{i=1}^n x_i)^2 - \sum_{i=1}^n x_i^2) \\
				& = \frac{1}{2} \sum_{i=1}^n x_i^2 + \frac{1}{2} \ab(\sum_{i=1}^n x_i)^2 > 0
			\end{aligned}
			$$
			因此是正定的。
		\end{enumerate}
	\end{solution}
\end{problem}

\begin{problem}
	第五章习题 8

	\begin{solution}
		\begin{enumerate}
			\item[\textbf{1)}] 二次型的矩阵为:
			$$
			\begin{bmatrix}
				1 & t & -1 \\
				t & 1 & 2 \\
				-1 & 2 & 5
			\end{bmatrix}
			$$
			需要二阶、三阶顺序主子式不为 $0$,那么:
			$$
			\begin{gathered}
				1 - t^2 > 0 \\
				5 - 2t - 2t - 1 - 5t^2 - 4 > 0
			\end{gathered}
			$$
			因此 $-\frac{4}{5} < t < 0$。

			\item[\textbf{2)}] 二次型的矩阵为:
			$$
			\begin{bmatrix}
				1 & t & 5 \\
				t & 4 & 3 \\
				5 & 3 & 1
			\end{bmatrix}
			$$
			需要二阶、三阶顺序主子式不为 $0$,那么:
			$$
			\begin{gathered}
				4 - t^2 \neq 0 \\
				4 + 15t + 15t - 100 - 9 - t^2 \neq 0
			\end{gathered}
			$$
			无解,因此矩阵一定不正定。
		\end{enumerate}
	\end{solution}
\end{problem}

\begin{problem}
	第五章习题 9

	\begin{proof}
		合同变换不改变矩阵的正定性。因此对于任意一个主子式 $(i_1, i_2, \dots, i_k)$,都可以通过同时交换行/列的合同变换来把它变成 $(1, 2, \dots, k)$。这样根据顺序主子式均不为 $0$ 就得证。
	\end{proof}
\end{problem}

\begin{problem}
	第五章习题 10

	\begin{proof}
		对于一组 $x_1, x_2, \dots, x_n$,我们只需要考虑其中的非零项,因此不妨认为所有 $x_i \neq 0$。那么设 $M = \max\{|x_i|\}$,再设 $K = \max_{1 \le i < j \le n}\{|a_{i,j}|\}$,就有:
		$$
		f(\vect{x}) \ge (t - K)  M^2 - n (n - 1) K M^2
		$$
		那么只需要 $t > K + n (n - 1) K$ 即可保证 $f(\vect{x}) > 0$。
	\end{proof}
\end{problem}

\begin{problem}
	第五章习题 13

	\begin{proof}
		显然:
		$$
		\transpose{\vect{x}} (\matr{A} + \matr{B}) \vect{x} = \transpose{\vect{x}} \matr{A} \vect{x} + \transpose{\vect{x}} \matr{B} \vect{x} > 0
		$$
	\end{proof}
\end{problem}

\begin{problem}
	第五章习题 14

	\begin{proof}
		正惯性指数与秩相等 $\Leftrightarrow$ 负惯性指数为 $0$ $\Leftrightarrow$ 规范型矩阵半正定 $\Leftrightarrow$ 原矩阵半正定。
	\end{proof}
\end{problem}

\begin{problem}
	第五章习题 15

	\begin{proof}
		设 $\matr{A} \in P^{n \times m}$,那么 $\transpose{\matr{A}} \matr{A} \in P^{m \times m}$。设方程组
		$$
		\matr{A} \vect{x} = \vect{0},\ \transpose{\matr{A}} \matr{A} \vect{x} = \vect{0}
		$$
		的解空间分别为 $V_1, V_2$。那么只需要证明 $V_1, V_2$ 的维数相等。而
		$$
		\begin{gathered}
			\matr{A} \vect{x} = \vect{0} \Rightarrow \transpose{\matr{A}} \matr{A} \vect{x} = \vect{0} \\
			\transpose{\matr{A}} \matr{A} \vect{x} = \vect{0} \Rightarrow \transpose{\vect{x}} \transpose{\matr{A}} \matr{A} \vect{x} = 0 \Rightarrow |\matr{A} \vect{x}|^2 = 0 \Rightarrow \matr{A} \vect{x} = \vect{0}
		\end{gathered}
		$$
		于是 $V_1 = V_2$,得证。
	\end{proof}
\end{problem}

\begin{problem}
	第六章习题 3
	
	\begin{solution}
		\begin{enumerate}
			\item[\textbf{1)}] 不构成。$(x^n) - (x^n) = 0$ 不是 $n$ 次多项式;
			\item[\textbf{2)}] 是线性空间。对称、反称、上三角矩阵在相加、数乘之后仍然是对称、反称、上三角矩阵;
			\item[\textbf{7)}] 不是线性空间。$0\vect{\alpha} \neq 1\cdot \vect{\alpha} - 1 \cdot \vect{\alpha}$;
			\item[\textbf{8)}] 是线性空间。把 $x \mapsto \ln x$ 之后就是线性空间,而这是一个双射。
		\end{enumerate}
	\end{solution}
\end{problem}