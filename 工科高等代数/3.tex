\section{行列式 II}

\subsection{拉普拉斯定理}

\begin{definition}[子式、余子式、代数余子式]
	在一个 $n$ 阶行列式中选定 $k$ 个不同的行标 $i_1,i_2,\cdots,i_k$ 和 $k$ 个不同的列标 $j_1,j_2,\cdots,j_k$,这些行列交叉处元素组成一个 \textbf{$k$ 阶子式} $M$。而在原行列式中划去这些行列后剩下元素组成的行列式称为 $M$ 的\textbf{余子式} $M'$。而 $(-1)^{\sum_{m=1}^k i_m + \sum_{m=1}^k j_m} M'$ 称为 $M$ 的\textbf{代数余子式}。
\end{definition}

\begin{theorem}[拉普拉斯定理]
	在行列式 $D$ 中任意选定 $k$ 行,这 $k$ 行元素组成的所有 $k$ 阶子式记作 $M_1,M_2,\cdots,M_t$,它们的代数余子式分别为 $A_1,A_2,\cdots,A_t$。则 $D = \sum_{i=1}^t A_i M_i$。

	\begin{proof}
		先证明 $M_i A_i$ 的每一项都是 $D$ 的一项,且符号一致。当 $M$ 占据左上角的时候:
		$$
		D = \left|\begin{array}{ccc:ccc}
			a_{1,1} & \cdots & a_{1,k} & a_{1,k+1} & \cdots & a_{1,n} \\
			\vdots & M & \vdots & \vdots & \ddots & \vdots \\
			a_{k,1} & \cdots & a_{k,k} & a_{k,k+1} & \cdots & a_{k,n} \\
			\hdashline
			a_{k+1,1} & \cdots & a_{k+1,k} & a_{k+1,k+1} & \cdots & a_{k+1,n} \\
			\vdots & \ddots & \vdots & \vdots & M' & \vdots \\
			a_{n,1} & \cdots & a_{n,k} & a_{n,k+1} & \cdots & a_{n,n}
		\end{array}\right|
		$$
		那么 $M_i A_i$ 的每项:
		$$
		\begin{aligned}
			& (-1)^{\tau(\alpha)} \prod_{i=1}^k a_{i,\alpha_i} \cdot (-1)^{\tau(\beta)} \prod_{i=k+1}^n a_{i,\beta_{i-k}} \\
			= & (-1)^{\tau(\alpha)+\tau(\beta)} \prod_{i=1}^k a_{i,\alpha_i} \cdot \prod_{i=k+1}^n a_{i,\beta_{i-k}} \\
			= & (-1)^{\tau(\alpha_1,\cdots,\alpha_k,\beta_1,\cdots,\beta_{n-k})} \prod_{i=1}^k a_{i,\alpha_i} \cdot  \prod_{i=k+1}^n a_{i,\beta_{i-k}} \\
		\end{aligned}
		$$
		而 $(-1)^{\sum_{i=1}^k i + \sum_{i=1}^k i}=1$,不需考虑。所以这显然是 $D$ 的一项。

		对于一般情形而言,通过不断交换行列到相应位置即可证明。

		而原定理的右侧式子显然是可以覆盖到 $D$ 的每一项的,因此得证。
	\end{proof}
\end{theorem}

行列式按行、按列展开其实就是拉普拉斯定理的 $k=1$ 特殊情形。

据此有一个常用结论:

\begin{theorem}
	记 $A_{i,j}$ 表示行列式 $D = |a_{i,j}|_n$ 在 $(i,j)$ 的代数余子式,那么:
	$$
	\sum_{s=1}^n a_{k,s} A_{i,s} = [k=i] D
	$$
	对于列也适用。

	\begin{proof}
		根据行展开规则,$k=i$ 时其值就是 $D$,否则相当于求把 $i$ 行替换为 $k$ 行的新行列式,而这个行列式有两行相同,因此为 $0$。
	\end{proof}
\end{theorem}

\subsection{克拉默法则}

\begin{theorem}[克拉默法则]
	对于一个线性方程组
	$$
	\sum_{i=1}^n a_{k,i} x_i = b_k \quad (k=1,\dots,n)
	$$
	记 $\matr{A} = [a_{i,j}]_n,\ d = \det(\matr{A})$,再记 $d_i$ 表示把 $\matr{A}$ 的第 $i$ 列替换为 $\vect{b}$ 的行列式,那么方程组有唯一解的充要条件是 $d \neq 0$,且解一定为:
	$$
	x_k = \frac{d_k}{d} \quad (k=1,\dots,n)
	$$
\end{theorem}

\subsection{作业}

\begin{problem}
	课后习题 12
	\begin{proof}
		我们有:
		$$
		\begin{vmatrix}
			1 & 1 & \cdots & 1 \\
			1 & 1 & \cdots & 1 \\
			\vdots & \vdots & \ddots & \vdots \\
			1 & 1 & \cdots & 1
		\end{vmatrix} = \sum_{p_1,\dots,p_n} (-1)^{\tau(p)} = 0
		$$
		和式的每一项中,$p$ 为奇排列时是 $-1$,为偶排列时是 $+1$,而求和后为 $0$,说明奇偶排列数量相等。
	\end{proof}
\end{problem}

\begin{problem}
	课后习题 13
	\begin{solution}
		\begin{enumerate}
			\item[\textbf{1)}] 根据行列式定义,行列式每一项中最多包含一个 $x^k$,而 $k$ 取值为 $1,2,\dots,n-1$,因此原行列式为 $n-1$ 次多项式
			\item[\textbf{2)}] 注意到,当 $x=a_1,a_2,\dots,a_{n-1}$ 时,行列式中有两行相等,那么 $P(x) = 0$。这说明 $a_1,a_2,\dots,a_{n-1}$ 为 $P(x)$ 的 $n-1$ 个不同的根。根据代数基本定理,$n-1$ 次多项式有 $n-1$ 个复根,因此 $P(x)$ 的根就是 $a_1,a_2,\dots,a_{n-1}$。
		\end{enumerate}
	\end{solution}
\end{problem}

\begin{problem}
	课后习题 14
	\begin{solution}
		\begin{enumerate}
			\item[\textbf{2)}]
			$$
			\begin{aligned}
				\text{原式} & = \begin{vmatrix}
					1 & 1 & 1 & 1 \\
					0 & x & y & x+y \\
					0 & y & x+y & x \\
					0 & x+y & x & y
				\end{vmatrix} & \\
				& = \begin{vmatrix}
					1 & 1 & 1 & 1 \\
					-(x+y) & -y & -x & 0 \\
					-(x+y) & -x & 0 & -y \\
					-(x+y) & 0 & -y & -x
				\end{vmatrix} & \text{(第 $1$ 行的 $-(x+y)$ 倍加到第 $2,3,4$ 行)} \\
				& = \begin{vmatrix}
					-2 & 1 & 1 & 1 \\
					0 & -y & -x & 0 \\
					0 & -x & 0 & -y \\
					0 & 0 & -y & -x
				\end{vmatrix} = -2(x^3 + y^3) & \text{(第 $2,3,4$ 列加到第 $1$ 列)}\\
			\end{aligned}
			$$

			\item[\textbf{4)}]
			$$
			\begin{aligned}
				\text{原式} & = \begin{vmatrix}
					1 & 2 & 3 & 4 \\
					0 & -1 & -2 & -7 \\
					0 & -2 & -8 & -10 \\
					0 & -7 & -10 & -13
				\end{vmatrix} = \begin{vmatrix}
					1 & 2 & 3 & 4 \\
					0 & -1 & -2 & -7 \\
					0 & 0 & -4 & 4 \\
					0 & 0 & 4 & 36
				\end{vmatrix} = \begin{vmatrix}
					1 & 2 & 3 & 4 \\
					0 & -1 & -2 & -7 \\
					0 & 0 & -4 & 4 \\
					0 & 0 & 0 & 40
				\end{vmatrix} = 160 \\
			\end{aligned}
			$$

			\item[\textbf{5)}]
			$$
			\begin{aligned}
				\text{原式} & = \begin{vmatrix}
					1 & 1 & 1 & 1 & 1 \\
					0 & 1 + x & 1 & 1 & 1 \\
					0 & 1 & 1 - x & 1 & 1 \\
					0 & 1 & 1 & 1 + y & 1 \\
					0 & 1 & 1 & 1 & 1 - y
				\end{vmatrix} \\
				& = \begin{vmatrix}
					1 & 1 & 1 & 1 & 1 \\
					-1 & x & 0 & 0 & 0 \\
					-1 & 0 & -x & 0 & 0 \\
					-1 & 0 & 0 & y & 0 \\
					-1 & 0 & 0 & 0 & -y	
				\end{vmatrix} & \text{(将第一行的 $-1$ 倍加到其他行)} \\
				& = \begin{vmatrix}
					1 & 1 & 1 & 1 & 1 \\
					0 & x & 0 & 0 & 0 \\
					0 & 0 & -x & 0 & 0 \\
					0 & 0 & 0 & y & 0 \\
					0 & 0 & 0 & 0 & -y	
				\end{vmatrix} = x^2 y^2 & \text{(用 $2,3,4,5$ 列消去第 $1$ 列的 $-1$)}\\
			\end{aligned}
			$$
		\end{enumerate}
	\end{solution}
\end{problem}

\begin{problem}
	课后习题 15
	\begin{proof}
		$$
		\begin{aligned}
			\text{LHS} & = \begin{vmatrix}
				2(a + b + c) & c + a & a + b \\
				2(a_1 + b_1 + c) & c_1 + a_1 & a_1 + b_1 \\
				2(a_2 + b_2 + c) & c_2 + a_2 & a_2 + b_2
			\end{vmatrix} = 2\begin{vmatrix}
				a + b + c & c + a & a + b \\
				a_1 + b_1 + c & c_1 + a_1 & a_1 + b_1 \\
				a_2 + b_2 + c & c_2 + a_2 & a_2 + b_2
			\end{vmatrix} \\
			& = 2\begin{vmatrix}
				a + b + c & -b & -c \\
				a_1 + b_1 + c & -b_1 & -c_1 \\
				a_2 + b_2 + c & -b_2 & -c_2
			\end{vmatrix} = 2\begin{vmatrix}
				a & -b & -c \\
				a_1 & -b_1 & -c_1 \\
				a_2 & -b_2 & -c_2
			\end{vmatrix} \\
			& = \text{RHS}
		\end{aligned}
		$$
	\end{proof}
\end{problem}

\begin{problem}
	课后习题 17
	\begin{solution}
		\begin{enumerate}
			\item[\textbf{1)}]
			$$
			\begin{aligned}
				\text{原式} & = \begin{vmatrix}
					1 & 1 & 1 & 1 \\
					0 & -1 & -1 & -5 \\
					0 & 1 & 1 & 4 \\
					0 & -1 & -2 & -3
				\end{vmatrix} = \begin{vmatrix}
					1 & 1 & 1 & 1 \\
					0 & -1 & -1 & -5 \\
					0 & 0 & 0 & -1 \\
					0 & 0 & -1 & -2
				\end{vmatrix} \\
				& = -\begin{vmatrix}
					1 & 1 & 1 & 1 \\
					0 & -1 & -1 & -5 \\
					0 & 0 & -1 & -2 \\
					0 & 0 & 0 & -1
				\end{vmatrix} = 1 \\
			\end{aligned}
			$$

			\item[\textbf{3)}]
			$$
			\begin{aligned}
				\text{原式} & = -\begin{vmatrix}
					-1 & 3 & 5 & 1 & 2 \\
					2 & 0 & 1 & 2 & 1 \\
					0 & 1 & 2 & -1 & 4 \\
					3 & 3 & 1 & 2 & 1 \\
					2 & 1 & 0 & 3 & 5
				\end{vmatrix} = -\begin{vmatrix}
					-1 & 3 & 5 & 1 & 2 \\
					0 & 6 & 11 & 4 & 5 \\
					0 & 1 & 2 & -1 & 4 \\
					0 & 12 & 16 & 5 & 7 \\
					0 & 7 & 10 & 5 & 9
				\end{vmatrix} \\
				& = \begin{vmatrix}
					-1 & 3 & 5 & 1 & 2 \\
					0 & 1 & 2 & -1 & 4 \\
					0 & 6 & 11 & 4 & 5 \\
					0 & 12 & 16 & 5 & 7 \\
					0 & 7 & 10 & 5 & 9
				\end{vmatrix} = \begin{vmatrix}
					-1 & 3 & 5 & 1 & 2 \\
					0 & 1 & 2 & -1 & 4 \\
					0 & 0 & -1 & 10 & -19 \\
					0 & 0 & -8 & 17 & -41 \\
					0 & 0 & -4 & 12 & -19
				\end{vmatrix} \\
				& = \begin{vmatrix}
					-1 & 3 & 5 & 1 & 2 \\
					0 & 1 & 2 & -1 & 4 \\
					0 & 0 & -1 & 10 & -19 \\
					0 & 0 & 0 & -63 & 111 \\
					0 & 0 & 0 & -28 & 57 \\
				\end{vmatrix} = (-1) \cdot 1 \cdot (-1) \cdot ((-63) \cdot 57 - 111 \cdot (-28)) = -483
			\end{aligned}
			$$
		\end{enumerate}
	\end{solution}
\end{problem}

\begin{problem}
	课后习题 18
	\begin{solution}
		\begin{enumerate}
			\item[\textbf{1)}] 对于某非 $0$ 项:若最后一行选择了 $a_{n,1}$,那么前 $n-1$ 行只能选择 $a_{1,2},a_{2,3},\dots,a_{n-1,n}$;若最后一行选择了 $a_{n,n}$,那么前 $n-1$ 行只能选择 $a_{1,1},a_{2,2},\dots,a_{n-1,n-1}$。由此可得,$\text{原式} = x^n + (-1)^{n-1} y^n$。

			\item[\textbf{4)}]
			$$
			\begin{aligned}
				\text{原式} & = \begin{vmatrix}
					1 & 2 & 2 & 2 & \cdots & 2 \\
					1 & 0 & 0 & 0 & \cdots & 0 \\
					1 & 0 & 1 & 0 & \cdots & 0 \\
					1 & 0 & 0 & 2 & \cdots & 0 \\
					\vdots & \vdots & \vdots & \vdots & \ddots & \vdots \\
					1 & 0 & 0 & 0 & \cdots & n-2
				\end{vmatrix} & \text{(用第一行 $-1$ 倍加到其余行)} \\
				& = \begin{vmatrix}
					1 - \sum_{i=1}^{n-2} 2/i & 2 & 2 & 2 & \cdots & 2 \\
					1 & 0 & 0 & 0 & \cdots & 0 \\
					0 & 0 & 1 & 0 & \cdots & 0 \\
					0 & 0 & 0 & 2 & \cdots & 0 \\
					\vdots & \vdots & \vdots & \vdots & \ddots & \vdots \\
					0 & 0 & 0 & 0 & \cdots & n-2
				\end{vmatrix} & \text{(用 $3,\dots,n$ 列消去第一列的 $1$)} \\
				& = \begin{vmatrix}
					1 - \sum_{i=1}^{n-2} 2/i & 2 \\
					1 & 0
				\end{vmatrix} \cdot |\operatorname{diag}(1,2,\dots,n-2)| \\
				& = -2(n-2)!
			\end{aligned}
			$$
		\end{enumerate}
	\end{solution}
\end{problem}

\begin{problem}
	课后习题 19
	\begin{proof}
		\begin{enumerate}
			\item[\textbf{2)}] 枚举第 $n$ 列选择了哪个元素,则剩下的选择可以推出来:
			$$
			\begin{aligned}
				\text{LHS} & = x^{n-1}(x + a_{n-1}) + \sum_{k=0}^{n-2} x^k (-1)^{n-k-1} a_k \cdot \tau(1,2,\dots,k,n,k+1,k+2,\dots,n-1) \\
				& = x^{n-1}(x + a_{n-1}) + \sum_{k=0}^{n-2} x^k (-1)^{n-k-1} a_k \cdot (-1)^{n-k-1} \\
				& = x^{n-1}(x + a_{n-1}) + \sum_{k=0}^{n-2} x^k a_k \\
				& = x^n + \sum_{k=0}^{n-1} x^k a_k = \text{RHS}
			\end{aligned}
			$$
			于是得证。

			\item[\textbf{3)}] 使用数学归纳法。设左式的值为 $f_n$。那么 $f_1 = \alpha + \beta,\ f_2 = \alpha^2 + \beta^2 + \alpha \beta$,均满足条件,归纳奠基完成。
			
			然后假设 $<n\ (n \ge 3)$ 时均满足条件,要证明 $f_n$ 也满足条件,考虑第 $n$ 行选择哪个元素:
			
			\begin{itemize}
				\item 如果选择 $a_{n,n}$,那么贡献就是 $(\alpha + \beta) f_{n-1}$。
				\item 如果选择 $a_{n,n-1}$,那么可以推出第 $n$ 列只能选取 $a_{n-1,n}$,所以贡献是 $-\alpha \beta f_{n-2}$。
			\end{itemize}

			于是我们得到:
			$$
			\begin{aligned}
				f_n & = (\alpha + \beta) f_{n-1} - \alpha \beta f_{n-2} \\
				& = \frac{(\alpha + \beta) (\alpha^n - \beta ^n) - \alpha \beta (\alpha^{n-1} - \beta^{n-1})}{\alpha - \beta} \\
				& = \frac{\alpha^n - \beta^n}{\alpha - \beta}
			\end{aligned}
			$$
			也满足条件,故得证。
		\end{enumerate}
	\end{proof}
\end{problem}

\begin{problem}
	课后习题 20
	\begin{solution}
		对第一个行列式进行初等变换:
		$$
		\begin{aligned}
			\begin{vmatrix}
				x & y & z & x + y + z \\
				3 & 0 & 2 & 0 \\
				1 & 1 & 1 & 1 \\
				2 & 2 & 2 & 1
			\end{vmatrix} & \xrightarrow{\text{第 $3$ 行乘 $2$}} \begin{vmatrix}
				x & y & z & x + y + z \\
				3 & 0 & 2 & 0 \\
				2 & 2 & 2 & 2 \\
				2 & 2 & 2 & 1
			\end{vmatrix} \\
			& \xrightarrow{\text{第 $3$ 行减去第 $2$ 行}} \begin{vmatrix}
				x & y & z & x + y + z \\
				3 & 0 & 2 & 0 \\
				2 - x & 2 - y & 2 - z & 2 - x - y - z \\
				2 & 2 & 2 & 1
			\end{vmatrix} \\
			& \xrightarrow{\text{第 $3$ 列减去第 $1$ 列}} \begin{vmatrix}
				x & y & z - x & x + y + z \\
				3 & 0 & -1 & 0 \\
				2 - x & 2 - y & x - z & 2 - x - y - z \\
				2 & 2 & 0 & 1
			\end{vmatrix} \\
			& \xrightarrow{\text{第 $1$ 列减去第 $2$ 列}} \begin{vmatrix}
				x - y & y & z - x & x + y + z \\
				3 & 0 & -1 & 0 \\
				y - x & 2 - y & x - z & 2 - x - y - z \\
				0 & 2 & 0 & 1
			\end{vmatrix} \\
			& \xrightarrow{\text{第 $4$ 行加上第 $2$ 行}} \begin{vmatrix}
				x - y & y & z - x & x + y + z \\
				3 & 0 & -1 & 0 \\
				y - x & 2 - y & x - z & 2 - x - y - z \\
				3 & 2 & -1 & 1
			\end{vmatrix}
		\end{aligned}
		$$
		得到了下面的行列式。因此所求行列式为 $\frac{1}{2} \cdot 2 = 1$。
	\end{solution}
\end{problem}

\begin{problem}
	课后习题 22 1)
	\begin{solution}
		$$
		\begin{cases}
			x_1 = 1 \\
			x_2 = 1 \\
			x_3 = 1 \\
			x_4 = 1
		\end{cases}
		$$
	\end{solution}
\end{problem}

\subsection{补充题}

\begin{problem}
	补充题 1
	\begin{solution}
		$$
		\text{原式} = \sum_{j_1,j_2,\dots,j_n} (-1)^{\tau(j_1,j_2,\dots,j_n)} |a_{i,j}|_n = 0
		$$
	\end{solution}
\end{problem}

\begin{problem}
	补充题 3.1)
	\begin{proof}
		$$
		\begin{aligned}
			\text{LHS} & = \sum_{i=1}^n (a_{1,i} + x) A_{1,i} \\
			& = \sum_{i=1}^n a_{1,i} A_{1,i} + \sum_{i=1}^n x A_{1,i} \\
			& = \begin{vmatrix}
				a_{1,1} & a{1,2} & \cdots & a_{1,n} \\
				a_{2,1} + x & a_{2,2} + x & \cdots & a_{1,n} + x \\
				\vdots & \vdots & \ddots & \vdots \\
				a_{n,1} + x & a_{n,2} + x & \cdots & a_{n,n} + x
			\end{vmatrix} + \sum_{i=1}^n x A_{1,i}
		\end{aligned}
		$$
		继续展开 $2,3,\dots,n$ 行,归纳即证。
	\end{proof}
\end{problem}

\begin{problem}
	补充题 4.4)
	\begin{solution}
		设答案为 $d_n$:
	\end{solution}
\end{problem}
