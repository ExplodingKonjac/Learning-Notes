\section{线性方程组 IV}

\subsection{求解秩}

求解向量组的秩相当于求解这些向量拼接形成的矩阵的秩。求解矩阵的秩只需要用初等行变换将矩阵化为阶梯矩阵,那么非零行的数量就是矩阵的秩。

\subsection{克拉默法则及其逆定理}

首先对于齐次线性方程组考虑:

\begin{lemma} \label{lemma:齐次线性方程组唯一解}
	设矩阵 $\matr{A} = [a_{i,j}]_n$,那么齐次线性方程组
	$$
	\matr{A} \vect{x} = \vect{0}
	$$
	有非零解的充要条件是 $\det \matr{A} = 0$。

	\begin{proof}
		方程组有非零解 $\Leftrightarrow$ 矩阵 $\matr{A}$ 的行向量组线性相关 $\Leftrightarrow$ 矩阵 $\matr{A}$ 不满秩 $\Leftrightarrow \det \matr{A} = 0$。
	\end{proof}
\end{lemma}

\begin{theorem}[克拉默法则及其逆定理]
	给定矩阵 $\matr{A} = [a_{i,j}]_n$ 和向量 $\vect{b} = [b_i]_{n \times 1}$,那么线性方程组
	$$
	\matr{A} \vect{x} = \vect{b}
	$$
	有唯一解的充要条件是 $\det \matr{A} \neq 0$。

	\begin{proof}
		充分性是克拉默法则。下证明必要性:

		若 $\det \matr{A} \neq 0$,那么方程组要么无解,要么有至少两个解。因为若存在一个解 $\vect{x}_1$,根据引理 \ref{lemma:齐次线性方程组唯一解},一定存在非零向量 $\vect{x}_0$ 使得 $\matr{A} \vect{x_0} = \vect{0}$,于是 $\matr{A} (\vect{x_1} + \vect{x_0}) = \vect{b}$,这说明 $(\vect{x}_1 + \vect{x}_0)$ 也是一个解。
	\end{proof}
\end{theorem}

\subsection{作业}

\begin{problem}
	第三章习题 7

	\begin{proof}
		考虑反证。若这 $r$ 个向量不是极大线性无关组,那么再添加一个向量后依然线性无关,这说明秩 $\ge r + 1$,矛盾。
	\end{proof}
\end{problem}

\begin{problem}
	第三章习题 8

	\begin{proof}
		只需证明 $\vect{\alpha}_{i_1}, \vect{\alpha}_{i_2}, \dots, \vect{\alpha}_{i_r}$ 线性无关
		
		考虑反证。若它们线性相关,找到它一个极大线性无关组 $\vect{\beta}_1, \vect{\beta}_2, \dots, \vect{\beta}_s\ (s<r)$,那么 $\vect{\beta}_1, \vect{\beta}_2, \dots, \vect{\beta}_s$ 能表出 $\vect{\alpha}_1, \vect{\alpha}_2, \dots, \vect{\alpha}_n$,因此也能表出它的任意一个极大线性无关组 $\vect{\gamma}_1, \vect{\gamma}_2, \dots, \vect{\gamma}_r$。根据向量组相关推论,$r > s$ 说明 $\vect{\gamma}_1, \vect{\gamma}_2, \dots, \vect{\gamma}_r$ 线性相关,矛盾。
		
		故得证。
	\end{proof}
\end{problem}

\begin{problem}
	第三章习题 9

	\begin{proof}
		设这个部分组为 $\mathcal{G}$,只需要一直在向量组中找一个未加入的向量,使得 $\mathcal{G}$ 加入该向量后依然线性无关,并将该向量加入 $\mathcal{G}$,即可得到一个极大线性无关组。显然该过程的步数是有限的。
	\end{proof}
\end{problem}

\begin{problem}
	第三章习题 10
	
	\begin{proof}
		\begin{enumerate}
			\item[\textbf{1)}] 即证,不存在不全为 $0$ 的 $k_1,k_2$ 使得 $k_1 \alpha_1 + k_2 \alpha_2 = \vect{0}$。要使这个方程成立,先考察第一维分量得到 $k_1 = 0$,然后可知仅有 $k_2 = 0$ 才满足条件。故不存在不全为 $0$ 的 $k_1,k_2$。
			
			\item[\textbf{2)}] $\vect{\alpha}_1, \vect{\alpha}_2, \vect{\alpha}_4$。
		\end{enumerate}
	\end{proof}
\end{problem}

\begin{problem}
	第三章习题 11

	\begin{proof}
		\begin{enumerate}
			\item[\textbf{2)}] 所有极大线性无关组为:$(\vect{\alpha}_1, \vect{\alpha_2}, \vect{\alpha}_4)$, $(\vect{\alpha}_1, \vect{\alpha_2}, \vect{\alpha}_5)$, $(\vect{\alpha}_1, \vect{\alpha_3}, \vect{\alpha}_4)$, $(\vect{\alpha}_1, \vect{\alpha_3}, \vect{\alpha}_5)$, $(\vect{\alpha}_2, \vect{\alpha_3}, \vect{\alpha}_4)$, $(\vect{\alpha}_2, \vect{\alpha_3}, \vect{\alpha}_5)$;
			
			故原向量组的秩为 $3$。
		\end{enumerate}
	\end{proof}
\end{problem}

\begin{problem}
	第三章习题 12

	\begin{proof}
		若 (I) 能被 (II) 表出,则 (I) 的极大线性无关组能被 (II) 的极大线性无关组表出,那么 (I) 的极大线性无关组的大小不大于 (II) 的极大线性无关组的大小,即 (I) 的秩不大于 (II) 的秩。
	\end{proof}
\end{problem}

\begin{problem}
	第三章习题 18

	\begin{proof}
		只需要构造一组初等变换将 $\vect{\alpha}_1, \vect{\alpha}_2, \dots, \vect{\alpha}_r$ 变换为 $\vect{\beta}_1, \vect{\beta}_2, \dots, \vect{\beta}_r$ 即证。构造如下:

		\begin{enumerate}
			\item 初始时 $\vect{\beta}_i = \vect{\alpha}_i$;
			\item 令 $\vect{\beta}_i \gets \vect{\beta}_i + \sum_{i=2}^r \vect{\beta}_i$(此次操作后,$\vect{\beta}_1 = \sum_{k=1}^r \vect{\alpha}_k$);
			\item 对于 $2 \le i \le r$,令 $\vect{\beta}_i \gets \vect{\beta}_i - \vect{\beta}_1$;
			\item 对于 $2 \le i \le r$,令 $\vect{\beta}_i \gets -\vect{\beta}_i$(此次操作后,$\forall\,2 \le i \le r: \vect{\beta}_i = \sum_{k=1}^r \vect{\alpha}_k - \vect{\alpha}_i$);
			\item 令 $\vect{\beta}_i \gets (r - 1) \vect{\beta}_i$;
			\item 令 $\vect{\beta}_i \gets \vect{\beta}_i - \sum_{k=2}^r \vect{\beta}_k$(此次操作后,$\vect{\beta}_1 = \sum_{k=2}^r \vect{\alpha}_k$)。
		\end{enumerate}

		最终得到 $\vect{\beta}_i = \sum_{k=1}^r \vect{\alpha}_k - \vect{\alpha}_i$,得证。
	\end{proof}
\end{problem}

\begin{problem}
	第三章习题 19

	\begin{solution}
		\begin{enumerate}
			\item[\textbf{3)}] 设矩阵的行向量为 $\vect{\alpha}_1, \vect{\alpha}_2, \vect{\alpha}_3, \vect{\alpha}_4$。那么有 $\vect{\alpha}_1 = 2 \vect{\alpha}_3 = \frac{2}{5} \vect{\alpha}_4$。这说明 $\vect{\alpha}_1, \vect{\alpha}_3, \vect{\alpha}_4$ 两两之间线性相关。因此 $\vect{\alpha}_1, \vect{\alpha}_2$ 是一个极大线性无关组,矩阵的秩为 $2$。
		\end{enumerate}
	\end{solution}
\end{problem}
