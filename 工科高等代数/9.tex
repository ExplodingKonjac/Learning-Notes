\section{矩阵 II}

\subsection{矩阵的运算 continue}

\begin{definition}[矩阵的方幂]
	对于 $\matr{A} \in P^{n \times n}$,定义 $\matr{A}^k = \underbrace{\matr{A} \times \matr{A} \times \dots \times \matr{A}}_{\text{$k$ 个 $\matr{A}$}}$
\end{definition}

容易得到方幂的若干性质:

\begin{property}[矩阵方幂的性质]
	\ 
	\begin{itemize}
		\item $\matr{A}^k \matr{A}^l = \matr{A}^{k + l}$;
		\item $\ab(\matr{A}^k)^l = \matr{A}^{kl}$。
	\end{itemize}
\end{property}

\begin{definition}[矩阵的数乘]
	若 $k \in P,\ \matr{A} = [a_{i,j}]_{n \times m}$,那么称 $[ka_{i,j}]_{n \times m}$ 为 $\matr{A}$ 的\textbf{数量乘积},记作 $k \matr{A}$。
\end{definition}

数乘具有下面的性质:

\begin{property}[矩阵数乘的性质]
	\ 
	\begin{itemize}
		\item $(k + l) \matr{A} = k \matr{A} + l \matr{A}$;
		\item $k (\matr{A} + \matr{B}) = k \matr{A} + k \matr{B}$;
		\item $(k \times l) \matr{A} = k (l \matr{A})$;
		\item $1 \matr{A} = \matr{A}$;
		\item $k(\matr{A} \matr{B}) = (k \matr{A}) \matr{B} = \matr{A} (k \matr{B})$。
	\end{itemize}
\end{property}

特别地,通常记 $k\matr{E}_n$ 为\textbf{数量矩阵}。数量矩阵在矩阵乘法中表现和数一样。同时,若某个矩阵和所有 $n \times n$ 矩阵的乘法是可交换的,那么这个矩阵一定是数量矩阵。

\begin{definition}[矩阵的转置]
	对于矩阵 $\matr{A} = [a_{i,j}]_{n \times m}$,定义 $\matr{A}^\mathrm{T} = [a_{j,i}]_{m \times n}$。
\end{definition}

那么矩阵转置具有下面性质:

\begin{property}[矩阵转置的性质]
	\begin{itemize}
		\item $\ab(\matr{A}^\mathrm{T})^\mathrm{T} = \matr{A}$;
		\item $(\matr{A} + \matr{B})^\mathrm{T} = \matr{A}^\mathrm{T} + \matr{B}^\mathrm{T}$;
		\item $(k \matr{A})^\mathrm{T} = k \matr{A}^\mathrm{T}$;
		\item $(\matr{A} \matr{B})^\mathrm{T} = \matr{B}^\mathrm{T} \matr{A}^\mathrm{T}$。
	\end{itemize}
\end{property}

\subsection{矩阵乘法与行列式与秩}

\begin{theorem}
	设 $\matr{A},\matr{B} \in P^{n \times n}$,那么:
	$$
	\det(\matr{A} \matr{B}) = \det(\matr{A}) \det(\matr{B})
	$$
\end{theorem}

\begin{definition}
	对于方阵 $\matr{A}$,若 $\det A = 0$,称 $A$ 为\textbf{退化的},否则称 $A$ 为\textbf{非退化的}。
\end{definition}

\begin{theorem}
	设 $\matr{A},\matr{B} \in P^{n \times n}$,那么:
	$$
	\rank(\matr{A} \matr{B}) \le \min\ab\{\rank \matr{A}, \rank \matr{B}\}
	$$

	\begin{proof}
		根据矩阵乘法定义,$\matr{A} \matr{B}$ 的列向量组能被 $\matr{A}$ 的列向量组线性表出,因此 $\rank(\matr{A} \matr{B}) \le \rank \matr{A}$。

		同理,$\matr{A} \matr{B}$ 的行向量组能被 $\matr{B}$ 的行向量组线性表出,因此 $\rank(\matr{A} \matr{B}) \le \rank \matr{B}$。

		于是得证。
	\end{proof}
\end{theorem}

\subsection{作业}

\begin{problem}
	第三章习题 22

	\begin{solution}
		对系数矩阵做初等行变换:
		$$
		\begin{bmatrix}
			3 & 4 & -5 & 7 \\
			2 & -3 & 3 & -2 \\
			4 & 11 & -13 & 16 \\
			7 & -2 & 1 & 3
		\end{bmatrix} \longrightarrow
		\begin{bmatrix}
			3 & 4 & -5 & 7 \\
			0 & -\frac{17}{3} & \frac{19}{3} & -\frac{20}{3} \\
			0 & \frac{17}{3} & -\frac{19}{3} & \frac{20}{3} \\
			0 & -\frac{34}{3} & \frac{38}{3} & -\frac{40}{3}
		\end{bmatrix} \longrightarrow
		\begin{bmatrix}
			3 & 4 & -5 & 7 \\
			0 & -\frac{17}{3} & \frac{19}{3} & -\frac{20}{3} \\
			0 & 0 & 0 & 0 \\
			0 & 0 & 0 & 0
		\end{bmatrix}
		$$
		那么可得一个基础解系:
		$$
		\vect{\eta}_1 = \cvec{\frac{5}{3},\frac{19}{17},1,0},\ \vect{\eta}_2 = \cvec{-\frac{7}{3},-\frac{20}{17},0,1}
		$$
		因此方程组的解为:
		$$
		\vect{\eta} = k_1 \vect{\eta}_1 + k_2 \vect{\eta}_2
		$$
	\end{solution}
\end{problem}

\begin{problem}
	第三章习题 24

	\begin{proof}
		必要性是显然的,下面证明充分性:注意到有解
		$$
		\vect{\gamma}_0 = \cvec{a_1 + a_2 + a_3 + a_4, a_2 + a_3 + a_4, a_3 + a_4, a_4, 0}
		$$
		于是充分性得证。

		方程系数矩阵的秩
		$$
		\rank \ab(\begin{bmatrix}
			1 & -1 & 0 & 0 & 0 \\
			0 & 1 & -1 & 0 & 0 \\
			0 & 0 & 1 & -1 & 0 \\
			0 & 0 & 0 & 1 & -1 \\
			-1 & 0 & 0 & 0 & 1
		\end{bmatrix}) = 
		\rank \ab(\begin{bmatrix}
			1 & -1 & 0 & 0 & 0 \\
			0 & 1 & -1 & 0 & 0 \\
			0 & 0 & 1 & -1 & 0 \\
			0 & 0 & 0 & 1 & -1 \\
			0 & 0 & 0 & 0 & 0
		\end{bmatrix}) = 4
		$$
		因此方程组的导出组的基础解系大小为 $1$。可得
		$$
		\vect{\eta}_1 = \cvec{1,1,1,1,1}
		$$
		是一个基础解系,因此方程组的通解为
		$$
		\vect{\gamma} = \vect{\gamma}_0 + k \vect{\eta}_1 = \cvec{a_1 + a_2 + a_3 + a_4 + k, a_2 + a_3 + a_4 + k, a_3 + a_4 + k, a_4 + k, k}
		$$
	\end{proof}
\end{problem}

\begin{problem}
	第三章习题 26

	\begin{proof}
		设方程组的一个基础解系为 $\vect{\eta}_1, \vect{\eta}_2, \dots, \vect{\eta}_{n-r}$,另有 $n-r$ 个线性无关的解 $\vect{\beta}_1, \vect{\beta}_2, \dots, \vect{\beta}_{n-r}$,那么可知存在 $c_{i,j}$:
		$$
		\vect{\beta}_i = \sum_{j=1}^{n-r} c_{i,j} \vect{\eta}_j
		$$
		要证 $\vect{\beta}_1, \vect{\beta}_2, \dots, \vect{\beta}_{n-r}$ 是基础解系,只需证其与 $\vect{\eta}_1, \vect{\eta}_2, \dots, \vect{\eta}_{n-r}$ 等价,只需证每个 $\vect{\eta}_j$ 都能被 $\vect{\beta}_1, \vect{\beta}_2, \dots, \vect{\beta}_{n-r}$ 线性表出。

		设 $\vect{\eta}_i = \sum_{i=1}^{n-r} k_{i,j} \vect{\beta}_j$,那么:
		$$
		\begin{bmatrix}
			c_{1,1} & c_{1,2} & \cdots & c_{1,n-r} \\
			\vdots & \vdots & \ddots & \vdots \\
			c_{i-1,1} & c_{i-1,2} & \cdots & c_{i-1,n-r} \\
			c_{i,1} & c_{i,2} & \cdots & c_{i,n-r} \\
			c_{i+1,1} & c_{i+1,2} & \cdots & c_{i+1,n-r} \\
			\vdots & \vdots & \ddots & \vdots \\
			c_{n-r,1} & c_{n-r,2} & \cdots & c_{n-r,n-r}
		\end{bmatrix} \cvec{k_{i,1}, \vdots, k_{i-1,i}, k_{i,i}, k_{i+1,i}, \vdots, k_{i,n-r}} = \cvec{0, \vdots, 0, 1, 0, \vdots, 0}
		$$
		由 $\vect{\beta}_1, \vect{\beta}_2, \dots, \vect{\beta}_{n-r}$ 线性无关可得 $|c_{i,j}| \neq 0$,因此上面的方程组有解。于是得证。 
	\end{proof}
\end{problem}

\begin{problem}
	第三章习题 27

	\begin{proof}
		设方程组的一个基础解系为 $\vect{\gamma}_1, \vect{\gamma}_2, \dots, \vect{\gamma}_s$,方程组的一个特解为 $\vect{\eta}_0$。那么:
		$$
		\vect{\eta}_i = \vect{\eta}_0 + \sum_{j=1}^s c_{i,j} \vect{\gamma}_j
		$$
		因此
		$$
		\begin{aligned}
			\sum_{i=1}^t u_i \vect{\eta}_i & = \sum_{i=1}^t u_i \ab(\vect{\eta}_0 + \sum_{j=1}^n c_{i,j} \vect{\gamma}_j) \\
			& = \sum_{i=1}^t u_i \vect{\eta}_0 + \sum_{i=1}^t u_i \sum_{j=1}^s c_{i,j} \vect{\gamma}_j \\
			& = \vect{\eta}_0 + \sum_{j=1}^s \ab(\sum_{i=1}^t u_i c_{i,j}) \vect{\gamma}_j
		\end{aligned}
		$$
		可知也是一个解,得证。
	\end{proof}
\end{problem}

\begin{problem}
	第四章习题 1

	\begin{solution}
		\begin{enumerate}
			\item[\textbf{1)}]
			$$
			\matr{A} \matr{B} = \begin{bmatrix}
				6 & 2 & -2 \\
				6 & 1 & 0 \\
				8 & -1 & 2
			\end{bmatrix},\ \matr{A} \matr{B} - \matr{B} \matr{A} = \begin{bmatrix}
				2 & 2 & -2 \\
				2 & 0 & 0 \\
				4 & -4 & -2
			\end{bmatrix}
			$$
		\end{enumerate}
	\end{solution}
\end{problem}

\begin{problem}
	第四章习题 2

	\begin{solution}
		\begin{enumerate}
			\item[\textbf{4)}] 注意到:
			$$
			\begin{aligned}
				& \begin{bmatrix}
					\cos n \varphi & -\sin n \varphi \\
					\sin n \varphi & \cos n \varphi
				\end{bmatrix} \begin{bmatrix}
					\cos \varphi & -\sin \varphi \\
					\sin \varphi & \cos \varphi
				\end{bmatrix} \\
				= & \begin{bmatrix}
					\cos n \varphi \cos \varphi - \sin n \varphi \sin \varphi &
					-\cos n \varphi \sin \varphi - \sin n \varphi \cos \varphi \\
					\sin n \varphi \cos \varphi + \cos n \varphi \sin \varphi &
					-\sin n \varphi \sin \varphi + \cos n \varphi \cos \varphi
				\end{bmatrix} \\
				= & \begin{bmatrix}
					\cos (n+1)\varphi & -\sin (n+1)\varphi \\
					\sin (n+1)\varphi & \cos (n+1)\varphi
				\end{bmatrix}
			\end{aligned}
			$$
			而
			$$
			\begin{bmatrix}
				\cos \varphi & -\sin \varphi \\
				\sin \varphi & \cos \varphi
			\end{bmatrix}^0 = \begin{bmatrix}
				1 & 0 \\
				0 & 1
			\end{bmatrix} = \begin{bmatrix}
				\cos 0\varphi & -\sin 0\varphi \\
				\sin 0\varphi & \cos 0\varphi
			\end{bmatrix}
			$$
			因此归纳可证
			$$
			\begin{bmatrix}
				\cos \varphi & -\sin \varphi \\
				\sin \varphi & \cos \varphi
			\end{bmatrix}^n = \begin{bmatrix}
				\cos n \varphi & -\sin n \varphi \\
				\sin n \varphi & \cos n \varphi
			\end{bmatrix}
			$$

			\item[\textbf{8)}] 据观察有
			$$
			\begin{bmatrix}
				\lambda & 1 & 0 \\
				0 & \lambda & 1 \\
				0 & 0 & \lambda
			\end{bmatrix}^n = \begin{bmatrix}
				\lambda^n & n \lambda^{n-1} & \frac{n(n-1)}{2} \lambda^{n-2} \\
				0 & \lambda^n & n \lambda^{n-1} \\
				0 & 0 & \lambda^n
			\end{bmatrix}
			$$
			下面归纳证明:
			$$
			\begin{aligned}
				& \begin{bmatrix}
					\lambda^n & n \lambda^{n-1} & \frac{n(n-1)}{2} \lambda^{n-2} \\
					0 & \lambda^n & n \lambda^{n-1} \\
					0 & 0 & \lambda^n
				\end{bmatrix} \begin{bmatrix}
					\lambda & 1 & 0 \\
					0 & \lambda & 1 \\
					0 & 0 & \lambda
				\end{bmatrix} \\
				= & \begin{bmatrix}
					\lambda^{n+1} & \lambda^n + n \lambda^n & n \lambda^{n-1} + \frac{n(n-1)}{2} \lambda^{n-1} \\
					0 & \lambda^{n+1} & \lambda^n + n \lambda^n \\
					0 & 0 & \lambda^{n+1}
				\end{bmatrix} \\
				= & \begin{bmatrix}
					\lambda^{n+1} & (n+1)\lambda^n & \frac{n(n+1)}{2} \lambda^{n-1} \\
					0 & \lambda^{n+1} & (n+1)\lambda^n \\
					0 & 0 & \lambda^{n+1}
				\end{bmatrix}
			\end{aligned}
			$$
			即证。
		\end{enumerate}
	\end{solution}
\end{problem}

\begin{problem}
	第四章习题 10

	\begin{proof}
		设 $\matr{C} = [c_{i,j}] = \matr{A}^2$。那么:
		$$
		c_{i,i} = \sum_{k=1}^n a_{i,k} a_{k,i} = \sum_{k=1}^n a_{i,k}^2 = 0
		$$
		因此 $a_{i,k} = 0$,于是 $\matr{A} = \matr{O}$。
	\end{proof}
\end{problem}

\begin{problem}
	第四章习题 11

	\begin{proof}
		$$
		(\matr{A} \matr{B})^\mathrm{T} = \matr{A} \matr{B} \Leftrightarrow \matr{B}^\mathrm{T} \matr{A}^\mathrm{T} = \matr{A} \matr{B} \Leftrightarrow \matr{B} \matr{A} = \matr{A} \matr{B}
		$$
	\end{proof}
\end{problem}

\begin{problem}
	第四章习题 12

	\begin{proof}
		设 $\matr{A} = [a_{i,j}]$,那么构造
		$$
		\matr{B} = \ab[\frac{a_{i,j} + a_{j,i}}{2}],\ \matr{C} = \ab[\frac{a_{i,j} - a_{j,i}}{2}]
		$$
		显然 $\matr{A} = \matr{B} + \matr{C}$,且 $\matr{B}$ 是对称矩阵,$\matr{C}$ 是反称矩阵。
	\end{proof}
\end{problem}

\begin{problem}
	第四章习题 14

	\begin{proof}
		\begin{itemize}
			\item 充分性:将 $\matr{A}$ 按照列向量组分块:
			$$
			\matr{A} = \begin{bmatrix}
				\vect{\alpha}_1 & \vect{\alpha}_2 & \cdots & \vect{\alpha}_n
			\end{bmatrix}
			$$
			因为 $\det \matr{A} = 0$,因此存在不全为 $0$ 的 $k_1, k_2, \dots, k_n$ 使得 $\sum_{i=1}^n k_i \vect{\alpha}_i = \vect{0}$。于是构造 $\matr{B}$:
			$$
			\matr{B} = \begin{bmatrix}
				k_1 & 0 & \cdots & 0 \\
				k_2 & 0 & \cdots & 0 \\
				\vdots & \vdots & \ddots & \vdots \\
				k_n & 0 & \cdots & 0
			\end{bmatrix}
			$$
			那么 $\matr{A} \matr{B} = \matr{O}$。

			\item 必要性:若 $\matr{A} \matr{B} = \matr{O}$,而 $\det \matr{A} \neq 0$,那么 $\matr{A}^{-1}$ 存在,因此 $\matr{B} = \matr{A}^{-1} \matr{O} = \matr{O}$,与 $\matr{B}$ 非零矛盾。因此 $\det \matr{A} = 0$。
		\end{itemize}
	\end{proof}
\end{problem}

\begin{problem}
	第四章习题 16

	\begin{proof}
		将 $\matr{C}$ 按行向量组分块:
		$$
		\matr{B} = [b_{i,j}],\ \matr{C} = \begin{bmatrix}
			\vect{\gamma}_1 \\ \vect{\gamma}_2 \\ \vdots \\ \vect{\gamma}_n
		\end{bmatrix}
		$$
		可得 $\vect{\gamma}_1, \vect{\gamma}_2, \dots, \vect{\gamma}_n$ 线性无关。

		\begin{enumerate}
			\item[\textbf{1)}]
			$$
			\matr{B} \matr{C} = \begin{bmatrix}
				\sum_{j=1}^n b_{1,j} \vect{\gamma}_j \\
				\sum_{j=1}^n b_{2,j} \vect{\gamma}_j \\
				\vdots \\
				\sum_{j=1}^n b_{n,j} \vect{\gamma}_j
			\end{bmatrix} = \matr{O} \Rightarrow \sum_{j=1}^n b_{i,j} \vect{\gamma}_j = \vect{0}
			$$
			结合 $\vect{\gamma}_1, \vect{\gamma}_2, \dots, \vect{\gamma}_n$ 线性无关那么可知 $b_{i,j} = 0$,即 $\matr{B} = \matr{O}$。

			\item[\textbf{2)}]
			$$
			\matr{B} \matr{C} = \begin{bmatrix}
				\sum_{j=1}^n b_{1,j} \vect{\gamma}_j \\
				\sum_{j=1}^n b_{2,j} \vect{\gamma}_j \\
				\vdots \\
				\sum_{j=1}^n b_{n,j} \vect{\gamma}_j
			\end{bmatrix} = \matr{C} \Rightarrow \sum_{j=1}^n b_{i,j} \vect{\gamma}_j = \vect{\gamma}_i
			$$
			因为 $\vect{\gamma}_1, \vect{\gamma}_2, \dots, \vect{\gamma}_n$ 线性无关,对任何一个向量的表出系数都是唯一的,因此只有 $b_{i,j} = [i=j]$,即 $\matr{B} = \matr{E}$。
		\end{enumerate}
	\end{proof}
\end{problem}

\begin{problem}
	第四章习题 18
	
	\begin{proof}
		将 $\matr{B}$ 按照列向量组分块
		$$
		\matr{B} = \begin{bmatrix}
			\vect{\beta}_1 & \vect{\beta}_2 & \cdots & \vect{\beta}_n
		\end{bmatrix}
		$$
		那么
		$$
		\matr{A} \matr{B} = \begin{bmatrix}
			\matr{A} \vect{\beta}_1 & \matr{A} \vect{\beta}_2 & \cdots & \matr{A} \vect{\beta}_n
		\end{bmatrix} = \matr{O}
		$$
		因此 $\vect{\beta}_i$ 是其次线性方程组 $\matr{A} \vect{x} = \vect{0}$ 的解。记 $r = \rank \matr{A}$,那么 $\vect{\beta}_1, \vect{\beta}_2, \dots, \vect{\beta}_n$ 能被该方程组的一个基础解系 $\vect{\eta}_1, \vect{\eta}_2, \dots, \vect{\eta}_{n-r}$ 线性表出,因此 $\rank(\vect{\beta}_1, \vect{\beta}_2, \dots, \vect{\beta}_n) \le \rank(\vect{\eta}_1, \vect{\eta}_2, \dots, \vect{\eta}_{n-r}) = n - r$,即 $\rank \matr{A} + \rank \matr{B} \le n$。
	\end{proof}
\end{problem}