\section{线性方程组 I}

\subsection{消元法}

一个线性方程组指如下形式的方程组:

$$
\sum_{i=1}^n a_{k,i} x_i = b_k \quad (k=1,2,\dots,n)
$$

其中 $x_1,x_2,\dots,x_n$ 是未知量。也可以记作:

$$
\matr{A} \vect{x} = \vect{b}
$$

其中 $A = [a_{i,j}]_n,\ x = [x_1,x_2,\dots,x_n]^\mathrm{T},\ b = [b_1,b_2,\dots,b_n]^\mathrm{T}$。

一组 $k_1,k_2,\dots,k_n$ 在带入 $x_i = k_i$ 后使得方程组变成恒等式,那么 $(k_i)_{i=1}^n$ 就是方程组的一个\textbf{解}。所有 $(k_i)_{i=1}^n$ 的集合就是方程组的\textbf{解集}。两个具有相同解集的方程组称为\textbf{同解}的。

再定义方程组的\textbf{增广矩阵}为:

$$
\ab[\begin{array}{cccc|c}
	a_{1,1} & a_{1,2} & \dots & a_{1,n} & b_1 \\
	a_{2,1} & a_{2,2} & \dots & a_{2,n} & b_2 \\
	\vdots & \vdots & \ddots & \vdots & \vdots \\
	a_{n,1} & a_{n,2} & \dots & a_{n,n} & b_n
\end{array}]
$$

\begin{definition}[线性方程组的初等变换]
	对增广矩阵进行初等行变换,对应到方程组上的变化就是线性方程组的\textbf{初等变换}。初等变换总是把方程组变为同解的方程组。
\end{definition}

应用初等变换消元后,方程组变为下面的阶梯形式:

$$
\begin{aligned}
	\sum_{i=k}^n c_{k,i} x_i & = d_k & \quad (k=1,2,\dots,r) \\
	0 & = d_{r+1} \\
	0 & = 0 \\
	& \cdots \\
	0 & = 0
\end{aligned}
$$

其中 $c_{k,k} \neq 0$。下面进行一些讨论:

\begin{itemize}
	\item 若存在方程 $0 = d_{r+1}$ 且 $d_{r+1} \neq 0$,那么方程组无解;
	\item 如果不存在 $0 = d_{r+1}$ 或 $d_{r+1} = 0$,且 $r<n$,那么方程组有无穷多组解,且这些解由 $x_{r+1},x_{r+2},\dots,x_n$ 表示,这个解称为方程组的\textbf{一般解},$x_{r+1},x_{r+2},\dots,x_n$ 称为方程组的\textbf{自由未知量};
	\item 否则,方程组有且仅有一组解。
\end{itemize}

这些结论可以应用于齐次线性方程组:

\begin{theorem}
	对于齐次线性方程组
	$$
	\sum_{i=1}^n a_{k,i} x_i \quad (k=1,2,\dots,s)
	$$
	若 $s < n$,那么该方程组有非零解。

	\begin{proof}
		消元后 $r \le s < n$,所以方程组并非唯一解,即有非零解。
	\end{proof}
\end{theorem}

\subsection{$n$ 维向量空间}

\begin{definition}[$n$ 维向量]
	数域 $P$ 上的一个 $n$ 维向量是由数域 $P$ 中 $n$ 个元素组成的有序 $n$ 元组:
	$$
	[a_1,a_2,\dots,a_n]
	$$
	其中 $a_i$ 称为向量的\textbf{分量}。全体 $n$ 维向量的集合可记作 $P^n$。
\end{definition}

\begin{definition}[向量相等]
	称两个向量相等当且仅当两个向量的每个分量都对应相等。
\end{definition}

\begin{definition}[零向量]
	我们称 $(0,0,\dots,0)$ 为零向量,记作 $\vect{0}$。
\end{definition}

\begin{definition}[向量运算]
	定义向量的运算:

	\begin{itemize}
		\item 加法:对于 $\vect{a} = [a_1,a_2,\dots,a_n],\ \vect{b} = [b_1,b_2,\dots,b_n]$,定义 $\vect{a} + \vect{b} = [a_1 + b_1, a_2 + b_2, \dots, a_n + b_n]$;
		\item 负向量:对于 $\vect{a} = [a_1,a_2,\dots,a_n]$,定义 $-\vect{a} = [-a_1, -a_2, \dots, -a_n]$;
		\item 减法:对于 $\vect{a},\vect{b} \in P^n$,定义 $\vect{a} - \vect{b} = \vect{a} + (-\vect{b})$;
		\item 数乘:对于 $k \in P,\ \vect{a} = [a_1,a_2,\dots,a_n]$,定义 $k\vect{a} = [k a_1, k a_2, \dots, k a_n]$。
	\end{itemize}
\end{definition}

根据向量运算的定义,我们可以推出,向量加法满足交换律和结合律,数乘和加法满足结合律。

向量可以写成一行或者一列的形式,在仅讨论向量的时候它们是等价的:

$$
\vect{a} = \begin{bmatrix}
	a_1 \\ a_2 \\ \vdots \\ a_n
\end{bmatrix} = [a_1, a_2, \dots, a_n]
$$

\subsection{作业}

\begin{problem}
	第二章课后习题 21
	
	\begin{solution}
		行列式的某项 $\prod_{i=1}^{2n} a_{i,p_i}$ 非零当且仅当所有 $i$ 都有 $p_i=i \lor p_i=2n-i+1$。这说明 $p_i=i \Leftrightarrow p_{2n-i+1}=2n-i+1,\ p_i=2n-i+1 \Leftrightarrow p_{2n-i+1}=i$,也就是说把 $1,2,\dots,2n$ 按照 $(i,2n-i+1)$ 分成 $n$ 组,合法的 $p$ 一定是由 $1,2,\dots,2n$ 通过若干次交换一组的两个元素得到。

		交换一次排列奇偶性改变,那么枚举最终有几个组是逆序的,就得到:
		$$
		\begin{aligned}
			D_{2n} & = \sum_{k=0}^n \binom{n}{k} (-1)^k b^{2k} a^{2n-2k} \\
			& = a^{2n} \sum_{k=0}^n \binom{n}{k} \ab(-\frac{b^2}{a^2})^k \\
			& = a^{2n} \ab(1 - \frac{b^2}{a^2})^n \\
			& = (a^2 - b^2)^n
		\end{aligned}
		$$

	\end{solution}
\end{problem}

\begin{problem}
	第二章课后习题 23

	\begin{solution}
		求出 $f(x)$ 就是要解线性方程组:
		$$
		\sum_{i=0}^n c_i a_k^i = b_k \quad (k=1,\dots,n)
		$$
		那么这个方程组的系数矩阵为:
		$$
		\matr{A} = \begin{bmatrix}
			1 & a_1^1 & a_1^2 & \cdots & a_1^{n-1} \\
			1 & a_2^1 & a_2^2 & \cdots & a_2^{n-1} \\
			\vdots & \vdots & \vdots & \ddots & \vdots \\
			1 & a_n^1 & a_n^2 & \cdots & a_n^{n-1}
		\end{bmatrix}
		$$
		它的行列式是一个范德蒙德行列式,$\det \matr{A} = \prod_{i<j} (a_i - a_j)$,因为 $a_i$ 互不相同,所以 $\det \matr{A} \neq 0$。根据克拉默法则,方程组有唯一解,也即存在这样的 $f(x)$。
	\end{solution}
\end{problem}

\begin{problem}
	第三章课后习题 1
	\begin{solution}
		\begin{enumerate}
			\item[\textbf{1)}] 对增广矩阵进行初等行变换:
			$$
			\begin{aligned}
				\left[\begin{array}{ccccc|c}
					1 & 3 & 5 & -4 & 0 & 1 \\
					1 & 3 & 2 & -2 & 1 & -1 \\
					1 & -2 & 1 & -1 & -1 & 3 \\
					1 & -4 & 1 & 1 & -1 & 3 \\
					1 & 2 & 1 & -1 & 1 & -1
				\end{array}\right]
				& \xrightarrow{(2,3,4,5) - (1)} \left[\begin{array}{ccccc|c}
					1 & 3 & 5 & -4 & 0 & 1 \\
					0 & 0 & -3 & 2 & 1 & -2 \\
					0 & -5 & -4 & 3 & -1 & 2 \\
					0 & -7 & -4 & 5 & -1 & 2 \\
					0 & -1 & -4 & 3 & 1 & -2
				\end{array}\right] \\
				& \xrightarrow{\operatorname{SWAP}(2,5)} \left[\begin{array}{ccccc|c}
					1 & 3 & 5 & -4 & 0 & 1 \\
					0 & -1 & -4 & 3 & 1 & -2 \\
					0 & -5 & -4 & 3 & -1 & 2 \\
					0 & -7 & -4 & 5 & -1 & 2 \\
					0 & 0 & -3 & 2 & 1 & -2 \\
				\end{array}\right] \\
			\end{aligned}
			$$
			$$
			\begin{aligned}
				& \xrightarrow{(3)-5 \times (2)} \left[\begin{array}{ccccc|c}
					1 & 3 & 5 & -4 & 0 & 1 \\
					0 & -1 & -4 & 3 & 1 & -2 \\
					0 & 0 & 16 & -12 & -6 & 12 \\
					0 & -7 & -4 & 5 & -1 & 2 \\
					0 & 0 & -3 & 2 & 1 & -2 \\
				\end{array}\right] \\
				& \xrightarrow{(4)-7 \times (2)} \left[\begin{array}{ccccc|c}
					1 & 3 & 5 & -4 & 0 & 1 \\
					0 & -1 & -4 & 3 & 1 & -2 \\
					0 & 0 & 16 & -12 & -6 & 12 \\
					0 & 0 & 24 & -16 & -8 & 16 \\
					0 & 0 & -3 & 2 & 1 & -2 \\
				\end{array}\right] \\
				& \xrightarrow{(4) \times \frac{1}{8},\ \operatorname{SWAP}(3,4)} \left[\begin{array}{ccccc|c}
					1 & 3 & 5 & -4 & 0 & 1 \\
					0 & -1 & -4 & 3 & 1 & -2 \\
					0 & 0 & 3 & -2 & -1 & 2 \\
					0 & 0 & 16 & -12 & -6 & 12 \\
					0 & 0 & -3 & 2 & 1 & -2 \\
				\end{array}\right] \\
				& \xrightarrow{(4) - (3) \times \frac{16}{3}} \left[\begin{array}{ccccc|c}
					1 & 3 & 5 & -4 & 0 & 1 \\
					0 & -1 & -4 & 3 & 1 & -2 \\
					0 & 0 & 3 & -2 & -1 & 2 \\
					0 & 0 & 0 & -4/3 & -2/3 & -4/3 \\
					0 & 0 & -3 & 2 & 1 & -2 \\
				\end{array}\right] \\
				& \xrightarrow{(5) + (3)} \left[\begin{array}{ccccc|c}
					1 & 3 & 5 & -4 & 0 & 1 \\
					0 & -1 & -4 & 3 & 1 & -2 \\
					0 & 0 & 3 & -2 & -1 & 2 \\
					0 & 0 & 0 & -4/3 & -2/3 & 4/3 \\
					0 & 0 & 0 & 0 & 0 & 0 \\
				\end{array}\right] \\
				& \xrightarrow{(4) \times (-\frac{3}{4})} \left[\begin{array}{ccccc|c}
					1 & 3 & 5 & -4 & 0 & 1 \\
					0 & -1 & -4 & 3 & 1 & -2 \\
					0 & 0 & 3 & -2 & -1 & 2 \\
					0 & 0 & 0 & 1 & 1/2 & -1 \\
					0 & 0 & 0 & 0 & 0 & 0 \\
				\end{array}\right]
			\end{aligned}
			$$
			也就是说,方程组有无穷多组解,以 $x_5$ 为自由元就有:
			$$
			\begin{cases}
				x_1 = -\frac{x_5}{2} \\
				x_2 = -1 - \frac{x_5}{2} \\
				x_3 = 0 \\
				x_4 = -1 - \frac{x_5}{2} \\
			\end{cases}
			$$

			\item[\textbf{4)}] 对增广矩阵进行初等行变换:
			$$
			\begin{aligned}
				\left[\begin{array}{cccc|c}
					3 & 4 & -5 & 7 & 0 \\
					2 & -3 & 3 & -2 & 0 \\
					4 & 11 & -13 & 16 & 0 \\
					7 & -2 & 1 & 3 & 0 \\
				\end{array}\right]
				& \xrightarrow{(2) - \frac{2}{3} \times (1)} \left[\begin{array}{cccc|c}
					3 & 4 & -5 & 7 & 0 \\
					0 & -\frac{17}{3} & \frac{19}{3} & -\frac{20}{3} & 0 \\
					4 & 11 & -13 & 16 & 0 \\
					7 & -2 & 1 & 3 & 0 \\
				\end{array}\right] \\
				& \xrightarrow{(3) - \frac{4}{3} \times (1)} \left[\begin{array}{cccc|c}
					3 & 4 & -5 & 7 & 0 \\
					0 & -\frac{17}{3} & \frac{19}{3} & -\frac{20}{3} & 0 \\
					0 & \frac{17}{3} & -\frac{19}{3} & \frac{20}{3} & 0 \\
					7 & -2 & 1 & 3 & 0 \\
				\end{array}\right] \\
				& \xrightarrow{(4) - \frac{7}{3} \times (1)} \left[\begin{array}{cccc|c}
					3 & 4 & -5 & 7 & 0 \\
					0 & -\frac{17}{3} & \frac{19}{3} & -\frac{20}{3} & 0 \\
					0 & \frac{17}{3} & -\frac{19}{3} & \frac{20}{3} & 0 \\
					0 & -\frac{34}{3} & \frac{38}{3} & -\frac{40}{3} & 0 \\
				\end{array}\right] \\
				& \xrightarrow{(3) + (2),\ (4) - 2 \times (2)} \left[\begin{array}{cccc|c}
					3 & 4 & -5 & 7 & 0 \\
					0 & -\frac{17}{3} & \frac{19}{3} & -\frac{20}{3} & 0 \\
					0 & 0 & 0 & 0 & 0 \\
					0 & 0 & 0 & 0 & 0 \\
				\end{array}\right] \\
			\end{aligned}
			$$
			故原方程有无穷多组解。将 $x_3,x_4$ 定为自由元就有:
			$$
			\begin{cases}
				x_1 = \frac{3}{17} x_3 - \frac{13}{17} x_4 \\
				x_2 = \frac{19}{17} x_3 - \frac{17}{20} x_4
			\end{cases}
			$$

			\item[\textbf{6)}] 对增广矩阵进行初等行变换:
			$$
			\begin{aligned}
				\left[\begin{array}{cccc|c}
					1 & 2 & 3 & -1 & 1 \\
					3 & 2 & 1 & -1 & 1 \\
					2 & 3 & 1 & 1 & 1 \\
					2 & 2 & 2 & -1 & 1 \\
					5 & 5 & 2 & 0 & 2
				\end{array}\right]
				& \xrightarrow{(2) - 3 \times (1)} \left[\begin{array}{cccc|c}
					1 & 2 & 3 & -1 & 1 \\
					0 & -4 & -8 & 2 & -2 \\
					2 & 3 & 1 & 1 & 1 \\
					2 & 2 & 2 & -1 & 1 \\
					5 & 5 & 2 & 0 & 2
				\end{array}\right] \\
				& \xrightarrow{(3) - 2 \times (1)} \left[\begin{array}{cccc|c}
					1 & 2 & 3 & -1 & 1 \\
					0 & -4 & -8 & 2 & -2 \\
					0 & -1 & -5 & 3 & -1 \\
					2 & 2 & 2 & -1 & 1 \\
					5 & 5 & 2 & 0 & 2
				\end{array}\right] \\
				& \xrightarrow{(4) - 2 \times (1)} \left[\begin{array}{cccc|c}
					1 & 2 & 3 & -1 & 1 \\
					0 & -4 & -8 & 2 & -2 \\
					0 & -1 & -5 & 3 & -1 \\
					0 & -2 & -4 & 1 & -1 \\
					5 & 5 & 2 & 0 & 2
				\end{array}\right] \\
			\end{aligned}
			$$
			$$
			\begin{aligned}
				& \xrightarrow{(5) - 5 \times (1)} \left[\begin{array}{cccc|c}
					1 & 2 & 3 & -1 & 1 \\
					0 & -4 & -8 & 2 & -2 \\
					0 & -1 & -5 & 3 & -1 \\
					0 & -2 & -4 & 1 & -1 \\
					0 & -5 & -13 & 5 & -3
				\end{array}\right] \\
				& \xrightarrow{\operatorname{SWAP}(2,3)} \left[\begin{array}{cccc|c}
					1 & 2 & 3 & -1 & 1 \\
					0 & -1 & -5 & 3 & -1 \\
					0 & -4 & -8 & 2 & -2 \\
					0 & -2 & -4 & 1 & -1 \\
					0 & -5 & -13 & 5 & -3
				\end{array}\right] \\
				& \xrightarrow{3 - 4 \times (2)} \left[\begin{array}{cccc|c}
					1 & 2 & 3 & -1 & 1 \\
					0 & -1 & -5 & 3 & -1 \\
					0 & 0 & 12 & -10 & 2 \\
					0 & -2 & -4 & 1 & -1 \\
					0 & -5 & -13 & 5 & -3
				\end{array}\right] \\
				& \xrightarrow{4 - 2 \times (2)} \left[\begin{array}{cccc|c}
					1 & 2 & 3 & -1 & 1 \\
					0 & -1 & -5 & 3 & -1 \\
					0 & 0 & 12 & -10 & 2 \\
					0 & 0 & 6 & -5 & 1 \\
					0 & -5 & -13 & 5 & -3
				\end{array}\right] \\
				& \xrightarrow{5 - 5 \times (2)} \left[\begin{array}{cccc|c}
					1 & 2 & 3 & -1 & 1 \\
					0 & -1 & -5 & 3 & -1 \\
					0 & 0 & 12 & -10 & 2 \\
					0 & 0 & 6 & -5 & 1 \\
					0 & 0 & 12 & -10 & 2
				\end{array}\right] \\
			\end{aligned}
			$$

			此时容易发现后三个方程组是等价的,最后化简为:
			$$
			\left[\begin{array}{cccc|c}
				1 & 2 & 3 & -1 & 1 \\
				0 & -1 & -5 & 3 & -1 \\
				0 & 0 & 12 & -10 & 2 \\
				0 & 0 & 0 & 0 & 0 \\
				0 & 0 & 0 & 0 & 0
			\end{array}\right]
			$$

			那么方程有无穷多组解。设 $x_4$ 为自由元就有:
			$$
			\begin{cases}
				x_1 = \frac{1 + 5x_4}{6} \\
				x_2 = \frac{1 - 7x_4}{6} \\
				x_3 = \frac{1 + 5x_4}{6}
			\end{cases}
			$$
		\end{enumerate}
	\end{solution}
\end{problem}

\begin{problem}
	验证课本 P80 (6)(7)(8)(9)。
	
	\begin{solution}
		\begin{enumerate}
			\item[\textbf{6)}]
			$$
			\begin{aligned}
				k(\vect{\alpha} + \vect{\beta})
				& = k \begin{bmatrix} a_1 + b_1 \\ a_2 + b_2 \\ \vdots \\ a_n+b_n \end{bmatrix}
				= \begin{bmatrix} k(a_1 + b_1) \\ k(a_2 + b_2) \\ \vdots \\ k(a_n + b_n) \end{bmatrix}
				= \begin{bmatrix} k a_1 + k b_1 \\ k a_2 + k b_2 \\ \vdots \\ k a_n + k b_n \end{bmatrix} \\
				& = \begin{bmatrix} k a_1 \\ k a_2 \\ \vdots \\ k a_n \end{bmatrix}
				+ \begin{bmatrix} k b_1 \\ k b_2 \\ \vdots \\ k b_n \end{bmatrix}
				= k \vect{\alpha} + k \vect{\beta} \\
			\end{aligned}
			$$

			\item[\textbf{7)}]
			$$
			\begin{aligned}
				(k + l) \vect{a}
				& = \begin{bmatrix} (k + l) a_1 \\ (k + l) a_2 \\ \vdots \\ (k + l) a_n \end{bmatrix}
				= \begin{bmatrix} k a_1 + l a_1 \\ k a_2 + l a_2 \\ \vdots \\ k a_n + l a_n \end{bmatrix}
				= \begin{bmatrix} k a_1 \\ k a_2 \\ \vdots \\ k a_n \end{bmatrix}
				+ \begin{bmatrix} l a_1 \\ l a_2 \\ \vdots \\ l a_n \end{bmatrix}
				= k \vect{a} + l \vect{a}
			\end{aligned}
			$$

			\item[\textbf{8)}]
			$$
			\begin{aligned}
				k (l \vect{a})
				& = k \begin{bmatrix} l a_1 \\ l a_2 \\ \vdots \\ l a_n \end{bmatrix}
				= \begin{bmatrix} k l a_1 \\ k l a_2 \\ \vdots \\ k l a_n \end{bmatrix}
				= \begin{bmatrix} (k l) a_1 \\ (k l) a_2 \\ \vdots \\ (k l) a_n \end{bmatrix}
				= (k l) \vect{a}
			\end{aligned}
			$$

			\item[\textbf{9)}]
			$$
			\begin{aligned}
				1 \vect{a} = \begin{bmatrix} 1 \cdot a_1 \\ 1 \cdot a_2 \\ \vdots \\ 1 \cdot a_n \end{bmatrix}
				= \begin{bmatrix} a_1 \\ a_2 \\ \vdots \\ a_n \end{bmatrix}
				= \vect{a}
			\end{aligned}
			$$
		\end{enumerate}
	\end{solution}
\end{problem}
