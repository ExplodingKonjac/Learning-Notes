\section{行列式 I}

\subsection{定义}

\begin{definition}[排列的定义]
	由 $1,2,\cdots,n$ 组成的元素互不相同的有序数组称作 $n$ 阶排列。所有 $n$ 阶排列组成集合记作 $\mathscr{P}_n$。
\end{definition}

\begin{definition}[行列式的定义]
	对于方阵 $\matr{A} = [a_{i,j}]_{n \times n}$,定义其行列式
	$$
	\det(\matr{A}) = |a_{i,j}|_{n \times n} = \sum_{p \in \mathscr{P}_n} (-1)^{\tau(p)} \prod_{i = 1}^n a_{i,p_i}
	$$
\end{definition}

\subsection{性质}

可以使用的性质:

\begin{property}
	$$
	\det(\matr{A}) = \det(\matr{A}^{\text{T}})
	$$
\end{property}

\begin{property}
	$$
	\begin{vmatrix}
		\vect{a_1} \\
		\vdots \\
		k \vect{a_i} \\
		\vdots \\
		\vect{a_n}
	\end{vmatrix} =
	k \begin{vmatrix}
		\vect{a_1} \\
		\vdots \\
		\vect{a_i} \\
		\vdots \\
		\vect{a_n}
	\end{vmatrix}
	$$
\end{property}

\begin{property}
	$$
	\begin{vmatrix}
		\vect{a_1} \\
		\vdots \\
		\vect{b}+\vect{c} \\
		\vdots \\
		\vect{a_n}
	\end{vmatrix} =
	\begin{vmatrix}
		\vect{a_1} \\
		\vdots \\
		\vect{b} \\
		\vdots \\
		\vect{a_n}
	\end{vmatrix} +
	\begin{vmatrix}
		\vect{a_1} \\
		\vdots \\
		\vect{c} \\
		\vdots \\
		\vect{a_n}
	\end{vmatrix}
	$$
\end{property}

\begin{property}
	行列式中有两行/两列成比例时,行列式为 $0$。
\end{property}

\begin{property}
	将行列式一行/一列的倍数加到另一行/一列上,行列式不变。
\end{property}

\begin{property}
	交换行列式中两行/两列,行列式取反。
\end{property}

交换两行/两列、将一行/一列的倍数加到另一行/一列上,将一行/一列乘一个倍数,这些操作称为初等行/列变换。

\subsection{作业}

\begin{problem}
	课后习题 1
	\begin{solution}
		\begin{enumerate}
			\item $\tau(1,3,4,7,8,2,6,9,5) = 4 + 2 + 4 = 10$,是偶排列;
			\item $\tau(2,1,7,9,8,6,3,5,4) = 1 + 1 + 3 + 4 + 4 + 5 = 18$,是偶排列;
			\item $\tau(9,8,7,6,5,4,3,2,1) = 9 \times 8 / 2 = 36$,是偶排列。
		\end{enumerate}
	\end{solution}
\end{problem}

\begin{problem}
	课后习题 2
	\begin{solution}
		\begin{enumerate}
			\item $i = 8, k = 3$;
			\item $i = 3, k = 6$。
		\end{enumerate}
	\end{solution}
\end{problem}

\begin{problem}
	课后习题 4
	\begin{solution}
		$$
		\tau(n,n-1,\cdots,1) = \frac{n(n-1)}{2}
		$$
		当 $n = 4k \text{ 或 } 4k-3\ (k \in \mathbb{Z}^+)$ 时,该排列为偶排列,否则为奇排列。
	\end{solution}
\end{problem}

\begin{problem}
	课后习题 5
	\begin{solution}
		将原排列反转,所有顺序对变成逆序对,所有逆序对变成顺序对。因此:
		$$
		\tau(x_n,x_{n-1},\cdots,x_1) = \frac{n(n-1)}{2} - \tau(x_1,x_2,\cdots,x_n) = \frac{n(n-1)}{2} - k
		$$
	\end{solution}
\end{problem}

\begin{problem}
	课后习题 6
	\begin{solution}
		$a_{23} a_{31} a_{42} a_{56} a_{14} a_{65} = a_{14} a_{23} a_{31} a_{42} a_{56} a_{65}$,带符号为正。

		$a_{32} a_{43} a_{14} a_{51} a_{66} a_{25} = a_{14} a_{25} a_{32} a_{43} a_{51} a_{66}$,带符号为正。
	\end{solution}
\end{problem}

\begin{problem}
	课后习题 7
	\begin{solution}
		\begin{itemize}
			\item $a_{11} a_{23} a_{32} a_{44}$
			\item $a_{12} a_{23} a_{34} a_{41}$
			\item $a_{14} a_{23} a_{31} a_{42}$
		\end{itemize}
	\end{solution}
\end{problem}

\begin{problem}
	课后习题 8
	\begin{solution}
		\begin{enumerate}
			\item $\displaystyle\text{原式} = \begin{vmatrix}
				a_{11} & a_{12} & a_{13} & a_{14} \\
				b_{21} & b_{22} & b_{23} & b_{24} \\
				a_{31} & a_{32} & a_{33} & a_{34} \\
				a_{41} & a_{42} & a_{43} & a_{44}
			\end{vmatrix}$
			\item $\displaystyle\text{原式} = \begin{vmatrix}
				a_{11} & a_{12} & a_{13} & a_{14} \\
				a_{31} & a_{32} & a_{33} & a_{34} \\
				a_{21} & a_{22} & a_{23} & a_{24} \\
				a_{41} & a_{42} & a_{43} & a_{44}
			\end{vmatrix}$
			\item $\displaystyle\text{原式} = \begin{vmatrix}
				a_{11} & a_{12} & a_{13} & a_{14} \\
				a_{31} & a_{32} & a_{33} & a_{34} \\
				a_{31} & a_{32} & a_{33} & a_{34} \\
				a_{41} & a_{42} & a_{43} & a_{44}
			\end{vmatrix} = 0$
			\item $\displaystyle\text{原式} = \begin{vmatrix}
				a_{11} & a_{13} & a_{13} & a_{14} \\
				a_{21} & a_{23} & a_{23} & a_{24} \\
				a_{31} & a_{33} & a_{33} & a_{34} \\
				a_{41} & a_{43} & a_{43} & a_{44}
			\end{vmatrix} = 0$
			\item $\displaystyle\text{原式} = \begin{vmatrix}
				a_{11} & a_{12} & a_{13} & a_{14} \\
				b_{21} & b_{22} & b_{23} & b_{24} \\
				a_{31} & a_{32} & a_{33} & a_{34} \\
				a_{41} & a_{42} & a_{43} & a_{44}
			\end{vmatrix}$
		\end{enumerate}
	\end{solution}
\end{problem}

\begin{problem}
	课后习题 9
	\begin{solution}
		\begin{enumerate}
			\item 仅 $a_{1,n} a_{2,n-1} \cdots a_{n,1}$ 项非 $0$,故原行列式为 $-\prod_{i=1}^n i$;
			\item 仅 $a_{1,2} a_{2,3} \cdots a_{n-1,n} a_{n,1}$ 项非 $0$,故原行列式为 $(-1)^{n-1} \prod_{i=1}^n i$;
			\item 仅 $a_{1,n-1} a_{2,n-2} \cdots a_{n-1,1} a_{n,n}$ 项非 $0$,故原行列式为 $(-1)^{(n-1)(n-2)/2} \prod_{i=1}^n i$。
		\end{enumerate}
	\end{solution}	
\end{problem}

\begin{problem}
	课后习题 10
	\begin{proof}
		由定义知,行列式每项中不包含相同行或者相同列的元素。

		而该行列式中不为 $0$ 的项,必然只能选择前两行或者前两列的元素。这样最多只能选择 $4$ 个元素,这是一个 $5$ 阶行列式,故不存在不为 $0$ 的项。

		因此该行列式为 $0$。
	\end{proof}
\end{problem}

\begin{problem}
	课后习题 11
	\begin{solution}
		\begin{enumerate}
			\item $x^4$ 系数:仅 $a_{11} a_{22} a_{33} a_{44}$ 项包含 $x^4$,故系数为 $2$。
			\item $x^3$ 系数:仅 $a_{12} a_{21} a_{33} a_{44}$ 项包含 $x^3$,故系数为 $-1$。
		\end{enumerate}
	\end{solution}
\end{problem}