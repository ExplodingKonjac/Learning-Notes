\section{线性空间 III}

\subsection{线性空间的维与基}

\begin{definition}[线性空间的维]
	对于线性空间 $V$,如果其中存在 $n$ 个线性无关的向量,并且任意 $n+1$ 个向量都线性相关,那么称 $V$ 是 $n$ 维的,记作 $\dim V = n$。特别地,如果线性空间中可以存在任意多个线性无关的向量,那么称 $V$ 是无限维的。
\end{definition}

\begin{definition}[线性空间的基]
	$n$ 维线性空间 $V$ 中的 $n$ 个线性无关的向量 $\vect{\varepsilon}_1, \vect{\varepsilon}_2, \dots, \vect{\varepsilon}_n$ 称为线性空间 $V$ 的一组基。设 $\vect{\alpha} \in V$,那么 $\vect{\alpha}$ 能被这组基唯一地线性表出:
	$$
	\vect{\alpha} = a_1 \vect{\varepsilon}_1 + a_2 \vect{\varepsilon}_2 + \cdots + a_n \vect{\varepsilon}_n
	$$
	那么称 $(a_1, a_2, \dots, a_n)$ 是 $\vect{\alpha}$ 在基 $\vect{\varepsilon}_1, \vect{\varepsilon}_2, \dots, \vect{\varepsilon}_n$ 下的坐标。
\end{definition}

线性空间的基和维的关系可由如下定理刻画:

\begin{theorem}
	若线性空间 $V$ 中有 $n$ 个线性无关的向量 $\vect{\alpha}_1, \vect{\alpha}_2, \dots, \vect{\alpha}_n$,且 $V$ 中任何一个向量都能被它们线性表出,那么 $\dim V = n$ 且 $\vect{\alpha}_1, \vect{\alpha}_2, \dots, \vect{\alpha}_n$ 是 $V$ 的一组基。

	\begin{proof}
		只需证明任意 $n+1$ 个向量组 $\vect{\eta}_1, \vect{\eta}_2, \dots, \vect{\eta}_{n+1}$ 都线性相关即可。
		
		而根据向量组的性质,因为 $\vect{\eta}_1, \vect{\eta}_2, \dots, \vect{\eta}_{n+1}$ 能被 $\vect{\alpha}_1, \vect{\alpha}_2, \dots, \vect{\alpha}_n$ 线性表出,并且 $n+1 > n$,因此 $\vect{\eta}_1, \vect{\eta}_2, \dots, \vect{\eta}_{n+1}$ 一定线性相关。
	\end{proof}
\end{theorem}

对于线性空间的两组基,可以研究它们之间的关系:

\begin{definition}
	设 $\vect{\varepsilon}_1, \vect{\varepsilon}_2, \dots, \vect{\varepsilon}_n$ 和 $\vect{\varepsilon'}_1, \vect{\varepsilon'}_2, \dots, \vect{\varepsilon'}_n$ 是线性空间 $V$ 的两组基,且满足:
	$$
	\rvec{\vect{\varepsilon'}_1, \vect{\varepsilon'}_2, \cdots, \vect{\varepsilon'}_n} = \cvec{\vect{\varepsilon}_1, \vect{\varepsilon}_2, \cdots, \vect{\varepsilon}_n} \begin{bmatrix}
		a_{1,1}, & a_{1,2} & \cdots & a_{1,n} \\
		a_{2,1}, & a_{2,2} & \cdots & a_{2,n} \\
		\vdots & \vdots & \ddots & \vdots \\
		a_{n,1}, & a_{n,2} & \cdots & a_{n,n}
	\end{bmatrix}
	$$
	这称为线性空间 $V$ 上的一组\textbf{基变换}。也称矩阵
	$$
	A = [a_{i,j}]
	$$
	为 $\vect{\varepsilon}_1, \vect{\varepsilon}_2, \dots, \vect{\varepsilon}_n$ 到 $\vect{\varepsilon'}_1, \vect{\varepsilon'}_2, \dots, \vect{\varepsilon'}_n$ 的\textbf{过渡矩阵}。显然过渡矩阵是满秩、可逆的。
\end{definition}

\subsection{线性子空间}

\begin{definition}
	对于线性空间 $V$,如果它的一个子集 $W \subset V$ 关于与 $V$ 相同的数域和运算也构成线性空间,那么称 $W$ 为 $V$ 的一个\textbf{线性子空间}
\end{definition}

