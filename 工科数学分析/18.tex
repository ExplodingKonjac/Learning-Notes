\section{不定积分 III}

\subsection{有理函数的积分}

我们称两个多项式的商为\textbf{有理函数}:

$$
\frac{P(x)}{Q(x)} = \frac{\sum_{i=0}^n a_i x^i}{\sum_{i=0}^m b_i x^i}
$$

当有理函数为假分式(即 $n \ge m$)时,可以提取出一个商,化成真分式。因此我们只考虑 $n < m$ 的有理函数。为了方便处理,我们希望能够将有理函数分解成若干个部分分式。

\begin{theorem}[代数基本定理]
	对于任意实系数多项式,都可以分解成若干个一次或二次的不可约多项式
\end{theorem}

在代数基本定理的基础上,可以探究出一些有理函数分界成部分分式的规律。

\begin{itemize}
	\item 若分母含有 $(x - a)^k$,那么分解后必定含有这些项:
	$$
	\frac{A_1}{x - a}, \frac{A_2}{(x - a)^2}, \dots, \frac{A_k}{(x - a)^k}
	$$
	\item 若分母含有 $(x^2 + p x + q)^k$,那么分解之后必定含有这些项:
	$$
	\frac{M_1 x + N_1}{x^2 + p x + q}, \frac{M_2 x + N_2}{(x^2 + p x + q)^2}, \dots, \frac{M_k x + N_k}{(x^2 + p x + q)^k}
	$$
\end{itemize}

通过待定系数法,可以求出所有 $A_i, M_i, N_i$。那么剩下就只需要解决三类积分:

\begin{itemize}
	\item 多项式:容易积分;
	\item $\frac{A}{(x - a)^k}$:是一个幂函数的平移,容易积分;
	\item $\frac{M x + N}{(x^2 + p x + q)^k}$:因为 $x^2 + p x + q = \ab(x + \frac{p}{2})^2 + \ab(q - \frac{p^2}{4})$,因此可以令 $t \gets x + \frac{p}{2}$。设 $x^2 + p x + q = t^2 + a^2,\ M x + N = M t + b$。
	$$
	\begin{aligned}
		\int \frac{M x + N}{(x^2 + p x + q)^k} \dif x & = \int \frac{M t + b}{(t^2 + a^2)^k} \dif t \\
		& = M \int \frac{t}{(t^2 + a^2)^k} \dif t + \int \frac{b}{(t^2 + a^2)^k} \dif t \\
	\end{aligned}
	$$
	对这两部分分开考虑。先考虑简单的:
	$$
	\begin{aligned}
		I_1 & = \int \frac{\dif t}{t^2 + a^2} = \frac{1}{a} \arctan \frac{t}{a} + C \\
		I_k & = \int \frac{\dif t}{(t^2 + a^2)^k} = \frac{1}{a^2} \int \frac{(t^2 + a^2) - t^2}{(t^2 + a^2)^k} \dif t \\
		& = \frac{1}{a^2} I_{n-1} - \frac{1}{a^2} \int \frac{t^2}{(t^2 + a^2)^k} \dif t \\
		& = \frac{1}{a^2} I_{n-1} + \frac{1}{2 a^2 (n-1)} \int t \dif \ab(\frac{1}{(t^2 + a^2)^{n-1}}) \\
		& = \frac{1}{a^2} I_{n-1} + \frac{1}{2 a^2 (n-1)} \ab(\frac{t}{(t^2 + a^2)^{n-1}} - I_{n-1}) \\
	\end{aligned}
	$$
	于是可以进行递推。同时在这个推导过程中实际上也解决了 $\displaystyle \int \frac{t}{(t^2 + a^2)^k} \dif t$ 的求法。于是这个积分被解决了。
\end{itemize}

至此我们成功证明,所有有理函数都可积,并且原函数是初等函数。

\subsection{三角有理式的积分}

对于三角函数和常数经过有限次四则运算得到的函数,我们称之为三角有理式。一般记作 $R(\sin x, \cos x)$。

我们知道万能公式:

$$
\begin{gathered}
	\sin x = \frac{2 \tan \frac{x}{2}}{1 + \tan^2 \frac{x}{2}} \\
	\cos x = \frac{1 - \tan^2 \frac{x}{2}}{1 + \tan^2 \frac{x}{2}}	
\end{gathered}
$$

同时,若记 $u = \tan \frac{x}{2}$,我们可以得到 $\dif x = \dif \ab(2 \arctan u) = \frac{2}{1 + u^2} \dif u$,也是一个有理函数。因此三角有理式的积分在使用万能公式代换之后,可以转换为关于 $u$ 的有理函数积分。

\subsection{简单无理函数的积分}

对于形式为 $R\ab(x, \sqrt[n]{ax + b}), R\ab(x, \sqrt[n]{\frac{ax + b}{cx + e}})$ 的无理函数,可以使用换元 $t \gets \sqrt[n]{*}$ 的方法消去根号。

\subsection{作业}

\begin{problem}
	课后作业 6.1.2

	\begin{solution}
		\begin{enumerate}
			\item[\textbf{2)}]
			$$
			\text{原式} = \int \E^x \dif x - 2 \int \sin x \dif x + 2 \int x^{1.5} \dif x = \E^x + 2 \cos x + \frac{4}{5} x^{2.5} + C
			$$
			\item[\textbf{4)}]
			$$
			\text{原式} = \int \frac{1 - \cos^2 x}{\cos x} \dif x = \int \ab(\sec^2 x - 1) \dif x = \tan x - x + C
			$$
			\item[\textbf{6)}]
			$$
			\text{原式} = \int \frac{\cos^2 x - \sin^2 x}{\sin^2 x \cos^2 x} \dif x = \int \ab(\csc^2 x - \sec^2 x) \dif x = -\cot x - \tan x + C
			$$
			\item[\textbf{8)}]
			$$
			\text{原式} = \int \sec^2 x + \int \frac{\sin x}{\cos^2 x} \dif x = \tan x - \int \frac{\dif \cos x}{\cos^2 x} = \tan x + \sec x + C
			$$
			\item[\textbf{10)}]
			$$
			\text{原式} = \int \ab(\E^{2x} + \E^x + 1) \dif x = \frac{1}{2} \E^{2x} + \E^x + x + C
			$$
			\item[\textbf{12)}]
			$$
			\text{原式} = \frac{\sgn x}{2} x^2 + C
			$$
			\item[\textbf{14)}]
			$$
			\text{原式} = \int \frac{\E^x \dif x}{\E^x (\E^x - 1)} = \int \ab(\frac{1}{\E^x - 1} - \frac{1}{\E^x}) \dif \E^x = \ln (\E^x - 1) - x + C
			$$
			\item[\textbf{16)}]
			$$
			\begin{aligned}
				\text{原式} & = \int 2 \sin x \cos x (4 \cos^3 x - 3 \cos x) \dif x \\
				& = -2 \int (4 \cos^4 x - 3 \cos^2 x) \dif \cos x \\
				& = -\frac{8}{5} \cos^5 x + 2 \cos^3 x + C
			\end{aligned}
			$$
			\item[\textbf{18)}]
			$$
			\text{原式} = \int (x^3 - 3x^{2.5} + 3x^2 - x^{1.5}) \dif x = \frac{1}{4} x^4 - \frac{6}{7} x^{3.5} + x^3 - \frac{2}{5} x^{2.5} + C
			$$
			\item[\textbf{20)}]
			$$
			\text{原式} = \int \frac{\dif x}{1 + 2 \cos^2 x - 1} = \frac{1}{2} \int \sec^2 x \dif x = \frac{1}{2} \tan x + C
			$$
		\end{enumerate}
	\end{solution}
\end{problem}

\begin{problem}
	课后习题 6.2.1

	\begin{solution}
		\begin{enumerate}
			\item[\textbf{3)}]
			$$
			\begin{aligned}
				\text{原式} & = \int \frac{\dif x}{1 - 2 \sin \frac{x}{2} \cos \frac{x}{2}} = \int \frac{\dif x}{\ab(\sin \frac{x}{2} + \cos \frac{x}{2})^2} \\
				& = \int \frac{\dif x}{2 \cos^2 \ab(\frac{x}{2} - \frac{\pi}{4})} = \int \frac{\dif \ab(\frac{x}{2} - \frac{\pi}{4})}{\cos^2 (\frac{x}{2} - \frac{\pi}{4})} \\
				& = \tan\ab(\frac{x}{2} - \frac{\pi}{4}) + C
			\end{aligned}
			$$
			\item[\textbf{4)}]
			$$
			\begin{aligned}
				\text{原式} & = \int \frac{2(x + 2)}{((x + 2)^2 + 1)^2} \dif x = \int \frac{\dif (x + 2)^2}{((x + 2)^2 + 1)^2} \\
				& = -\frac{1}{(x + 2)^2 + 1}
			\end{aligned}
			$$
			\item[\textbf{6)}]
			$$
			\begin{aligned}
				\text{原式} & = -\frac{1}{3} \int (5 - 3x)^{-\frac{1}{3}} \dif (5 - 3x) = -\frac{1}{2} (5 - 3x)^{\frac{2}{3}}
			\end{aligned}
			$$
			\item[\textbf{7)}]
			$$
			\begin{aligned}
				\text{原式} & = \int \frac{\dif \ab(\ln x)}{\ln x \ln \ln x}  = \int \frac{\dif (\ln \ln x)}{\ln \ln x} = \ln \ln \ln x + C
			\end{aligned}
			$$
			\item[\textbf{8)}]
			$$
			\begin{aligned}
				\text{原式} & = \int \frac{\E^x \dif x}{\E^{2x} + 1} = \int \frac{\dif (\E^x)}{(\E^x)^2 + 1} = \arctan \E^x + C
			\end{aligned}
			$$
			\item[\textbf{10)}]
			$$
			\begin{aligned}
				\text{原式} & = 2 \int \arctan \sqrt{x} \dif \ab(\arctan \sqrt{x}) = \ab(\arctan \sqrt{x})^2
			\end{aligned}
			$$
			\item[\textbf{12)}]
			$$
			\begin{aligned}
				\text{原式} & = \int \frac{\dif \ab(\sin t)}{1 + \sqrt{1 - \sin^2 t}} = \int \frac{\cos t}{1 + \cos t} \dif t \\
				& = \int \ab(1 - \frac{1}{1 + \cos t}) \dif t = t - \tan \frac{t}{2} + C \\
				& = \arcsin x - \tan \frac{\arcsin x}{2} + C 
			\end{aligned}
			$$
			\item[\textbf{14)}]
			$$
			\begin{aligned}
				\text{原式} & = \int \ab(a^2 \tan^2 t + a^2)^{-\frac{3}{2}} \dif \ab(a \tan t) \\
				& = \int \frac{\cos^3 t}{a^3} \cdot \frac{a}{\cos^2 t} \dif t \\
				& = \frac{\sin t}{a} + C = \frac{x}{a^2 \sqrt{a^2 + x^2}} + C
			\end{aligned}
			$$
			\item[\textbf{15)}]
			$$
			\begin{aligned}
				\text{原式} & = \int \sqrt{3^2 - (x + 2)^2} \dif x \\
				& = \int 3 \sqrt{1 - \sin^2 t} \dif \ab(3 \sin t - 2) \\
				& = 9 \int \cos^2 t \dif t  = \frac{9}{2} \int \ab(\cos 2t + 1) \dif t \\
				& = \frac{9}{2} t + \frac{9}{2} \sin t \cos t + C \\
				& = \frac{9}{2} \arcsin \frac{x + 2}{3} + \frac{x + 2}{2} \sqrt{5 - 4x - x^2} + C
			\end{aligned}
			$$
		\end{enumerate}
	\end{solution}
\end{problem}

\begin{problem}
	课后习题 6.2.2
	
	\begin{solution}
		\begin{enumerate}
			\item[\textbf{2)}]
			$$
			\begin{aligned}
				\text{原式} & = x \ln(1 + x) - \int x \dif \ab(\ln(1 + x)) \\
				& = x \ln(1 + x) - \int \frac{x}{1 + x} \dif x \\
				& = x \ln(1 + x) - x + \ln(1 + x) + C
			\end{aligned}
			$$
			\item[\textbf{3)}]
			$$
			\begin{aligned}
				\text{原式} & = \int \arctan x \dif x - \int \arctan x \cdot \frac{\dif x}{1 + x^2} \\
				& = \ab(x \arctan x - \int x \dif (\arctan x)) - \int \arctan x \dif \arctan x \\
				& = \ab(x \arctan x - \int \frac{x}{1 + x^2} \dif x) - \frac{1}{2} (\arctan x)^2 \\
				& = \ab(x \arctan x - \frac{1}{2} \int \frac{\dif \ab(x^2)}{1 + x^2}) - \frac{1}{2} (\arctan x)^2 \\
				& = x \arctan x - \frac{1}{2} \ln(1 + x^2) - \frac{1}{2} (\arctan x)^2 + C
			\end{aligned}
			$$
			\item[\textbf{5)}]
			$$
			\begin{aligned}
				\text{原式} & = -\int x^2 \dif \ab(\E^{-x}) \\
				& = -x^2 \E^{-x} + \int \E^{-x} \dif \ab(x^2) = -x^2 \E^x + 2 \int x \E^{-x} \dif x \\
				& = -x^2 \E^{-x} - 2 \int x \dif \E^{-x} = -x^2 \E^{-x} - 2x \E^{-x} + 2 \int \E^{-x} \dif x \\
				& = -x^2 \E^{-x} - 2x \E^{-x} - 2 \E^{-x} + C
			\end{aligned}
			$$
			\item[\textbf{6)}]
			$$
			\begin{aligned}
				\text{原式} & = \int \E^t \dif \ab(t^3) = 3 \int t^2 \E^t \dif t \\
				& = 3 \E^t (t^2 - 2t + 2) + C \\
				& = 3 \E^{\sqrt[3]{x}} \ab(x^{\frac{2}{3}} - 2 \sqrt[3]{x} + 2) + C
			\end{aligned}
			$$
			\item[\textbf{8)}]
			$$
			\begin{aligned}
				\text{原式} & = \int \frac{1}{2} \ab(\frac{1}{x} - \frac{x^9}{2 + x^{10}}) \dif x \\
				& = \frac{1}{2} \int \frac{\dif x}{x} - \frac{1}{20} \int \frac{\dif \ab(x^{10})}{2 + x^{10}} \\
				& = \frac{1}{2} \ln x - \frac{1}{20} \ln \ab(2 + x^{10})
			\end{aligned}
			$$
			\item[\textbf{9)}]
			$$
			\begin{aligned}
				\text{原式} & = \int \frac{1}{2x^4} \cdot \frac{2 x^3}{\sqrt{1 + x^4}} \dif x \\
				& = \int \frac{1}{2x^4} \dif \ab(\sqrt{1 + x^4}) \\
				& = \int \frac{\dif t}{2(t^2 - 1)} = \frac{1}{4} \int \ab(\frac{1}{t - 1} - \frac{1}{t + 1}) \dif t \\
				& = \frac{1}{4} \ln |t - 1| - \frac{1}{4} \ln |t + 1| + C \\
				& = \frac{1}{4} \ln \frac{\sqrt{1 + x^4} - 1}{\sqrt{1 + x^4} + 1} + C
			\end{aligned}
			$$
		\end{enumerate}
	\end{solution}
\end{problem}

\begin{problem}
	课后习题 6.2.3

	\begin{solution}
		由题可知 $f(x) = \ab(\frac{\sin x}{x})' = \frac{x \cos x - \sin x}{x^2}$,那么:
		$$
		\int x f'(x) \dif x = \int x \dif f(x) = x f(x) - \int f(x) \dif x = \frac{x \cos x - \sin x}{x} - \frac{\sin x}{x} + C
		$$
	\end{solution}
\end{problem}

\begin{problem}
	课后习题 6.2.4

	\begin{solution}
		\begin{enumerate}
			\item[\textbf{2)}]
			$$
			\begin{aligned}
				\text{原式} & = \int \frac{\ab(\frac{t - b}{a})^2}{t} \dif \ab(\frac{t - b}{a}) \\
				& = \frac{1}{a} \int \ab(\frac{t}{a^2} - \frac{2b}{a^2} + \frac{b^2}{a^2 t}) \dif t \\
				& = \frac{1}{a^3} \ab(\frac{1}{2} t^2 - 2b t + b^2 \ln |t|) + C_1 \\
				& = \frac{1}{2 a^2} x^2 - \frac{b}{a^2} x + \frac{b^2}{a^3} \ln |a x + b| + C_2
			\end{aligned}
			$$
			\item[\textbf{4)}]
			$$
			\begin{aligned}
				\text{原式} & = \int \frac{\frac{t - b}{a}}{t^2} \dif \ab(\frac{t - b}{a}) \\
				& = \frac{1}{a^2} \int \frac{\dif t}{t} - \frac{b}{a^3} \int \frac{\dif t}{t^2} \\
				& = \frac{1}{a^2} \ln |t| + \frac{b}{a^2} \frac{1}{t} + C_1 \\
				& = \frac{1}{a^2} \ln |a x + b| + \frac{b}{a^2(a x + b)} + C_2
			\end{aligned}
			$$
			\item[\textbf{6)}]
			$$
			\begin{aligned}
				\text{原式} & = \int \frac{t^2 - b}{a} t \dif \ab(\frac{t^2 - b}{a}) \\
				& = \frac{2}{a^2} \int t^2 (t^2 - b) \dif t \\
				& = \frac{2}{5a^2} t^5 - \frac{2b}{3a^2} t^3 + C_1 \\
				& = \frac{2}{15 a^2} (a x + b)^{\frac{3}{2}} (3a x - 2b) + C_2
			\end{aligned}
			$$
			\item[\textbf{8)}] 记原积分式为 $I$,那么:
			$$
			\begin{aligned}
				I & = x (x^2 + a^2)^{\frac{3}{2}} - 3 \int x^2 (x^2 + a^2)^{\frac{1}{2}} \dif x \\
				& = x (x^2 + a^2)^{\frac{3}{2}} - 3 \int ((x^2 + a^2) - a^2) (x^2 + a^2)^{\frac{1}{2}} \dif x \\
				& = x (x^2 + a^2)^{\frac{3}{2}} - 3I + 3a^2 \int \sqrt{x^2 + a^2} \dif x \\
			\end{aligned}
			$$
			我们知道
			$$
			\int \sqrt{x^2 + a^2} \dif x = \frac{1}{2} x \sqrt{a^2 + x^2} + \frac{1}{2} a^2 \ln \ab|x + \sqrt{x^2 + a^2}| + C
			$$
			代入回原式得到:
			$$
			I = \frac{1}{4} x (x^2 + a^2)^{\frac{3}{2}} + \frac{3}{8} a^2 x \sqrt{a^2 + x^2} + \frac{3}{8} a^4 \ln \ab|x + \sqrt{x^2 + a^2}| + C
			$$
			\item[\textbf{10)}] 记原积分式为 $I$
			$$
			\begin{aligned}
				I & = \frac{1}{a} \int \sin b x \dif \ab(\E^{a x}) \\
				& = \frac{1}{a} \E^{a x} \sin b x - \frac{b}{a} \int \E^{a x} \cos b x \dif x \\
				& = \frac{1}{a} \E^{a x} \sin b x - \frac{b}{a^2} \ab(\E^{a x} \cos b x + b \int \E^{a x} \sin b x \dif x) \\
				& = \frac{1}{a} \E^{a x} \sin b x - \frac{b}{a^2} \E^{a x} \cos b x - \frac{b^2}{a^2} I \\
				\ab(1 + \frac{b^2}{a^2}) I & = \frac{\E^{a x}}{a^2} \ab(a \sin b x - b \cos b x) \\
				I & = \frac{\E^{a x}}{a^2 + b^2} \ab(a \sin b x - b \cos b x) \\
			\end{aligned}
			$$
			\item[\textbf{12)}]
			$$
			\begin{aligned}
				\text{原式} & = \int \E^t \dif \ab(\frac{t^2 - b}{a}) = \frac{2}{a} \int t \E^t \dif t \\
				& = \frac{2}{a} (t - 1) \E^t + C \\
				& = \frac{2}{a} \ab(\sqrt{a x + b} - 1) \E^{\sqrt{a x + b}} + C
			\end{aligned}
			$$
			\item[\textbf{14)}] 注意到
			$$
			\ab(\ln \ab(x + \sqrt{1 + x^2}))' = \frac{1 + \frac{x}{\sqrt{1 + x^2}}}{x + \sqrt{1 + x^2}} = \frac{1}{\sqrt{1 + x^2}}
			$$
			那么进行换元:
			$$
			\begin{aligned}
				\text{原式} & = \int \sqrt{\ln \ab(x + \sqrt{1 + x^2})} \dif \ab(\ln \ab(x + \sqrt{1 + x^2})) \\
				& = \frac{2}{3} \ab(\ln \ab(x + \sqrt{1 + x^2}))^{\frac{3}{2}} + C
			\end{aligned}
			$$
			\item[\textbf{16)}]
			$$
			\begin{aligned}
				\text{原式} & = \int \frac{\E^x}{\E^x \sqrt{1 + \E^x}} \dif x = \int \frac{\dif \ab(\E^x)}{\E^x (1 + \E^x)} \\
				& = \int \frac{\dif t}{t \sqrt{1 + t^2}} & \quad & (t \gets \E^x) \\
				& = \int \frac{\dif \ab(\tan^2 \theta)}{\tan^2 \theta \sec \theta} & \quad & (t \gets \tan^2 \theta) \\
				& = 2 \int \frac{\sec \theta}{\tan \theta} \dif \theta = 2 \int \frac{\dif \theta}{\sin \theta} \\
				& = -\ln \frac{1 + \cos \theta}{1 - \cos \theta} + C \\
				& = -\ln \frac{\sqrt{1 + \E^x} + 1}{\sqrt{1 + \E^x} - 1} + C
			\end{aligned}
			$$
			\item[\textbf{18)}]
			$$
			\begin{aligned}
				\text{原式} & = \int t \ln^2 t^2 \dif \ab(t^2) & \quad & (x \gets t^2) \\
				& = 8 \int t^2 \ln^2 t \dif t = \frac{8}{3} \int \ln^2 t \dif (t^3) \\
				& = \frac{8}{27} \int \ln^2 u \dif u & \quad & (u \gets t^3) \\
				& = \frac{8}{27} \ab(u \ln^2 u - \int u \dif \ab(\ln^2 u)) \\
				& = \frac{8}{27} \ab(u \ln^2 u - 2 \int \ln u \dif u) \\
				& = \frac{8}{27} \ab(u \ln^2 u - 2 u \ln u + 2u) + C \\
				& = \frac{2}{27} x^{\frac{3}{2}} \ab(9 \ln^2 x - 12 \ln x + 8) + C
			\end{aligned}
			$$
			\item[\textbf{20)}]
			$$
			\begin{aligned}
				\text{原式} & = \int t^2 \cos t \dif \ab(t^2) = 2 \int t^3 \cos t \dif t & \quad & (x \gets t^2) \\
				& = 2 \int t^3 \dif (\sin t) = 2t^3 \sin t - 6 \int t^2 \sin t \dif t \\
				& = 2t^3 \sin t + 6 \int t^2 \dif (\cos t) \\
				& = 2t^3 \sin t + 6t^2 \cos t - 12 \int t \cos t \dif t \\
				& = 2t^3 \sin t + 6t^2 \cos t - 12 \int t \dif (\sin t) \\
				& = 2t^3 \sin t + 6t^2 \cos t - 12t \sin t + 12 \int \sin t \dif t \\
				& = 2t^3 \sin t + 6t^2 \cos t - 12t \sin t - 12 \cos t + C \\
				& = 2x^{\frac{3}{2}} \sin \sqrt{x} + 6x \cos \sqrt{x} - 12 \sqrt{x} \sin \sqrt{x} - 12 \cos \sqrt{x} + C
			\end{aligned}
			$$
		\end{enumerate}
	\end{solution}
\end{problem}