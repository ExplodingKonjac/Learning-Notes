\section{函数极限 II}

\subsection{函数左右极限}

\begin{definition}
	若 $\forall\,\varepsilon > 0: \exists\,\delta > 0: \forall\,x \in (x_0, x_0 + \delta): |f(x) - A| < \varepsilon$,那么称 $A$ 为 $x$ 从右侧趋近于 $x_0$ 的极限,也称为 $f(x)$ 在 $x_0$ 的右极限,记作:
	$$
	\lim_{x \to x_0^+} f(x) = A \text{ 或 } f(x) \to A\ (x \to x_0^+) \text{ 或 } f(x_0^+) = A \text{ 或 } f(x_0+0) = a
	$$
	同理得到左极限的定义。记作:
	$$
	\lim_{x \to x_0^-} f(x) = A \text{ 或 } f(x) \to A\ (x \to x_0^-) \text{ 或 } f(x_0^-) = A \text{ 或 } f(x_0-0) = a
	$$
\end{definition}

结合左右极限的定义和 $\mathring{U}(x_0, \delta) = (x_0, x_0 + \delta) \cap (x_0 - \delta, x_0)$,得到定理:

\begin{theorem}
	$$
	\lim_{x \to x_0} f(x) = A \Leftrightarrow \lim_{x \to x_0^+} f(x) = \lim_{x \to x_0^-} f(x) = A
	$$
\end{theorem}

这个定理可用于证明某点的极限不存在。

\subsection{函数极限的性质}

函数极限具有和数列极限类似的一些性质:

\begin{theorem}[极限的唯一性]
	若函数极限存在,则极限唯一。	
\end{theorem}

\begin{theorem}[极限的局部有界性]
	若 $\lim\limits_{x \to x_0} f(x) = A$,则存在 $\delta > 0$ 使得 $f(x)$ 在 $\mathring{U}(x_0, \delta)$ 有界。	
\end{theorem}

\begin{theorem}[极限的局部保号性]
	若 $\lim\limits_{x \to x_0} f(x) = A > 0$,则存在 $\delta > 0$ 使得 $\forall\,x \in \mathring{U}(x_0, \delta): f(x) > 0$。大于号也可以同理换成小于号。
\end{theorem}

\begin{corollary}[极限的局部保序性]
	若 $\lim\limits_{x \to x_0} f(x) > \lim\limits_{x \to x_0} g(x)$,则存在 $\delta > 0$ 使得 $\forall\,x \in \mathring{U}(x_0, \delta): f(x) > g(x)$。
\end{corollary}

\begin{theorem}[极限的四则运算法则]
	若 $\lim\limits_{x \to x_0} f(x) = A,\ \lim\limits_{x \to x_0} g(x) = B$,则:

	\begin{itemize}
		\item $\lim\limits_{x \to x_0} f(x) + g(x) = A + B$;
		\item $\lim\limits_{x \to x_0} f(x) - g(x) = A - B$;
		\item $\lim\limits_{x \to x_0} f(x) \times g(x) = A \times B$;
		\item 若 $B \neq 0$ 且 $g(x)$ 在某个 $x_0$ 的去心邻域非零,$\lim\limits_{x \to x_0} \frac{f(x)}{g(x)} = \frac{A}{B}$。
	\end{itemize}
\end{theorem}

上述定理中 $x \to x_0$ 可替换为 $x \to x_0^+,\ x \to x_0^-,\ x \to +\infty,\ x \to -\infty$。

\begin{theorem}[夹逼定理]
	若 $\forall\,x \in \mathring{U}(x_0, \delta): g(x) \le f(x) \le h(x)$ 且 $\lim\limits_{x \to x_0} g(x) = \lim\limits_{x \to x_0} h(x) = A$,那么 $\lim\limits_{x \to x_0} f(x) = A$。

	同理也有 $x$ 趋向于正负无穷的版本。
\end{theorem}

\begin{theorem}[Heine 定理]
	$\lim\limits_{x \to x_0} f(x) = A$ 的充要条件是任意数列 $\{a_n\}$ 满足 $a_n \neq x_0 \land \lim\limits_{x \to \infty} a_n = x_0$,有 $\lim\limits_{n \to \infty} f(a_n) = A$。

	\begin{proof}
		\begin{itemize}
			\item 必要性:因为 $\lim\limits_{x \to x_0} f(x) = A$,那么
			$$
			\forall\,\varepsilon > 0: \exists\,\delta > 0: \forall\,x \in \mathring{U}(x_0, \delta): |f(x) - A| < \varepsilon
			$$
			又因为 $\lim\limits_{x \to \infty} a_n = x_0$,有
			$$
			\exists\,N \in \mathbb{Z}^+: \forall\,n \ge N: |a_n - x_0| < \delta \Rightarrow |f(a_n) - A|<\varepsilon
			$$
			于是得证。

			\item 充分性:考虑反证。若 $f(x)$ 在 $x_0$ 不存在极限或极限不为 $A$,那么:
			$$
			\exists\,\varepsilon_0 > 0: \forall\,\delta > 0: \exists\,x \in \mathring{U}(x_0, \delta): |f(x) - A| \ge \varepsilon_0
			$$
			令 $\delta_1 = 1, \delta_2 = \frac{1}{2}, \dots, \delta_n = \frac{1}{n}, \dots$,可以构造一个数列 $\{x_n\}$ 满足 $|x_n - x_0| < \frac{1}{n}$ 且 $|f(x_n) - A| \ge \varepsilon_0$。得到矛盾,故得证。
		\end{itemize}
	\end{proof}
\end{theorem}

如果只关注收敛性,也可以得到类似的定理:

\begin{theorem}
	$f(x)$ 在 $x_0$ 处极限存在当且仅当任意 $\{a_n\}$ 满足 $a_n \neq x \land \lim\limits_{n \to \infty} a_n = x_0$,都有 $\{f(a_n)\}$ 收敛。

	\begin{proof}
		只需要证明所有趋于 $x_0$ 的数列 $\{a_n\}$ 都有 $\lim\limits_{n \to \infty} f(a_n)$ 相等。

		对于两个趋于 $x_0$ 的数列 $\{x_n\},\{y_n\}$,构造新数列 $z_n$:
		$$
		z_n = \begin{cases}
			x_k & , n = 2k - 1\ (k \in \mathbb{Z}^+) \\
			y_k & , n = 2k\ (k \in \mathbb{Z}^+)
		\end{cases}
		$$
		那么 $\{f(z_n)\}$ 收敛,而 $\{f(x_n)\},\{f(y_n)\}$ 是它的子列,因此也具有相同的极限。
	\end{proof}
\end{theorem}

\begin{theorem}[函数极限的 Cauchy 收敛原理]
	设 $f(x)$ 在 $x_0$ 的某个去心邻域内有定义,那么 $\lim\limits_{x \to x_0} f(x)$ 存在的充要条件是 $\forall\,\varepsilon > 0: \exists\,\delta > 0: \forall\,0 < |x_0 - x_1|,|x_0 - x_2| < \delta: |f(x_1) - f(x_2)| < \varepsilon$。

	\begin{proof}
		\begin{enumerate}
			\item 充分性:绝对值三角不等式随便放缩即可;
			\item 必要性:取任意收敛于 $x_0$ 的数列 $\{a_n\}$,那么可得 $\{f(a_n)\}$ 是基本列,再根据 Heine 定理即证。
		\end{enumerate}
	\end{proof}
\end{theorem}

函数极限相比数列极限不同的是其还可以进行复合运算。

\begin{theorem}[复合函数的极限]
	设 $\lim\limits_{u \to u_0} f(u) = A,\ \lim\limits_{x \to x_0} g(x) = u_0$,且在 $x_0$ 的某个去心邻域中 $g(x) \neq u_0$,那么:
	$$
	\lim_{x \to x_0} f(g(x)) = \lim_{u \to u_0} f(u) = A
	$$
\end{theorem}

定理中必须有 $g(x) \neq u_0$。

\subsection{作业}

\begin{problem}
	课后习题 3.1.1
	
	\begin{proof}
		\begin{enumerate}
			\item[\textbf{1)}] 对于 $\forall\,\varepsilon > 0$,取 $\delta = \min\{1, \frac{\varepsilon}{3}\}$,那么:
			$$
			\forall\,x \in \mathring{U}(2, \delta): x^2 - 6x + 10 - 2 = (x - 2)(x - 4) < 3 \delta < \varepsilon 
			$$
			即证。

			\item[\textbf{2)}] 对于 $\forall\,\varepsilon > 0$,设 $\varepsilon' = \min\{\varepsilon, \frac{1}{3}\}$,取 $\delta = \frac{4\varepsilon'}{1 + 2 \varepsilon'}$,那么:
			$$
			\forall\,x \in \mathring{U}(1, \delta): \frac{x - 1}{x^2 - 1} = \frac{1}{x + 1} \in \ab(\frac{1}{2 + \frac{4\varepsilon'}{1 - 2\varepsilon'}}, \frac{1}{2 - \frac{4\varepsilon'}{1 + 2\varepsilon'}}) = \ab(\frac{1 - 2\varepsilon'}{2}, \frac{1 + 2\varepsilon'}{2})
			$$
			而 $\varepsilon' \le \varepsilon$,即证。
		\end{enumerate}
	\end{proof}
\end{problem}

\begin{problem}
	课后习题 3.1.3

	\begin{proof}
		由题知:
		$$
		\forall\,\varepsilon > 0: \exists\,\delta > 0: \forall\,x \in \mathring{U}(x_0, \delta): |f(x) - A| < \varepsilon
		$$
		那么可以得到:
		$$
		\begin{gathered}
			A - \varepsilon < f(x) < A + \varepsilon \\
			\min\{|A - \varepsilon|, |A + \varepsilon|\} < |f(x)| < \max\{|A - \varepsilon|, |A + \varepsilon|\} \\
			|A| - \varepsilon < |f(x)| < |A| + \varepsilon
		\end{gathered}
		$$
		也就是说 $\ab||f(x)|-|A|| < \varepsilon$,因此得证。
	\end{proof}
\end{problem}

\begin{problem}
	课后习题 3.1.4

	\begin{proof}
		$$
		|f(x) - 0| < \varepsilon \Leftrightarrow |f(x)| < \varepsilon \Leftrightarrow \ab||f(x)| - 0| < \varepsilon
		$$
	\end{proof}
\end{problem}

\begin{problem}
	课后习题 3.1.6

	\begin{solution}
		\begin{enumerate}
			\item[\textbf{1)}] 存在一个 $\delta > 0$,使得
			$$
			\forall\,x \in (0, \delta): 1 - x = x \ab(\frac{1}{x} - 1) \le x \floor*{\frac{1}{x}} \le x \cdot \frac{1}{x} = 1
			$$
			而 $\lim\limits_{x \to 0^+} (1 - x) = \lim\limits_{x \to 0^+} 1 = 1$,因此 $\lim\limits_{x \to 0^+} x \floor*{\frac{1}{x}} = 1$。

			\item[\textbf{2)}] 当 $x \in (0,1)$ 时:
			$$
			\begin{gathered}
				\ab(\frac{1}{x})^x > 1 \\
				\ab(\frac{1}{x})^x = \E^{x \ln \frac{1}{x}} < \E^{x \sqrt{\frac{2}{x}}} = \E^{\sqrt{2x}}
			\end{gathered}
			$$
			而 $\lim\limits_{x \to 0^+} \E^{\sqrt{2x}} = 1$,因此 $\lim\limits_{x \to 0^+} \ab(\frac{1}{x})^x = 1$。
		\end{enumerate}
	\end{solution}
\end{problem}

\begin{problem}
	课后习题 3.1.7
	
	\begin{solution}
		\begin{enumerate}
			\item[\textbf{2)}]
			$$
			\begin{aligned}
				\text{原式} & = \lim_{x \to 0} \frac{\sum_{i=1}^10 \binom{10}{i} x^i}{x} \\
				& = \lim_{x \to 0} \sum_{i=1}^10 \binom{10}{i} x^{i-1} \\
				& = \binom{10}{1} = 10
			\end{aligned}
			$$

			\item[\textbf{5)}]
			$$
			\begin{aligned}
				\text{原式} & = \lim_{x \to 1} \sum_{i=1}^k i x^{k-i} = k
			\end{aligned}
			$$

			\item[\textbf{7)}] 令 $t = x - 1$,那么:
			$$
			\begin{aligned}
				\text{原式} & = \lim_{t \to 0} \ab(\frac{k}{t \sum_{i=0}^{k-1} x^i} - \frac{l}{t \sum_{i=0}^{l-1} x^i}) \\
				& = \lim_{t \to 0} \frac{k \sum_{i=0}^{l-1} (t+1)^i - l \sum_{i=0}^{k-1} (t+1)^i}{t \sum_{i=0}^{k-1} x^i \sum_{i=0}^{l-1} x^i} \\
				& = \lim_{t \to 0} \frac{k \frac{l(l-1)}{2} t - l \frac{t(t-1)}{2} + O(t^2)}{t} \lim_{x \to 1} \frac{1}{\sum_{i=0}^{k-1} x^i \sum_{i=0}^{l-1} x^i} \\
				& = \frac{k l (l - k)}{2} - \frac{1}{k l} \\
				& = \frac{l - k}{2}
			\end{aligned}
			$$
		\end{enumerate}
	\end{solution}
\end{problem}

\begin{problem}
	课后习题 3.1.8

	\begin{proof}
		由题知:
		$$
		\forall\,0 < \varepsilon < A: \exists\,\delta > 0: \forall\,x \in \mathring{U}(x_0, \delta): |f(x) - A| < \varepsilon
		$$
		那么可以得到
		$$
		\ab|\frac{f(x)}{A} - 1| = \ab|\sqrt[n]{\frac{f(x)}{A}} - 1| \sum_{i=0}^{n-1} \ab(\frac{f(x)}{A})^{\frac{i}{n}} < \frac{\varepsilon}{A}
		$$
		又因为 $\frac{f(x)}{A} < 1 + \frac{\varepsilon}{A} < 2$,有:
		$$
		\ab|\sqrt[n]{\frac{f(x)}{A}} - 1| < \frac{\varepsilon}{2^n A} \Rightarrow \ab|\sqrt[n]{f(x)} - \sqrt[n]{A}| < \frac{\varepsilon}{2^n A^{\frac{n-1}{n}}}
		$$
		于是能够得证。
	\end{proof}
\end{problem}

\begin{problem}
	课后习题 3.1.11
	
	\begin{solution}
		\begin{enumerate}
			\item[\textbf{1)}]
			$$
			\begin{aligned}
				\text{原式} & = \lim_{x \to 0} \frac{\frac{\sin 3x}{x} - \frac{\sin x}{x}}{\frac{\sin 2x}{x}} \\
				& = \frac{\lim\limits_{x \to 0} \frac{\sin 3x}{x} - \lim\limits_{x \to 0} \frac{\sin x}{x}}{\lim\limits_{x \to 0} \frac{\sin 2x}{x}} \\
				& = \frac{3 - 1}{2} = 1
			\end{aligned}
			$$

			\item[\textbf{3)}]
			$$
			\begin{aligned}
				\text{原式} & = \lim_{x \to 0} \frac{1 - \cos(2x - x) \cos(2x + x)}{x^2} \\
				& = \lim_{x \to 0} \frac{1 - \cos^2 2x \cos^2 x + \sin^2 2x \sin^2 x}{x^2} \\
				& = \lim_{x \to 0} \frac{1 - (1 - \sin^2 2x)(1 - \sin^2 x) + \sin^2 2x + \sin^2 x}{x^2} \\
				& = \lim_{x \to 0} \frac{\sin^2 2x + \sin^2 x}{x^2} \\
				& = \ab(\lim_{x \to 0} \frac{\sin 2x}{x})^2 + \ab(\lim_{x \to 0} \frac{\sin x}{x})^2 = 5
			\end{aligned}
			$$

			\item[\textbf{6)}]
			$$
			\begin{aligned}
				\text{原式} & = \lim_{x \to 0} \frac{\tan(\tan x)}{\tan x} \cdot \frac{\tan x}{x} \\
				& = \ab(\lim_{x \to 0} \frac{\tan(\tan x)}{\tan x}) \cdot \ab(\lim_{x \to 0} \frac{\tan x}{x}) \\
				& = 1
			\end{aligned}
			$$

			\item[\textbf{7)}]
			$$
			\begin{aligned}
				\text{原式} & = \lim_{x \to 1} \frac{x - 1}{\tan\ab(\frac{\pi}{2} - \frac{\pi x}{2})} \\
				& = \lim_{t \to 0} \frac{-t}{\tan \frac{\pi t}{2}} \\
				& = -\frac{2}{\pi} \ab(\lim_{t \to 0} \frac{\tan \frac{\pi t}{2}}{\frac{\pi t}{2}})^{-1} \\
				& = -\frac{2}{\pi}
			\end{aligned}
			$$
		\end{enumerate}
	\end{solution}
\end{problem}

\begin{problem}
	课后习题 3.1.12

	\begin{solution}
		对于第一个极限:
		$$
		\begin{aligned}
			\text{原式} & = \lim_{n \to \infty} \sin\ab(\pi \sqrt{n^2 + \sqrt{n}} - n \pi) \\
			& = \lim_{n \to \infty} \sin\ab(\frac{\sqrt{n} \pi}{\sqrt{n^2 + \sqrt{n}} + n}) \\
			& = \lim_{n \to \infty} \frac{\sin\ab(\frac{\sqrt{n} \pi}{\sqrt{n^2 + \sqrt{n}} + n})}{\frac{\sqrt{n} \pi}{\sqrt{n^2 + \sqrt{n}} + n}} \cdot \lim_{n \to \infty} \frac{\sqrt{n} \pi}{\sqrt{n^2 + \sqrt{n}} + n} \\
			& = 0 \cdot 1 = 0
		\end{aligned}
		$$
		对于另一个极限:
		$$
		\begin{aligned}
			\text{原式} & = \lim_{n \to \infty} 2n \sin\ab(\pi \sqrt{4n^2 + 1} - 2n \pi) \\
			& = \lim_{n \to \infty} 2n \sin\ab(\frac{\pi}{\sqrt{4n^2 + 1} + 2n}) \\
			& = \lim_{n \to \infty} \frac{\sin\ab(\frac{\pi}{\sqrt{4n^2 + 1} + 2n})}{\frac{\pi}{\sqrt{4n^2 + 1} + 2n}} \cdot \lim_{n \to \infty} \frac{2n \pi}{\sqrt{4n^2 + 1} + 2n} \\
			& = \frac{\pi}{2}
		\end{aligned}
		$$
	\end{solution}
\end{problem}

\begin{problem}
	课后习题 3.1.13

	\begin{solution}
		记 $f(x,n) = \prod_{k=0}^n \cos \frac{x}{2^k}$,那么:
		$$
		\begin{aligned}
			f(x,n) \cdot \sin \frac{x}{2^n} & = \frac{1}{2} \cos x \cos \frac{x}{2} \cdots \cos \frac{x}{2^{n-1}} \sin \frac{x}{2^{n-1}} \\
			& = \frac{1}{4} \cos x \cos \frac{x}{2} \cdots \cos \frac{x}{2^{n-2}} \sin \frac{x}{2^{n-2}} \\
			& \vdots \\
			& = \frac{1}{2^{n+1}} \sin x \\
			f(x,n) & = \frac{\sin x}{2^{n+1} \sin \frac{x}{2^n}}
		\end{aligned}
		$$
		先对 $n$ 取极限:
		$$
		\begin{aligned}
			\lim_{n \to \infty} f(x,n) & = \lim_{n \to \infty} \frac{\sin x}{2x \cdot \frac{\sin (x / 2^n)}{x / 2^n}} \\
			& = \frac{\sin x}{2x} \ab(\lim_{n \to \infty} \frac{\sin (x / 2^n)}{x / 2^n})^{-1} \\
			& = \frac{\sin x}{2x}
		\end{aligned}
		$$
		再对 $x$ 取极限即可得到答案为 $\frac{1}{2}$。
	\end{solution}
\end{problem}