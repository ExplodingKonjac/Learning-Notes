\section{函数的导数 I}

\subsection{导数的定义}

\begin{definition}[函数的导数]
	设 $f(x)$ 在 $x_0$ 的某个邻域有定义,且极限
	$$
	A = \lim_{\Delta x \to 0} \frac{f(x_0 + \Delta x) - f(x_0)}{\Delta x}
	$$
	存在,那么称 $A$ 为 $f(x)$ 在 $x_0$ 处的\textbf{导数},记作
	$$
	f'(x_0) = A\ \text{或}\ \left.\frac{\dd f}{\dd x} \right|_{x = x_0} = A
	$$
	这个定义也可以等价于
	$$
	f'(x_0) = \lim_{x \to x_0} \frac{f(x) - f(x_0)}{x - x_0}
	$$
	若 $f(x)$ 在 $x_0$ 导数存在,则称 $f(x)$ \textbf{在 $x_0$ 可导}。
\end{definition}

导数的几何意义是函数某点的切线斜率,物理意义是物理量的瞬时变化率。

类似于极限有左右极限,也可以定义单侧导数:

\begin{definition}[单侧导数]
	若 $f(x)$ 在 $[x_0, x_0 + \delta)$ 有定义,那么定义 $x_0$ 处的\textbf{左导数}:
	$$
	f'_{-}(x_0) = \lim_{x \to x_0^-} \frac{f(x) - f(x_0)}{x - x_0}
	$$
	若 $f(x)$ 在 $(x_0 - \delta, x_0]$ 有定义,那么定义 $x_0$ 处的\textbf{右导数}:
	$$
	f'_{+}(x_0) = \lim_{x \to x_0^+} \frac{f(x) - f(x_0)}{x - x_0}
	$$
\end{definition}

那么可以自然地得到出下面定理:

\begin{theorem}
	$f(x)$ 在 $x_0$ 可导当且仅当 $f(x)$ 在 $x_0$ 的左右导数都存在且相等。
\end{theorem}

类似于连续性,我们可以定义区间上的可导性:

\begin{definition}
	若 $f(x)$ 在 $\forall\,x_0 \in (a, b)$ 处均可导,则称 $f(x)$ 在 $(a, b)$ 可导。若 $f(x)$ 在 $(a, b)$ 可导且 $f'_{+}(a), f'_{-}(b)$ 均存在,则称 $f(x)$ 在 $[a, b]$ 内可导。同理可以得到 $f(x)$ 在 $[a, b), (a, b]$ 内可导的定义。
\end{definition}

\begin{theorem}
	若 $f(x)$ 在 $x_0$ 处可导,那么 $f(x)$ 在 $x_0$ 处连续。

	\begin{proof}
		$$
		\lim_{x \to x_0} (f(x) - f(x_0)) = \lim_{x \to x_0} \frac{f(x) - f(x_0)}{x - x_0} \lim_{x \to x_0} (x - x_0) = f'(x_0) \cdot 0 = 0
		$$
	\end{proof}
\end{theorem}

即,对于一元函数而言,\textbf{可导必定连续,连续不一定可导}。

\begin{figure}[htbp]
	\centering
	\includegraphics[scale=0.6]{工科数学分析/graphics/1.png}
\end{figure}

\subsection{基本初等函数的导数}

下面给出基本初等函数的导数及推导:

\begin{itemize}
	\item 常函数 $f(x) = C$:
	$$
	f'(x) = \lim_{x \to x_0} \frac{C - C}{x - x_0} = 0
	$$

	\item 幂函数 $f(x) = x^{\mu}$:
	$$
	\begin{aligned}
		f'(x) & = \lim_{\Delta x \to 0} \frac{(x + \Delta x)^\mu - x^\mu}{\Delta x} \\
		& = x^\mu \lim_{\Delta x \to 0} \cdot \frac{\ab(1 + \frac{\Delta x}{x})^\mu - 1}{\Delta x} \\
		& = x^\mu \lim_{\Delta x \to 0} \cdot \frac{\frac{\mu \Delta x}{x}}{\Delta x} \\
		& = \mu x^{\mu - 1}
	\end{aligned}
	$$

	\item 指数函数 $f(x) = \E^x$:
	$$
	\begin{aligned}
		f'(x) & = \lim_{\Delta x \to 0} \frac{\E^{x + \Delta x} - \E^x}{\Delta x} \\
		& = \E^x \lim_{\Delta x \to 0} \cdot \frac{\E^{\Delta x} - 1}{\Delta x} \\
		& = \E^x
	\end{aligned}
	$$

	\item 对数函数 $f(x) = \ln x$:
	$$
	\begin{aligned}
		f'(x) & = \lim_{\Delta x \to 0} \frac{\ln(x + \Delta x) - \ln x}{\Delta x} \\
		& = \lim_{\Delta x \to 0} \frac{\ln\ab(1 + \frac{\Delta x}{x})}{\Delta x} \\
		& = \lim_{\Delta x \to 0} \frac{\frac{\Delta x}{x}}{\Delta x} = \frac{1}{x}
	\end{aligned}
	$$

	\item 三角函数 $f(x) = \sin x$:
	$$
	\begin{aligned}
		f'(x) & = \lim_{\Delta x \to 0} \frac{\sin(x + \Delta x) - \sin x}{\Delta x} \\
		& = \lim_{\Delta x \to 0} \frac{2 \cos \frac{2x + \Delta x}{2} \sin \frac{\Delta x}{2}}{\Delta x} \\
		& = \lim_{\Delta x \to 0} \frac{2 \cos \ab(x + \frac{\Delta x}{2}) \cdot \frac{\Delta x}{2}}{\Delta x} \\
		& = \cos x
	\end{aligned}
	$$
\end{itemize}

\subsection{导数的运算法则}

首先为方便推导,我们给出引理:

\begin{lemma}
	若 $f'(x_0)$ 存在,那么当 $\Delta x \to 0$ 时:
	$$
	f(x_0 + \Delta x) - f(x_0) \sim f'(x_0 + \Delta x) \Delta x
	$$
	这是根据导数的定义直接得到的。我们将这个无穷小量简记为 $\Delta f$。
\end{lemma}

\begin{theorem}[导数的加减法则]
	$$
	(f(x) \pm g(x))' = f'(x) \pm g'(x)
	$$
	\begin{proof}
		$$
		\begin{aligned}
			(f(x) \pm g(x))' & = \lim_{\Delta x \to 0} \frac{(f(x + \Delta x) \pm g(x + \Delta x)) - (f(x) \pm g(x))}{\Delta x} \\
			& = \lim_{\Delta x \to 0} \frac{\Delta f \pm \Delta g}{\Delta x} \\
			& = \lim_{\Delta x \to 0} \frac{\Delta f}{\Delta x} \pm \lim_{\Delta x \to 0} \frac{\Delta g}{\Delta x} \\
			& = f'(x) \pm g'(x)
		\end{aligned}
		$$
	\end{proof}
\end{theorem}

\begin{theorem}[导数的乘法法则]
	$$
	(f(x) \cdot g(x))' = f'(x) g(x) + f(x) g'(x)
	$$
	\begin{proof}
		$$
		\begin{aligned}
			(f(x) \cdot g(x))' & = \lim_{\Delta x \to 0} \frac{f(x + \Delta x) \cdot g(x + \Delta x) - f(x) \cdot g(x)}{\Delta x} \\
			& = \lim_{\Delta x \to 0} \frac{(f(x) + \Delta f)(g(x) + \Delta g) - f(x) g(x)}{\Delta x} \\
			& = \lim_{\Delta x \to 0} \frac{g(x) \Delta f + f(x) \Delta g - \Delta f \Delta g)}{\Delta x} \\
			& = f'(x) g(x) + f(x) g'(x)
		\end{aligned}
		$$
	\end{proof}
\end{theorem}

\begin{theorem}[导数的除法法则]
	$$
	\ab(\frac{f(x)}{g(x)})' = \frac{f'(x) g(x) - f(x) g'(x)}{g(x)^2}
	$$
	\begin{proof}
		$$
		\begin{aligned}
			\ab(\frac{f(x)}{g(x)})' & = \lim_{\Delta x \to 0} \frac{\frac{f(x) + \Delta f}{g(x) + \Delta g} - \frac{f(x)}{g(x)}}{\Delta x} \\
			& = \frac{g(x) \ab(f(x) + \Delta f) - f(x) \ab(g(x) + \Delta g)}{g(x) \ab(g(x) + \Delta g)} \cdot \frac{1}{\Delta x} \\
			& = \lim_{\Delta x \to 0} \frac{g(x) \Delta f - f(x) \Delta g}{\Delta x} \lim_{\Delta x \to 0} \frac{1}{g(x) (g(x) + \Delta x)} \\
			& = \frac{f'(x) g(x) - f(x) g'(x)}{g(x)^2}
		\end{aligned}
		$$
	\end{proof}
\end{theorem}

\begin{theorem}[复合函数的链式求导法则]
	设 $g(x)$ 在 $x_0$ 处可导,$f(x)$ 在 $g(x_0)$ 处可导,那么:
	$$
	\left.(f(g(x)))'\right|_{x = x_0} = f'(g(x_0)) g'(x_0)
	$$
	或者使用莱布尼茨记号:
	$$
	\frac{\dd f}{\dd x} = \frac{\dd f}{\dd g} \cdot \frac{\dd g}{\dd x}
	$$

	\begin{proof}
		$$
		\begin{aligned}
			\left.(f(g(x)))'\right|_{x = x_0} & = \lim_{\Delta x \to 0} \frac{f(g(x_0 + \Delta x)) - f(g(x_0))}{\Delta x} \\
			& = \lim_{\Delta x \to 0} \frac{f(g(x_0) + \Delta g) - f(g(x_0))}{\Delta g} \cdot \frac{\Delta g}{\Delta x} \\
			& = \lim_{\Delta x \to 0} \frac{f(g(x_0) + \Delta g) - f(g(x_0))}{\Delta g} \cdot \lim_{\Delta x \to 0} \frac{\Delta g}{\Delta x} \\
			& = f'(g(x_0)) g'(x)
		\end{aligned}
		$$
	\end{proof}
\end{theorem}

\begin{corollary}
	对于有限个函数 $f_1(x), f_2(x), \dots, f_m(x)$,设 $g(x) = (f_1 \circ f_2 \circ \cdots \circ f_m)(x)$,那么有:
	$$
	\frac{\dd g}{\dd x} = \frac{\dd f_1}{\dd f_2} \cdot \frac{\dd f_2}{\dd f_3} \cdot \cdots \cdot \frac{\dd f_{m-1}}{\dd f_m} \cdot \frac{\dd f_m}{\dd x}
	$$
\end{corollary}

\subsection{作业}

\begin{problem}
	课后习题 4.1.2

	\begin{solution}
		设切点横坐标为 $x_0$,那么:
		$$
		\begin{cases}
			a x_0^3 = \ln x_0 \\
			3a x_0^2 = \frac{1}{x_0}
		\end{cases}
		$$
		容易发现该方程的解满足 $a \neq 0,\ x_0 \neq 1$。两式相除得到:
		$$
		\frac{x_0}{3} = x_0 \ln x_0
		$$
		解得 $x_0 = \E^{\frac{1}{3}}$,回带得到 $a = \frac{1}{3\E}$。
	\end{solution}
\end{problem}

\begin{problem}
	课后习题 4.1.5

	\begin{solution}
		由题得 $x' = \frac{y}{2}$。

		\begin{enumerate}
			\item[\textbf{1)}]
			$$
			x' = 1 \Rightarrow y = 2,\ x = 1
			$$

			\item[\textbf{2)}]
			$$
			x' = -\frac{1}{4} \Rightarrow y = -\frac{1}{2},\ x = \frac{1}{16}
			$$
		\end{enumerate}
	\end{solution}
\end{problem}

\begin{problem}
	课后习题 4.1.7

	\begin{solution}
		由题可知 $f(0) = 0$。那么:
		$$
		\lim_{x \to 0} \ab|\frac{f(x)}{x}| \le \lim_{x \to 0} \ab|\frac{g(x)}{x}| = \lim_{x \to 0} \ab|\frac{g(x) - g(0)}{x - 0}| = |g'(0)| = 0
		$$
		因此 $\lim\limits_{x \to 0} \ab|\frac{f(x)}{x}| = 0$,这也说明 $\lim\limits_{x \to 0} \frac{f(x)}{x} = 0$,于是
		$$
		f'(0) = \lim_{x \to 0} \frac{f(x) - f(0)}{x - 0} = \lim_{x \to 0} \frac{f(x)}{x} = 0
		$$
	\end{solution}
\end{problem}

\begin{problem}
	课后习题 4.1.8

	\begin{solution}
		首先得到
		$$
		\lim_{x \to 0} \ab(f(1 + \sin x) - 3f(1 - \sin x)) = \lim_{x \to 0} (8x + \alpha(x)) = 0 \Rightarrow f(1) = 0
		$$
		那么令 $h \gets \sin x\ \ab(-\frac{\pi}{2} < x < \frac{\pi}{2})$ 得到:
		$$
		\begin{aligned}
			& & f(1 + h) - 3 f(1 - h) & = 8 \arcsin h + \alpha(\arcsin h) \\
			\Rightarrow & & \frac{f(1 + h) - f(1)}{h} + 3 \frac{f(1) - f(1 - h)}{h} & = \frac{8 \arcsin h + \alpha(\arcsin h)}{h} \\
			\Rightarrow & & \lim_{h \to 0} \frac{f(1 + h) - f(1)}{h} + 3 \lim_{h \to 0} \frac{f(1) - f(1 - h)}{h} & = \lim_{h \to 0} \frac{8 \arcsin h + \alpha(\arcsin h)}{h} \\
			\Rightarrow & & f'(1) + 3 f'(1) & = 8 \\
			\Rightarrow & & f'(1) & = 2
		\end{aligned}
		$$
		因此所求切线方程为 $y = 2x - 2$。
	\end{solution}
\end{problem}

\begin{problem}
	课后习题 4.1.9

	\begin{solution}
		$$
		\begin{aligned}
			\text{原式} & = \exp \ab(\lim_{n \to \infty} n \ln \ab(\frac{f(\frac{1}{n})}{f(0)})) \\
			& = \exp \ab(\lim_{n \to \infty} n \ln \ab(1 + \frac{f(\frac{1}{n}) - f(0)}{f(0)})) \\
			& = \exp \ab(\lim_{n \to \infty} n \cdot \frac{f(\frac{1}{n}) - f(0)}{f(0)}) \\
			& = \exp \ab(\lim_{n \to \infty} n \cdot \frac{f'(0)}{n}) \\
			& = a
		\end{aligned}
		$$
	\end{solution}
\end{problem}

\begin{problem}
	课后习题 4.1.10

	\begin{solution}
		因为 $f(0) = 0$,因此有 $f(x) = f'(0) x + o(x^2)$。于是
		$$
		\begin{aligned}
			x_n & = \sum_{i=1}^n \ab(\frac{i}{n^2} f'(0) + o\ab(\frac{i^2}{n^4})) \\
			& = \sum_{i=1}^n \ab(\frac{i}{n^2} f'(0) + i^2 o\ab(\frac{1}{n^4})) \\
			& = \frac{n+1}{2n} f'(0) + O(n^3) o\ab(\frac{1}{n^4}) \\
			& = \frac{n+1}{2n} f'(0) + o\ab(\frac{1}{n})
		\end{aligned}
		$$
		因此 $\lim\limits_{n \to \infty} x_n = \frac{1}{2} f'(0)$。

		\begin{enumerate}
			\item[\textbf{1)}] $\text{原式} = \frac{1}{2}$。
			\item[\textbf{2)}] $\text{原式} = \sqrt{\E}$。
		\end{enumerate}
	\end{solution}
\end{problem}

\begin{problem}
	课后习题 4.1.14

	\begin{proof}
		\begin{itemize}
			\item 必要性:可以构造
			$$
			g(x) = \begin{cases}
				\frac{f(x) - f(x_0)}{x - x_0} & , x \neq x_0 \\
				f'(x_0) & , x = x_0
			\end{cases}
			$$

			\item 充分性:若存在 $g(x)$,那么:
			$$
			g(x) = \begin{cases}
				\frac{f(x) - f(x_0)}{x - x_0} & , x \neq x_0 \\
				A & , x = x_0
			\end{cases}
			$$
			根据 $g(x)$ 的连续性,有 $\lim\limits_{x \to x_0} g(x) = A$,即 $f'(x_0) = g(x_0)$。
		\end{itemize}
	\end{proof}
\end{problem}

\begin{problem}
	课后习题 4.1.15

	\begin{proof} 不妨设 $f'_+(a) > 0,\ f'_-(a) > 0$。根据极限的保序性,存在 $\delta > 0$ 使得:
		$$
		\begin{gathered}
			\forall\,x \in (a, a + \delta): \frac{f(x) - f(a)}{x - a} = \frac{f(x)}{x - a} > 0 \Rightarrow f(x) > 0 \\
			\forall\,x \in (b - \delta, b): \frac{f(x) - f(b)}{x - b} = \frac{f(x)}{x - b} < 0 \Rightarrow f(x) < 0 \\
		\end{gathered}
		$$
		根据零点存在性定理得证。
	\end{proof}
\end{problem}

\begin{problem}
	课后习题 4.1.16

	\begin{solution}
		$$
		f(x) = \begin{cases}
			n^2 & , x \in \mathbb{Q} \\
			0 & , x \notin \mathbb{Q}
		\end{cases}
		$$
		则 $f(x)$ 在 $x = 0$ 处可导但在 $0$ 的任何邻域内都不连续。
	\end{solution}
\end{problem}