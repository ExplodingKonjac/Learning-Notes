\section{数列极限 III \& 函数极限 I}

\subsection{重要极限再放送}

我们有:
$$
\lim_{n \to \infty} \ab(1 + \frac{1}{n})^{n + 1} = \lim_{n \to \infty} \ab(1 + \frac{1}{n})^n \cdot \lim_{n \to \infty} \ab(1 + \frac{1}{n}) = \E
$$
而 $\ab(1 + \frac{1}{n})^{n + 1}$ 单调递减。根据极限的保序性有:
$$
\ab(1 + \frac{1}{n})^n < \E < \ab(1 + \frac{1}{n})^{n + 1}
$$
两边取对数并且化简就能得到:
$$
\frac{1}{n + 1} < \ln\ab(1 + \frac{1}{n}) < \frac{1}{n}
$$
这是个可能比较有用的不等式。

\begin{example}
	设 $a_n = \sum_{i=1}^n \frac{1}{n} - \ln n$,证明 $\{a_n\}$ 收敛。

	\begin{proof}
		首先有:
		$$
		a_{n+1} - a_n = \frac{1}{n + 1} - \ln\ab(\frac{n + 1}{n}) = \frac{1}{n + 1} - \ln\ab(1 + \frac{1}{n}) < 0
		$$
		说明 $\{a_n\}$ 单调递减。只需要再证明有界即可。
		$$
		\begin{aligned}
			a_n & > \sum_{i=1}^n \frac{1}{n} - \ln(n+1) \\
			& = \sum_{i=1}^n - \sum_{i=1}^n \ln\ab(\frac{n + 1}{n}) \\
			& = \sum_{i=1}^n \ab(\frac{1}{n} - \ln\ab(1 + \frac{1}{n})) > 0
		\end{aligned}
		$$
		因此 $\{a_n\}$ 单调有界,极限存在。
	\end{proof}
\end{example}

事实上,$\lim\limits_{n \to \infty} \ab(\sum_{i=1}^n \frac{1}{n} - \ln n) = \gamma$,其中 $\gamma \approx 0.57721$ 称为欧拉常数。

\subsection{实数连续性基本定理}

\begin{theorem}[闭区间套定理]
	若 $I_n = [a_n, b_n]$ 满足:

	\begin{itemize}
		\item $I_{n+1} \subset I_n$;
		\item $\lim\limits_{n \to \infty} (b_n - a_n) = 0$。
	\end{itemize}

	那么存在唯一的 $\xi$ 使得 $\xi \in \bigcap_{i=1}^{+\infty} I_i$。

	\begin{proof}
		因为 $a_1 \le a_2 \le \cdots \le a_n \le b_n \le \cdots \le b_2 \le b_1$,因此 $\{a_n\},\{b_n\}$ 均收敛。设 $\lim\limits_{n \to \infty} a_n = a,\ \lim\limits_{n \to \infty} b_n = b$,那么:
		$$
		\lim_{n \to \infty} (b_n - a_n) = \lim_{n \to \infty} b_n - \lim_{n \to \infty} a_n = b - a = 0
		$$
		于是 $a = b$。又因为 $a_n \le \xi \le b_n$,而 $a,b$ 分别是 $\{a_n\},\{b_n\}$ 的上/下确界,因此有 $a \le \xi \le b$,有唯一的 $\xi = a = b$。
	\end{proof}
\end{theorem}

\begin{theorem}[列紧性定理]
	任意有界数列 $\{x_n\}$ 必有收敛子列。

	\begin{proof}
		使用二分法。设数列的上确界为 $R$,下确界为 $L$。那么构造 $I_n = [l_n, r_n]$ 和 $i_n$:
		
		\begin{itemize}
			\item $I_1 = [L, R], i_1 = 1$;
			\item 在 $\ab[l_n, \frac{l_n + r_n}{2}], \ab[\frac{l_n + r_n}{2}, r_n]$ 中选择包含无穷多项 $x_m (m > i_n)$ 的那个,记作 $[l', r']$,那么 $l_{n+1}, r_{n+1}$ 为 $\{a_m: m > i_n \land l_{n+1} \le a_m \le r_{n+1}\}$ 的下/上确界;
			\item $i_n = \min\{i: i > i_{n-1} \land a_i \in I_n\}$。
		\end{itemize}

		归纳可知这个构造是合法的。因为 $I_n$ 构成闭区间套,因此 $\{a_{i_n}\}$ 就是一个合法的收敛子列。
	\end{proof}
\end{theorem}

\begin{definition}[柯西基本列]
	给定数列 $\{a_n\}$,若 $\forall\,\varepsilon > 0: \exists\,N \in \mathbb{Z}^+: \forall\, N \le n < m: |a_n - a_m| < \varepsilon$,那么称 $\{a_n\}$ 为基本列(Cauchy 列)。
\end{definition}

\begin{example}
	证明:设 $a_n = \sum_{i=1}^n \frac{1}{i^\alpha}\ (0 < \alpha \le 1)$,那么 $\{a_n\}$ 不是 Cauchy 列。

	\begin{proof}
		取 $\varepsilon = \frac{1}{2}$。那么对于任意 $N$,取 $n = N, m = 2N + 1$,那么:
		$$
		a_m - a_n = \sum_{i=N+1}^{2N+1} \frac{1}{i^\alpha} \ge \sum_{i=N+1}^{2N+1} \frac{1}{i} \ge \frac{N + 1}{2N + 1} > \frac{1}{2}
		$$
		因此得证。
	\end{proof}
\end{example}

\begin{theorem}[柯西收敛准则]
	一个数列 $\{a_n\}$ 收敛当且仅当它是基本列。

	\begin{proof}
		\begin{itemize}
			\item 充分性:设极限是 $a$,那么 $n$ 充分大时 $|a_n - a| < \frac{\varepsilon}{2}$。那么,$|a_n - a_m| \le |a_n - a| + |a_m - a| < \varepsilon$;

			\item 必要性:使用列紧性定理。首先取 $\varepsilon_0 = 1$,那么 $\exists\,N \in \mathbb{Z}^+: \forall\,n \ge N: |a_n - a_N| < \varepsilon_0 = 1$,这说明 $\{a_n\}$ 是有界的。那么 $\{a_n\}$ 有收敛子列 $\{a_{i_n}\}$,设其极限为 $a$。
			
			那么对于 $\forall\,\varepsilon > 0$,我们可以找到充分大的 $N = i_M$ 满足 $|a_N - a| < \frac{\varepsilon}{2}$ 且 $\forall\,n \ge N: |a_n - a_N| < \frac{\varepsilon}{2}$。这样就有 $\forall\,n \ge N: |a_n - a| < \varepsilon$,于是得证。
		\end{itemize}
	\end{proof}
\end{theorem}

结合上面的例,说明 $a_n = \sum_{i=1}^n \frac{1}{i^\alpha}\ (0 < \alpha \le 1)$ 是发散的。

给定集合 $A$,若有开区间族 $\{I_\lambda: \lambda \in \Lambda\}$,使得 $A \subseteq \bigcup_{\lambda \in \Lambda} I_\lambda$,那么这个开区间族覆盖了 $A$。也即:$\{I_\lambda\}$ 覆盖了 $A$ 当且仅当 $\forall\,x \in A: \exists\, I_{\lambda_0} \in \{I_\lambda\}: x \in I_{\lambda_0}$。据此有定理:

\begin{theorem}[有限覆盖定理(Heine-Borel 定理)]
	若有限闭区间 $[a,b]$ 被 $\{I_\lambda\}$ 开覆盖,则必可从中选择有限个开区间来覆盖 $[a,b]$。
\end{theorem}

\subsection{函数的极限}

和数列极限类似,给出函数趋于无穷的极限:

\begin{definition}
	若函数 $f(x)$ 在 $(a,\infty)$ 有定义,并且存在常数 $A$ 使得 $\forall\,\varepsilon > 0: \exists\,X > 0: \forall x > X: |f(x) - A| < \varepsilon$,那么称 $x \to +\infty$ 时函数 $f(x)$ 以 $A$ 为极限,记作

	$$
	\lim_{x \to +\infty} f(x) = A \text{ 或 } f(x) \to A\ (x \to +\infty) \text{ 或 } f(+\infty) = A
	$$

	同理也可以得到 $x \to -\infty$ 的极限的定义。一样可以记作:

	$$
	\lim_{x \to -\infty} f(x) = A \text{ 或 } f(x) \to A\ (x \to -\infty) \text{ 或 } f(-\infty) = A
	$$
\end{definition}

除此之外,函数的极限可以趋近于定点:

\begin{definition}
	若 $\forall\,\varepsilon > 0: \exists\,\delta > 0: \forall\,x \in \mathring{U}(x_0, \delta): |f(x) - A| < \varepsilon$,那么称 $A$ 为 $x \to x_0$ 的极限,记作:
	$$
	\lim_{x \to x_0} f(x) = A \text{ 或 } f(x) \to A\ (x \to x_0)
	$$

	这个定义就是所谓 $\varepsilon\text{-}\delta$ 语言。
\end{definition}

注意函数极限的定义不要求 $f(x)$ 在 $x = x_0$ 处有定义。

\subsection{作业}

\begin{problem}
	课后习题 2.4.8

	\begin{solution}
		\begin{enumerate}
			\item[\textbf{1)}]
			$$
			\begin{aligned}
				\text{原式} & = \lim_{n \to \infty} (1 + \frac{1}{n - 2})^{n - 2} \cdot (1 + \frac{1}{n - 2})^2 \\
				& = \lim_{n \to \infty} \ab(1 + \frac{1}{n - 2})^{n - 2} \cdot \ab(\lim_{n \to \infty} 
				\ab(1 + \frac{1}{n - 2}))^2 \\
				& = \E				
			\end{aligned}
			$$

			\item[\textbf{2)}] 当 $n \ge 2$ 时:
			$$
			\ab(\frac{1}{2})^{\frac{1}{n}} \le \ab(1 - \frac{1}{n})^{\frac{1}{n}} < 1^{\frac{1}{n}}
			$$
			而 $\lim\limits_{n \to \infty} \ab(\frac{1}{2})^{\frac{1}{n}} = \lim\limits_{n \to \infty} 1^{\frac{1}{n}} = 1$,由夹逼定理知 $\lim\limits_{n \to \infty} \ab(1 - \frac{1}{n})^{\frac{1}{n}} = 1$。
		\end{enumerate}
	\end{solution}
\end{problem}

\begin{problem}
	课后习题 2.4.9

	\begin{solution}
		\begin{enumerate}
			\item[\textbf{1)}] 使用 Stolz 定理:
			$$
			\text{原式} = \lim_{n \to \infty} \frac{\ln(n+1) - \ln n}{1} = \lim_{n \to \infty} \ln\ab(1 + \frac{1}{n})
			$$
			而 $0 < \ln\ab(1 + \frac{1}{n}) < \frac{1}{n}$,因此 $\lim\limits_{n \to \infty} \ln\ab(1 + \frac{1}{n}) = 0$。

			\item[\textbf{2)}] 使用 Stolz 定理:
			$$
			\text{原式} = \lim_{n \to \infty} \frac{1}{n (\ln n - \ln(n - 1))} = \lim_{n \to \infty} \frac{1}{n \ln\ab(1 + \frac{1}{n - 1})}
			$$
			而根据某不知名重要不等式,有:
			$$
			1 - \frac{1}{n} = \frac{1}{n \cdot \frac{1}{n - 1}} < \frac{1}{n \ln\ab(1 + \frac{1}{n - 1})} < \frac{1}{n \cdot \frac{1}{n}} = 1
			$$
			根据夹逼定理得到原极限为 $1$。
		\end{enumerate}
	\end{solution}
\end{problem}

\begin{problem}
	课后习题 2.4.11

	\begin{proof}
		\begin{enumerate}
			\item[\textbf{1)}]
			$$
			\begin{gathered}
				\begin{aligned}
					& \quad & \sqrt{n + 1} - \sqrt{n} & > \frac{1}{2 \sqrt{n + 1}} \\
					& \Leftrightarrow & 2(n + 1) - 2 \sqrt{n(n + 1)} & > 1 \\
					& \Leftrightarrow & (2n + 1)^2 & > 4n(n + 1) \\
					& \Leftrightarrow & 1 & > 0
				\end{aligned}
				\\
				\begin{aligned}
					& \quad & \sqrt{n + 1} - \sqrt{n} & < \frac{1}{2 \sqrt{n}} \\
					& \Leftrightarrow & 2 \sqrt{n(n + 1)} - 2n & < 1 \\
					& \Leftrightarrow & 4n(n + 1) & < (2n + 1)^2 \\
					& \Leftrightarrow & 0 & < 1
				\end{aligned}
			\end{gathered}
			$$

			\item[\textbf{2)}] 先证明 $\{x_n\}$ 的单调性:
			$$
			x_{n+1} - x_n = \frac{1}{\sqrt{n + 1}} - (2 \sqrt{n + 1} - 2 \sqrt{n}) < 0
			$$
			因此 $\{x_n\}$ 单调递减。再证明 $\{x_n\}$ 的有界性:
			$$
			\begin{aligned}
				x_n & = \frac{1}{\sqrt{n}} + \sum_{i=1}^{n-1} \frac{1}{\sqrt{i}} - 2 - 2 \sum_{i=1}^{n-1} \ab(\sqrt{i+1} - \sqrt{i}) \\
				& = -2 + \frac{1}{\sqrt{n}} + \sum_{i=1}^{n-1} \ab(\frac{1}{\sqrt{i}} - 2 \ab(\sqrt{i + 1} - \sqrt{i})) \\
				& > -2 + \frac{1}{\sqrt{n}} > -2
			\end{aligned}
			$$
			所以 $\{x_n\}$ 单调有界,极限存在。
		\end{enumerate}
	\end{proof}
\end{problem}

\begin{problem}
	课后习题 2.5.1

	\begin{proof}
		归纳证明 $x_n < y_n$。首先对于 $n = 1$ 成立,然后假设 $n$ 成立,那么根据基本不等式就有:
		$$
		x_n < x_{n+1} = \sqrt{x_n y_n} < \frac{x_n + y_n}{2} = y_{n+1} < y_n
		$$
		同时有
		$$
		y_{n+1} - x_{n+1} < \frac{x_n + y_n}{2} - x_n = \frac{y_n - x_n}{2}
		$$
		所以 $\lim\limits_{n \to \infty} (y_n - x_n) = 0$。因此 $I_n = [x_n, y_n]$ 构成一个闭区间套,根据闭区间套定理,$\lim\limits_{n \to \infty} x_n = \lim\limits_{n \to \infty} y_n$。
	\end{proof}
\end{problem}

\begin{problem}
	课后习题 2.5.3

	\begin{solution}
		\begin{enumerate}
			\item[\textbf{1)}] 注意到 $a_n = \ln n$ 满足条件并且发散。因此这种情况下 $\{a_n\}$ 不一定收敛。

			\item[\textbf{2)}] 此时有 $|a_{n+1} - a_n| < \frac{1}{n^2}$,因此:
			$$
			\begin{aligned}
				a_{n+p} - a_n & < \sum_{i=n+1}^{n+p} \frac{1}{i^2} \\
				& < \sum_{i=n+1}^{n+p} \frac{1}{i(i-1)} \\
				& = \frac{1}{n} - \frac{1}{n+p} < \frac{1}{n}
			\end{aligned}
			$$
			因此对于任意 $\varepsilon > 0$ 均可取到 $N = \ceil{\frac{1}{\varepsilon}}$,使得 $\forall,N \le n < m: |a_n - a_m| < \varepsilon$。所以 $\{a_n\}$ 是 Cauchy 列,根据柯西收敛准则 $\{a_n\}$ 收敛。
		\end{enumerate}
	\end{solution}
\end{problem}

\begin{problem}
	课后习题 2.5.4

	\begin{proof}
		\begin{enumerate}
			\item[\textbf{1)}] 考察 $a_{n+p} - a_n$:
			$$
			a_{n+p} - a_n = \sum_{i=n+1}^{n+p} \frac{(-1)^{i+1}}{i} = (-1)^n \sum_{i=1}^p \frac{(-1)^{i+1}}{n+i}
			$$
			若 $p$ 为偶数,将每一项分组:
			$$
			\begin{aligned}
				|a_{n+p} - a_n| & = \frac{1}{n+1} + \ab(-\frac{1}{n+2} + \frac{1}{n+3}) + \cdots + \ab(-\frac{1}{n+p-2} + \frac{1}{n+p-1}) - \frac{1}{n+p} \\
				& < \frac{1}{n+1} - \frac{1}{n+p} \\
				& < \frac{1}{n+1}
			\end{aligned}
			$$
			同理,若 $p$ 为奇数,也能得到 $a_{n+p} - a_n < \frac{1}{n+1}$。

			这样实际上证明了 $\{a_n\}$ 是 Cauchy 列,因此 $\{a_n\}$ 收敛。

			\item[\textbf{2)}]
			$$
			|a_{n+p} - a_n| = \ab|\sum_{i=n+1}^{n+p} \frac{\sin i}{2^i}| \le \sum_{i=n+1}^{n+p} \frac{1}{2^i} = \frac{1}{2^n} - \frac{1}{2^{n+p}} < \frac{1}{2^n}
			$$
			因此 $\{a_n\}$ 是 Cauchy 列,$\{a_n\}$ 收敛。

			\item[\textbf{3)}]
			$$
			|x_{n+p} - x_n| = \ab|\sum_{i=n+1}^{n+p} \frac{\cos i!}{i(i+1)}| \le \sum_{i=n+1}^{n+p} \frac{1}{i(i+1)} = \frac{1}{n+1} - \frac{1}{n+p+1} < \frac{1}{n+1}
			$$
			因此 $\{x_n\}$ 是 Cauchy 列,$\{x_n\}$ 收敛。
		\end{enumerate}
	\end{proof}
\end{problem}

\begin{problem}
	课后习题 2.5.7

	\begin{proof}
		考虑二分法证明列紧性定理的过程:

		\begin{enumerate}
			\item 初始时令 $l = \inf a_n,\ r = \sup a_n$;
			\item 找一个落在 $[l, r]$ 内且下标排在已选子列之后的 $a_i$ 加入子列;
			\item 令 $[l, r]$ 成为 $[l,\frac{l+r}{2}],[\frac{l+r}{2},r]$ 中包含无穷项 $\{a_n\}$ 的那个区间。回到第 2 步进行下一次迭代。
		\end{enumerate}

		若在某次迭代,第 3 步的两个区间都包含无穷项 $\{a_n\}$,那么就可以找到至少两个极限不同的子列。若每一次迭代中,第 3 步都只有一个选择,那么根据闭区间套定理可以证明 $\{a_n\}$ 收敛,矛盾。

		故得证。
	\end{proof}
\end{problem}

\begin{problem}
	课后习题 2.5.8

	\begin{proof}
		$x_{n+2} = 1 - \frac{1}{x_n + 2}$。设 $f(x) = 1 - \frac{1}{x + 2}$,定义 $a_n = x_{2n},\ b_n = x_{2n+1}$,那么 $a_{n+1} = f(a_n),\ b_{n+1} = f(b_n)$。

		解不等式可得 $\frac{1}{2} \le x < \frac{\sqrt{5} - 1}{2}$ 时,$x < f(x) < \frac{\sqrt{5} - 1}{2}$;$\frac{\sqrt{5} - 1}{2} < x \le 1$ 时,$\frac{\sqrt{5} - 1}{2} < f(x) < x$。而 $a_0 = 1,\ b_0 = \frac{1}{2}$,归纳可证 $\{a_n\}, \{b_n\}$ 单调有界,均收敛。

		设 $\lim\limits_{n \to \infty} a_n = A$,那么 $A = f(A)$,解得 $A = \frac{\sqrt{5} - 1}{2}$。同理解得 $\lim\limits_{n \to \infty} b_n = \frac{\sqrt{5} - 1}{2}$。因此 $\{x_n\}$ 的极限也是 $\frac{\sqrt{5} - 1}{2}$。
	\end{proof}
\end{problem}