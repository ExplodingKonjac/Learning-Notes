\section{函数极限 V}

\subsection{一致连续}

\begin{definition}[函数的一致连续性]
	设函数 $f(x)$ 在区间 $I$ 上有定义,若:
	$$
	\forall\,\varepsilon > 0: \exists\,\delta > 0: \forall\,x_1, x_2 \in I \land |x_1 - x_2| < \delta: |f(x_1) - f(x_2)| < \varepsilon
	$$
	则称 $f(x)$ 在 $I$ 上一致连续。
\end{definition}

容易看出,一致连续一定连续,反之则未必,如 $f(x) = \frac{1}{x}$。

\begin{example}
	证明 $f(x) = \sin x$ 一致连续。

	\begin{proof}
		$$
		|\sin x_1 - \sin x_2| = \ab|2 \sin \frac{x_1 - x_2}{2} \cos \frac{x_1 + x_2}{2}| \le \ab|2 \sin \frac{x_1 - x_2}{2}| \le x_1 - x_2
		$$
		取 $\delta = \varepsilon$ 即可。
	\end{proof}
\end{example}

将一致连续定义取反得到不一致连续的定义:

\begin{definition}[函数不一致连续]
	设函数 $f(x)$ 在区间 $I$ 上有定义,若:
	$$
	\exists\,\varepsilon > 0: \forall\,\delta > 0: \exists\,x_1, x_2 \in I \land |x_1 - x_2| < \delta: |f(x_1) - f(x_2)| > \varepsilon
	$$
	则称 $f(x)$ 在 $I$ 上不一致连续。
\end{definition}

根据定义可得到一个重要性质:

\begin{theorem}
	若 $f(x)$ 在 $(a, b)$ 上一致连续,那么 $f(a^+), f(b^-)$ 存在。

	\begin{proof}
		用定义结合 Cauchy 收敛准则即可证明。
	\end{proof}
\end{theorem}

这说明对于任意一个 $(a, b)$ 上一致连续的 $f(x)$,都可以延拓一个新的函数
$$
f^*(x) = \begin{cases}
	f(x) & , x \in (a, b) \\
	f(a^+) & , x = a \\
	f(b^-) & , x = b
\end{cases}
$$
那么 $f^*(x)$ 是 $[a, b]$ 上的连续函数。

证明函数是否一致连续一般不会直接用定义,我们有定理:

\begin{theorem}
	$f(x)$ 在 $I$ 上一致连续的充要条件是:任意 $I$ 中的数列 $\{x_n'\},\{x_n''\}$,若 $\lim\limits_{n \to \infty} (x_n' - x_n'') = 0$,那么 $\lim\limits_{n \to \infty} (f(x_n') - f(x_n'')) = 0$。

	\begin{proof}
		\begin{itemize}
			\item 必要性:函数一致连续,即
			$$
			\forall\,\varepsilon > 0: \exists\,\delta > 0: \forall\,x_1, x_2 \in I \land |x_1 - x_2| < \delta: |f(x_1) - f(x_2)| < \varepsilon
			$$
			而根据 $\lim\limits_{n \to \infty} (x_n' - x_n'') = 0$,有:
			$$
			\exists\,N \in \mathbb{Z}^+: \forall n \ge N: |x_n' - x_n''| < \delta \Rightarrow |f(x_n') - f(x_n'')| < \varepsilon
			$$
			于是得证。

			\item 充分性:类似 Heine 定理,使用反证法,若 $f(x)$ 不一致连续:
			$$
			\exists\,\varepsilon_0 > 0: \forall\,\delta > 0: \exists\,x_1, x_2 \in I \land |x_1 - x_2| < \delta: |f(x_1) - f(x_2)| > \varepsilon_0
			$$
			那么取 $x_n', x_n''$ 为 $\delta = \frac{1}{n}$ 时的一对 $x_1, x_2$。那么显然 $\lim\limits_{n \to \infty} (x_n' - x_n'') = 0$ 但 $\lim\limits_{n \to \infty} (f(x_n') - f(x_n''))$ 不存在或不等于 $0$,矛盾。故得证。
		\end{itemize}
	\end{proof}
\end{theorem}

\begin{example}
	证明 $f(x) = x^2$ 在 $[0, +\infty)$ 不一致连续,但在 $[0, A]\ (A \ge 0)$ 上一致连续。

	\begin{proof}
		取 $x_n' = \sqrt{n},\ x_n'' = \sqrt{n + 1}$,那么:
		$$
		\lim_{n \to \infty} (x_n' - x_n'') = \lim_{n \to \infty} \frac{1}{\sqrt{n} + \sqrt{n + 1}} = 0
		$$
		但
		$$
		f(x_n') - f(x_n'') = \ab(\sqrt{n})^2 - \ab(\sqrt{n+1})^2 = -1
		$$
		因此函数不一致连续。

		而对于 $[0, A]$ 的情况,有:
		$$
		\ab|f(x_n') - f(x_n'')| = \ab|(x_n' + x_n'') (x_n' - x_n'')| \le 2A \ab|x_n' - x_n''|
		$$
		根据有界函数乘无穷小还是无穷小,$\lim\limits_{n \to \infty} (f(x_n') - f(x_n'')) = 0$,因此函数一致连续。
	\end{proof}
\end{example}

\subsection{有限闭区间上连续函数的性质}

\begin{theorem}[Cantor 定理]
	若 $f(x)$ 在 $[a, b]$ 上连续,那么 $f(x)$ 在 $[a, b]$ 上一致连续。

	\begin{proof}
		考虑反证法。假设 $f(x)$ 不一致连续,那么:
		$$
		\exists\,\varepsilon_0 > 0: \forall\,n \in \mathbb{Z}^+: \exists\,s_n, t_n \in [a, b] \land |s_n - t_n| < \frac{1}{n}: |f(s_n) - f(t_n)| > \varepsilon_0
		$$
		因为 $s_n \in [a, b]$,则 $\{s_n\}$ 有收敛子列 $\{s_{n_k}\}$,设 $\lim\limits_{k \to \infty} s_{n_k} = s \in [a, b]$。

		而 $\lim\limits_{k \to \infty} (s_{n_k} - t_{n_k}) = \lim\limits_{n \to \infty} (s_n - t_n) = 0$,因此 $\lim\limits_{k \to \infty} t_{n_k} = \lim\limits_{k \to \infty} s_{n_k} = s$。于是:
		$$
		\begin{aligned}
			\lim_{k \to \infty} \ab(f(s_{n_k}) - f(t_{n_k})) & = \lim_{k \to \infty} f(s_{n_k}) - \lim_{k \to \infty} f(t_{n_k}) \\
			& = f\ab(\lim_{k \to \infty} s_{n_k}) - f\ab(\lim_{k \to \infty} t_{n_k}) \\
			& = f(s) - f(s) = 0
		\end{aligned}
		$$
		矛盾。故得证。
	\end{proof}
\end{theorem}

\begin{theorem}[有界性定理]
	若 $f(x)$ 在 $[a, b]$ 上连续,那么 $f(x)$ 在 $[a, b]$ 上有界。

	\begin{proof}
		考虑反证法。假设 $f(x)$ 在 $[a, b]$ 上无界,那么:
		$$
		\forall\,n \in \mathbb{Z}^+: \exists\,x_n \in [a, b]: f(x_n) > n
		$$
		而 $x_n \in [a, b]$,根据列紧性定理有子列 $\{x_{n_k}\}$ 收敛。设 $\lim\limits_{n \to \infty} x_{n_k} = \xi$,则 $\lim\limits_{n \to \infty} f(x_{n_k}) = f(\xi)$。但是这与 $f(x_n) > n$ 矛盾。故得证。
	\end{proof}
\end{theorem}

\begin{corollary}
	若 $f(x)$ 在 $(a, b)$ 上一致连续,那么 $f(x)$ 在 $(a, b)$ 上有界。
\end{corollary}

\begin{theorem}[最值定理]
	若 $f(x)$ 在 $I = [a, b]$ 上连续,那么 $f(I)$ 有最大值和最小值。

	\begin{proof}
		下面证明最大值存在,最小值同理:记 $M = \sup f(I)$,那么:
		$$
		\forall\,n \in \mathbb{Z}^+: \exists\,x_n \in [a, b]: M - f(x_n) < \frac{1}{n}
		$$
		而 $\{x_n\}$ 有界,根据列紧性定理有收敛子列 $\{x_{n_k}\}$,设其极限为 $\xi$,那么:
		$$
		\lim_{k \to \infty} f(x_{n_k}) = f\ab(\lim_{k \to \infty} x_{n_k}) = f(\xi)
		$$
		而根据 $x_n$ 定义知 $\lim\limits_{k \to \infty} f(x_{n_k}) = \lim\limits_{n \to \infty} f(x_n) = M$,即 $f(\xi) = M$,于是 $f(x)$ 有最大值。故得证。
	\end{proof}
\end{theorem}

\begin{theorem}[零点存在性定理]
	若 $f(x)$ 在 $[a, b]$ 上连续且 $f(a) f(b) < 0$,那么 $\exists\,\xi \in (a, b): f(\xi) = 0$。

	\begin{proof}
		使用二分法,归纳构造一列区间 $I_n = [a_n, b_n]$:$a_1 = a,\ b_1 = b$,设 $m_n = \frac{a_n + b_n}{2}$,不妨设 $f(a) < 0 \land f(b) > 0$,进行讨论:

		\begin{itemize}
			\item 若 $f(m_n) = 0$:取 $\xi = m_n$,命题得证;
			\item 若 $f(m_n) < 0$:那么 $a_{n+1} = m_n,\ b_{n+1} = b_n$;
			\item 若 $f(m_n) > 0$:那么 $a_{n+1} = a_n,\ b_{n+1} = m_n$。
		\end{itemize}

		归纳可知始终有 $f(a_n) < 0 \land f(b_n) > 0$。

		那么 $I_{n+1} \subset I_n$ 且 $\lim\limits_{n \to \infty} (b_n - a_n) = \lim\limits_{n \to \infty} \frac{b - a}{2^{n - 1}} = 0$。于是 $\{I_n\}$ 构成闭区间套,因此:
		$$
		\lim_{n \to \infty} a_n = \lim_{n \to \infty} b_n = \xi
		$$
		而根据函数连续性和极限的保号性,有:
		$$
		\begin{gathered}
			f(a_n) > 0 \Rightarrow \lim_{n \to \infty} f(a_n) = f(\xi) \ge 0 \\
			f(b_n) < 0 \Rightarrow \lim_{n \to \infty} f(b_n) = f(\xi) \le 0
		\end{gathered}
		$$
		因此 $f(\xi) = 0$,得证。
	\end{proof}
\end{theorem}

\begin{theorem}[介值定理]
	若 $f(x)$ 在 $[a, b]$ 连续,且 $C \in \ab(\min\{f(a), f(b)\}, \max\{f(a), f(b)\})$,那么 $\exists\,\xi \in (a, b): f(\xi) = C$。

	\begin{proof}
		构造 $g(x) = f(x) - C$ 使用零点存在性定理即可。
	\end{proof}
\end{theorem}

\subsection{作业}

\begin{problem}
	课后习题 3.5.2

	\begin{proof}
		对于任意 $[0, +\infty)$ 上数列 $\{x_n'\}, \{x_n''\}$ 满足 $\lim\limits_{n \to \infty} (x_n' - x_n'') = 0$,那么:
		$$
		\begin{aligned}
			|f(x_n') - f(x_n'')| & = |\sqrt{x_n'} - \sqrt{x_n''}| \\
			& = \ab|\frac{x_n' - x_n''}{\sqrt{x_n'} + \sqrt{x_n''}}| \\
			& \le \ab|\frac{x_n' - x_n''}{\sqrt{|x_n' - x_n''|}}| \\
			& = \sqrt{|x_n' - x_n''|}
		\end{aligned}
		$$
		因此 $\lim\limits_{n \to \infty} |f(x_n') - f(x_n'')| = 0$,得证。
	\end{proof}
\end{problem}

\begin{problem}
	课后习题 3.5.3

	\begin{solution}
		\begin{enumerate}
			\item[\textbf{2)}] 取定义域上任意两个数列 $\{x_n'\},\{x_n''\}$ 满足 $\lim\limits_{n \to \infty} (x_n' - x_n'') = 0$。
			\begin{enumerate}
				\item[\textbf{a)}] 不一致连续,原因:
				
				取 $x_n' = \frac{2}{n},\ x_n'' = \frac{1}{n}$,那么:
				$$
				\lim_{n \to \infty} f(x_n') - f(x_n'') = \lim_{n \to \infty} \ln 2 \neq 0
				$$
				于是得证。

				\item[\textbf{b)}] 一致连续,原因:
				$$
				\lim_{n \to \infty} (f(x_n') - f(x_n'')) = \lim_{n \to \infty} \ln{\frac{x_n'}{x_n''}} = \lim_{n \to \infty} \ln \ab(1 + \frac{x_n' - x_n''}{x_n''})
				$$
				因为 $x_n'' \ge 1$,因此 $\ab|\frac{x_n' - x_n''}{x_n''}| \le |x_n' - x_n''|$,于是:
				$$
				\lim_{n \to \infty} (f(x_n') - f(x_n'')) = \ln \ab(1 + \lim_{n \to \infty} \frac{x_n' - x_n''}{x_n''}) = \ln(1 + 0) = 0
				$$
				于是得证。
			\end{enumerate}
		\end{enumerate}
	\end{solution}
\end{problem}

\begin{problem}
	课后习题 3.5.4

	\begin{proof}
		由题知:
		$$
		\forall\,\varepsilon > 0: \forall\,x, y \in I \land |x - y| < \frac{\varepsilon}{L}: |f(x) - f(y)| < \varepsilon
		$$
		因此 $f(x)$ 在 $I$ 上一致连续。
	\end{proof}
\end{problem}

\begin{problem}
	课后习题 3.5.5

	\begin{solution}
		取 $x_n = 2n \pi + \frac{\pi}{2}, y_n = 2n \pi$,那么:
		$$
		f(x_n) - f(y_n) = \ab(2n \pi + \frac{\pi}{2}) - \frac{2n \pi}{1 + 2n \pi} = 2n \pi + \frac{1}{1 + 2n \pi} + \frac{\pi}{2} - 1
		$$
		因此
		$$
		\lim_{n \to \infty} (f(x_n) - f(y_n)) = +\infty
		$$
		若 $f(x)$ 一致连续,那么
		$$
		\forall\,\varepsilon > 0: \exists\,\delta > 0: |f(x_n) - f(y_n)| < \varepsilon \cdot \ceil{\frac{\pi / 2}{\delta}}
		$$
		这说明在选定 $\varepsilon, \delta$ 时 $|f(x_n) - f(y_n)|$ 有界,而这与上面结果矛盾,因此 $f(x)$ 不一致连续。
	\end{solution}
\end{problem}

\begin{problem}
	课后习题 3.6.3

	\begin{proof}
		根据无穷大定义,任取 $A$,存在 $\delta_1 > 0$ 使得
		$$
		\begin{gathered}
			\forall\,x \in (a, a + \delta_1): f(x) > A \\
			\forall\,x \in (b - \delta_1, b): f(x) > A
		\end{gathered}
		$$
		不妨设 $\delta_1$ 足够小,使得 $b - \delta_1 > a + \delta_1$。那么 $f(x)$ 在 $[a + \delta_1, b - \delta_1]$ 上有最小值 $m$,则 $\min\{m, A\}$ 是 $f(x)$ 在 $(a, b)$ 的一个下界。

		那么设 $f(x)$ 在 $(a, b)$ 的下确界为 $m'$,那么同理存在 $\delta_2 > 0$ 使得
		$$
		\begin{gathered}
			\forall\,x \in (a, a + \delta_2): f(x) > m'+1 \\
			\forall\,x \in (b - \delta_2, b): f(x) > m'+1
		\end{gathered}
		$$
		可知 $f(x)$ 在 $[a + \delta_2, b - \delta_2]$ 的下确界是 $m'$,根据最值定理,$f(x)$ 在 $[a + \delta_2, b + \delta_2]$ 中可取到最小值 $m'$。
	\end{proof}
\end{problem}

\begin{problem}
	课后习题 3.6.4

	\begin{proof}
		和上题同理。任取 $A$ 都存在 $X_1 > 0$ 使得:
		$$
		\begin{gathered}
			\forall\,x < -X_1: f(x) < A \\
			\forall\,x > X_1: f(x) < A
		\end{gathered}
		$$
		而 $f(x)$ 在 $[-X,X]$ 有最大值 $m$,那么 $\max\{m, A\}$ 是 $f(x)$ 的一个上界,因此 $f(x)$ 也有上确界。

		设 $m'$ 为 $f(x)$ 的上确界,那么存在 $X_2 > 0$:
		$$
		\begin{gathered}
			\forall\,x < -X_2: f(x) < m' - 1 \\
			\forall\,x > X_2: f(x) < m' - 1
		\end{gathered}
		$$
		则 $m'$ 也是 $f(x)$ 在 $[-X, X]$ 的上确界,根据最值定理 $f(x)$ 在 $[-X, X]$ 中可取到最大值 $m'$。
	\end{proof}
\end{problem}

\begin{problem}
	课后习题 3.6.6

	\begin{proof}
		考虑反证法。任取 $\varepsilon_0 \in (0, \min\{x^* - a, b - x^*\})$,若 $\lim\limits_{x \to \infty} x_n \neq x^*$ 那么:
		$$
		\forall N > \mathbb{Z}^+: \exists\, n \ge N: |x_n - x^*| \ge \varepsilon_0
		$$
		根据最值定理,存在 $M = \max\limits \{f(x) \mid x \in [a, x^* - \varepsilon_0] \cup [x^* + \varepsilon_0, b]\} > f(x^*)$。于是:
		$$
		\forall N > \mathbb{Z}^+: \exists\, n \ge N: f(x_n) - f(x^*) \ge M - f(x^*) > 0
		$$
		这说明 $\{f(x_n)\}$ 不收敛于 $f(x^*)$,矛盾,故得证。
	\end{proof}
\end{problem}

\begin{problem}
	课后习题 3.6.7

	\begin{proof}
		记 $f(x) = 2^x - 4x$。注意到 $f(0) = 1 > 0,\ f\ab(\frac{1}{2}) = \sqrt{2} - 2 < 0$,根据零点存在性定理得证。
	\end{proof}
\end{problem}

\begin{problem}
	课后习题 3.6.9

	\begin{proof}
		取一个 $\varepsilon_0 \in (0, \min\{\eta - A, B - \eta\})$,那么根据极限定义和极限的保序性可知,存在 $N \in \mathbb{Z}^+$ 使得:
		$$
		\forall n \ge N: f(x_n) < A + \varepsilon_0 \land f(y_n) > B - \varepsilon_0 \land x_n < y_n
		$$
		根据介值定理,$\exists\,z_n \in (x_{n-N+1}, y_{n-N+1}): f(z_n) = \eta$,那么 $\{z_n\}$ 就是一个符合题意的数列。
	\end{proof}
\end{problem}

\begin{problem}
	课后习题 3.6.10

	\begin{proof}
		若 $f(x)$ 在 $[a, b]$ 上等号,那么显然结论成立。否则,不妨设有 $f(x_1) > 0,\ f(x_2) < 0,\ x_1 < x_2$,那么根据介值定理 $\exists\,\xi \in (x_1, x_2): f(\xi) = 0$。于是 $|f(x_1)| > |f(\xi)|$ 但 $|f(\xi)| < |f(x_2)|$,这与 $|f(x)|$ 单调性矛盾。因此得证。
	\end{proof}
\end{problem}