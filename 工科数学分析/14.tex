\section{L'Hospital 法则 \& 微分}

\subsection{L'Hospital 法则}

如果当 $x \to a\ (x \to \pm \infty)$ 时两个函数 $f(x), F(x)$ 均趋于零或均趋于无穷大,那么成极限 $\lim \frac{f(x)}{F(x)}$ 为未定式。通常我们把这种极限简记为 $\frac{0}{0}$ 型和 $\frac{\infty}{\infty}$ 型。

\begin{theorem}[趋于定点时 $\frac{0}{0}$ 型的 L'Hospital 法则]
	设 $f(x), g(x)$ 在 $x_0$ 的某个去心邻域内可导,且 $g(x) \neq 0$,如果:

	\begin{itemize}
		\item $\lim\limits_{x \to x_0} f(x) = \lim\limits_{x \to x_0} g(x) = 0$;
		\item $g'(x) \neq 0$;
		\item $\lim\limits_{x \to x_0} \frac{f'(x)}{g'(x)} = a$($a$ 可以为有限数或无穷大)。
	\end{itemize}

	那么 $\lim\limits_{x \to x_0} \frac{f(x)}{g(x)} = \lim\limits_{x \to x_0} \frac{f'(x)}{g'(x)} = a$。极限为单侧的时候也有类似结论。

	\begin{proof}
		因为 $f(x), g(x)$ 在 $x_0$ 去心邻域内连续,那么也在这个邻域内连续。分类讨论:

		\begin{itemize}
			\item 若 $f(x), g(x)$ 在 $x_0$ 连续:那么 $f(x_0) = g(x_0) = 0$。根据柯西中值定理:
			$$
			\frac{f(x)}{g(x)} = \frac{f(x) - f(x_0)}{g(x) - g(x_0)} = \frac{f'(\xi_x)}{g'(\xi_x)}
			$$
			而 $|\xi_x - x_0| < |x - x_0|$,因此 $\xi_x \to x_0$,于是就证明了。

			\item 若 $f(x), g(x)$ 在 $x_0$ 不连续:那么 $x_0$ 是一个可去间断点。构造一个连续函数:
			$$
			\hat{f}(x) = \begin{cases}
				f(x) & , x \neq x_0 \\
				0 & , x = x_0
			\end{cases},\quad \hat{g}(x) = \begin{cases}
				g(x) & , x \neq x_0 \\
				0 & , x = x_0
			\end{cases}
			$$
			那么 $\hat{f}(x), \hat{g}(x)$ 在去心邻域内可导。仍然可以用上面的证明。
		\end{itemize}
	\end{proof}
\end{theorem}

\begin{theorem}[趋于无穷时 $\frac{0}{0}$ 型的 L'Hospital 法则]
	设 $f(x), g(x)$ 在 $(a, +\infty)$ 可导,且 $g(x) \neq 0$,如果:

	\begin{itemize}
		\item $\lim\limits_{x \to +\infty} f(x) = \lim\limits_{x \to x_0} g(x) = 0$;
		\item $g'(x) \neq 0$;
		\item $\lim\limits_{x \to +\infty} \frac{f'(x)}{g'(x)} = a$($a$ 可以为有限数或无穷大)。
	\end{itemize}

	那么 $\lim\limits_{x \to +\infty} \frac{f(x)}{g(x)} = \lim\limits_{x \to +\infty} \frac{f'(x)}{g'(x)} = a$。趋于负无穷也有类似结论。

	\begin{proof}
		$$
		\begin{aligned}
			\lim_{x \to +\infty} \frac{f(x)}{g(x)} & = \lim_{t \to 0^+} \frac{f\ab(\frac{1}{t})}{g\ab(\frac{1}{t})} \\
			& = \lim_{t \to 0^+} \frac{\ab(f\ab(\frac{1}{t}))'}{\ab(g\ab(\frac{1}{t}))'} \\
			& = \lim_{t \to 0^+} \frac{-f'(\frac{1}{t}) \cdot \frac{1}{t^2}}{-g'(\frac{1}{t}) \cdot \frac{1}{t^2}} \\
			& = \lim_{t \to 0^+} \frac{f'(\frac{1}{t})}{g'(\frac{1}{t})} = \lim_{x \to +\infty} \frac{f'(x)}{g'(x)}
		\end{aligned}
		$$
	\end{proof}
\end{theorem}

\begin{theorem}[趋于定点时 $\frac{\infty}{\infty}$ 型的 L'Hospital 法则]
	设 $f(x), g(x)$ 在 $x_0$ 的某个去心邻域内可导,且 $g(x) \neq 0$,如果:

	\begin{itemize}
		\item $\lim\limits_{x \to x_0} g(x) = \infty$;
		\item $g'(x) \neq 0$;
		\item $\lim\limits_{x \to x_0} \frac{f'(x)}{g'(x)} = a$($a$ 可以为有限数或无穷大)。
	\end{itemize}

	那么 $\lim\limits_{x \to x_0} \frac{f(x)}{g(x)} = \lim\limits_{x \to x_0} \frac{f'(x)}{g'(x)} = a$。极限为单侧的时候也有类似结论。

	\begin{proof}
		类似地使用 Cauchy 中值定理证明。
	\end{proof}
\end{theorem}

同理,趋于无穷时 $\frac{\infty}{\infty}$ 型的 L'Hospital 法则也成立。注意在 $\frac{\infty}{\infty}$ 的 L'Hospital 法则描述中,没有要求 $f(x) \to \infty$。

L'Hospital 法则的描述是,$\lim \frac{f'(x)}{g'(x)}$ \textbf{存在或为无穷时}才能推得原本的极限,如果不是这种情况,洛必大法则失效,此时原极限不一定不存在。

\subsection{函数的微分}

为了研究在自变量有微小改变的时候,函数值相应的改变,我们定义了函数的微分。

\begin{definition}[函数的微分]
	若有函数 $y = f(x)$,当 $x$ 有一个增加量 $\Delta x$ 时,记 $\Delta y = f(x_0 + \Delta x) - f(x_0)$,如果存在一个常数 $A$ 使得 $\Delta x \to 0$ 时有:
	$$
	\Delta y = f(x_0) + A \Delta x + o(\Delta x)
	$$
	那么称 $f(x)$ 在 $x_0$ 处\textbf{可微},记 $A \Delta x$ 为 $f(x)$ 在 $x_0$ 处的\textbf{微分},记作:
	$$
	\left.\dif f\right|_{x = x_0} = A \dif x
	$$
	函数的微分也被称为函数增加量的\textbf{线性主要部分}。
\end{definition}

\subsection{作业}

\begin{problem}
	课后习题 4.6.1

	\begin{solution}
		\begin{enumerate}
			\item[\textbf{1)}]
			$$
			f'(x) = \ab(\ln \ab(1 + \frac{1}{x}) - \frac{1}{x}) \ab(1 + \frac{1}{x})^{x + 1}
			$$
			而 $\ln\ab(1 + \frac{1}{x}) \le \frac{1}{x}$,因此 $f'(x) \le 0$,$f(x)$ 在 $(0, +\infty)$ 单调递减。
		\end{enumerate}
	\end{solution}
\end{problem}

\begin{problem}
	课后习题 4.6.2

	\begin{solution}
		\begin{enumerate}
			\item[\textbf{1)}] 设 $f(x) = \sin x - x$,那么 $x > 0$ 时:
			$$
			f'(x) = \cos x - 1 \le 0
			$$
			因此 $f(x)$ 单调递减, $f(x) < f(0) = 0$。

			再设 $g(x) = \sin x - x + \frac{x^3}{6}$,那么:
			$$
			\begin{gathered}
				g'(x) = \cos x - 1 + \frac{x^2}{2} \\
				g''(x) = -\sin x + x \ge 0
			\end{gathered}
			$$
			因此 $g'(x)$ 单调递增,$g'(x) > g'(0) = 0$;于是 $g(x)$ 单调递增,$g(x) > g(0) = 0$。得证。

			\item[\textbf{3)}] 设 $f(x) = x^2 - (1 + x) \ln^2 (1 + x)$,那么
			$$
			\begin{aligned}
				f'(x) & = 2x - \ln^2 (x + 1) - 2 \ln(x + 1) \\
				f''(x) & = 2 - 2 \frac{\ln(x + 1)}{x + 1} - \frac{2}{x + 1} \\
				& = 2 \ab(1 - \frac{\ln(x + 1) - 1}{x + 1}) \\
				& > 2 \ab(1 - \frac{x}{x + 1}) > 0
			\end{aligned}
			$$
			因此 $f'(x)$ 单调递增,$f'(x) > f'(0) = 0$;因此 $f(x)$ 单调递增,$f(x) > f(0) = 0$。得证。

			\item[\textbf{5)}] 设 $f(x) = \tan x + 2 \sin x - 3x$,那么:
			$$
			f'(x) = \frac{1}{\cos^2 x} + 2 \cos x - 3 = \frac{1 - 3 \cos^2 x + 2 \cos^3 x}{\cos^2 x}
			$$
			那么 $\cos x \in (0, 1)$。再设 $g(t) = 1 - 3t^2 + 2t^3$,那么:
			$$
			g'(t) = -6t + 6t^2 = 6t(t - 6)
			$$
			那么当 $t \in (0, 1)$ 时 $g'(t) < 0$,$g(t)$ 单调递减,$g(t) > g(1) = 0$。因此 $f'(x) > 0$,得到 $f(x)$ 单调递增,于是 $f(x) > f(0) = 0$。
		\end{enumerate}
	\end{solution}
\end{problem}

\begin{problem}
	课后习题 4.6.4

	\begin{solution}
		使用反证法。若存在 $f(x) \neq 0$,不妨设 $f(x) > 0$,那么存在一个区间 $(a, a + \varepsilon)$ 使得 $f(a) = 0,\ \forall x \in (a, b),\ f(x) > 0$。

		那么设 $g(x) = \ln f(x)$,则 $g'(x) = \frac{f'(x)}{f(x)}$,由题知 $|g'(x)| \le A$,也就是说 $g(x)$ 在 $(a, a + \varepsilon)$ 一致连续。然而 $x \to a^+, f(x) \to 0^+, g(x) \to -\infty$,因此 $g(x)$ 肯定不一致连续,矛盾。
	\end{solution}
\end{problem}

\begin{problem}
	课后习题 4.6.7

	\begin{solution}
		$f(x)$ 是极大值,可得 $f(x)$ 在一个邻域内是最大值,因此
		$$
		\begin{gathered}
			\lim_{x \to x_0^+} \frac{f(x) - f(x_0)}{x - x_0} \le 0 \\
			\lim_{x \to x_0^-} \frac{f(x) - f(x_0)}{x - x_0} \ge 0
		\end{gathered}
		$$
		因此 $f'(x) = 0$。

		同时 $f'(x)$ 在一个左邻域内必须 $\ge 0$,在一个右邻域内必须 $\le 0$。否则根据 $f(x)$ 单调性推出矛盾。因此:
		$$
		\lim_{x \to x_0} \frac{f'(x) - f'(x_0)}{x - x_0} \ge 0
		$$
		于是得证。
	\end{solution}
\end{problem}

\begin{problem}
	课后习题 4.6.12

	\begin{solution}
		将椭圆沿 $x, y$ 轴拉伸变为半径为 $1$ 的圆。设切点为 $(\cos \theta, \sin \theta)\ (\theta \in (0, \frac{\pi}{2}))$,那么 $x, y$ 轴上的截距分别为:
		$$
		\frac{1}{\cos \theta}, \frac{1}{\sin \theta}
		$$
		那么三角形面积为:
		$$
		S = \frac{1}{2 \cos \theta \sin \theta} = \frac{1}{\sin 2 \theta}
		$$
		当 $\theta = \frac{\pi}{4}$ 时 $S$ 取到最小值 $1$。将椭圆还原为原本的坐标系,得到 $\min S = ab$。
	\end{solution}
\end{problem}

\begin{problem}
	课后习题 4.6.15

	\begin{solution}
		对于某点 $x_0$,设 $\varphi(x) = \frac{f(x) - f(x_0)}{x - x_0}$,那么根据凸函数性质,$\varphi(x)$ 在 $(a, x_0), (x_0, b)$ 上单调递增。且任意 $x_1 \in (a, x_0), x_2 \in (x_0, b)$ 都有 $\varphi(x_1) < \varphi(x_2)$。因此 $\varphi(x)$ 在 $(a, x_0), (x_0, b)$ 上均单调有界,因此 $f'_-(x_0), f'_+(x_0)$ 存在,这也说明 $f(x)$ 在 $x_0$ 连续。
	\end{solution}
\end{problem}

\begin{problem}
	课后习题 4.6.16

	\begin{proof}
		设 $g(x) = \frac{f(x) - f(0)}{x - 0}$,那么可得
		$$
		\lim_{x \to \infty} g(x) = \lim_{x \to \infty} \frac{f(x) - f(0)}{x - 0} = \lim_{x \to \infty} \frac{f(x)}{x} + \lim_{x \to \infty} \frac{f(0)}{x} = \lim_{x \to \infty} \frac{f(x)}{x} = 0
		$$
		而根据凸函数性质,$g(x)$ 在 $\mathbb{R} \setminus \{0\}$ 单调递增。因此
		$$
		0 = \lim_{x \to -\infty} g(x) \le g(x) \le \lim_{x \to +\infty} g(x) = 0
		$$
		于是 $g(x) = 0$,也就是说 $f(x)$ 是常函数。
	\end{proof}
\end{problem}

\begin{problem}
	课后习题 4.6.18

	\begin{proof}
		\begin{enumerate}
			\item[\textbf{2)}]
			$$
			\begin{gathered}
				f'(x) = \frac{1}{x} - \frac{1}{x+1} \\
				f''(x) = -\frac{1}{x^2} + \frac{1}{(x+1)^2} < 0
			\end{gathered}
			$$
			因此 $f(x)$ 是凹函数。
		\end{enumerate}
	\end{proof}
\end{problem}

\begin{problem}
	课后习题 4.6.19

	\begin{solution}
		\begin{enumerate}
			\item[\textbf{4)}] 
			$$
			y'' = \frac{2 \ab(x^2 - 10 x - 2)}{9 (x - 2)^3} \ab(\frac{(x + 1)^2}{x - 2})^{-2/3}
			$$
			由 $y'' = 0$ 解得,$x = 5 \pm 3 \sqrt 3$。
		\end{enumerate}
	\end{solution}
\end{problem}

\begin{problem}
	课后习题 4.6.20

	\begin{solution}
		要证的不等式等价于:
		$$
		\frac{1}{p} \ln x + \frac{1}{q} \ln y \le \ln(\frac{x}{p} + \frac{y}{q})
		$$
		因为 $\frac{1}{p} + \frac{1}{q} = 1$,根据 $\ln x$ 的凹性和 Jensen 不等式得证。
	\end{solution}
\end{problem}

\begin{problem}
	课后习题 4.6.21

	\begin{proof}
		\begin{enumerate}
			\item[\textbf{1)}] 要证的不等式等价于:
			$$
			\frac{|a|^p}{2} + \frac{|b|^p}{2} \ge \ab(\frac{|a| + |b|}{2})^p
			$$
			因为 $p > 1$ 时 $y = |x|^p$ 是凸的,因此根据 Jensen 不等式得证。

			\item[\textbf{2)}] 同理,$0 < p < 1$ 时 $y = x^p\ (x > 0)$ 是凹的,根据 Jensen 不等式得证。
		\end{enumerate}
	\end{proof}
\end{problem}

\begin{problem}
	课后习题 4.6.22

	\begin{proof}
		\begin{enumerate}
			\item[\textbf{2)}] 等式两侧取对数:
			$$
			\begin{aligned}
				\frac{a + b + c}{3}(\ln a + \ln b + \ln c) & \le a \ln a + b \ln b + c \ln c \\
				\frac{1}{3}\ab((b + c - 2a) \ln a + (a + c - 2b) \ln b + (a + b - 2c) \ln c) & \le 0 \\
				\frac{1}{3}\ab((a - b) (\ln b - \ln a) + (b - c) (\ln c - \ln b) + (c - a) (\ln a - \ln c)) & \le 0
			\end{aligned}
			$$
			根据 $\ln x$ 单调性,$(a - b) (\ln b - \ln a) \le 0$,故得证。
		\end{enumerate}
	\end{proof}
\end{problem}

\begin{problem}
	课后习题 4.6.23

	\begin{proof}
		$$
		\begin{aligned}
			\ln \text{RHS} - \ln \text{LHS} & = \sum_{i=1}^n \ln \frac{1 + x_i}{x_i} - n \ln \ab(\frac{n + \sum_{i=1}^n x_i}{\sum_{i=1}^n x_i}) \\
			& \ge \sum_{i=1}^n \frac{1}{x_i} - \frac{n^2}{\sum_{i=1}^n x_i} \\
			= n \ab(\frac{\sum_{i=1}^n \frac{1}{x_i}}{n} - \frac{1}{\frac{\sum_{i=1}^n x_i}{n}})
		\end{aligned}
		$$
		记 $f(x) = \frac{1}{x}$,那么上式等价于:
		$$
		n \ab(\frac{\sum_{i=1}^n f(x_i)}{n} - f\ab(\frac{\sum_{i=1}^n x_i}{n}))
		$$
		那么根据 Jensen 不等式得到 $\ln \text{RHS} - \ln \text{LHS} \ge 0$,得证。
	\end{proof}
\end{problem}
