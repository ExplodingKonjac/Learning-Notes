\section{微分 II \& 泰勒公式}

\subsection{函数的微分 continue}

\begin{theorem}
	函数 $f(x)$ 在 $x$ 处可微的充要条件是函数 $f(x)$ 在 $x_0$ 处可导。

	\begin{proof}
		\begin{itemize}
			\item 充分性:取 $A = f'(x_0)$,那么:
			$$
			\lim_{\Delta x \to 0} \frac{f(x_0 + \Delta x) - f(x_0) - A \Delta x}{\Delta x} = \lim_{\Delta x \to 0} \frac{f(x + \Delta x) - f(x)}{\Delta x} - A = 0
			$$
			也就是说 $f(x + \Delta x) - f(x) = o(\Delta x)$。

			\item 必要性:可微即 $f(x) - f(x_0) = A \Delta x + o(\Delta x)$,那么:
			$$
			f'(x_0) = \lim_{\Delta x \to 0} \frac{f(x_0 + \Delta x) - f(x_0)}{\Delta x} = \lim_{\Delta x \to 0} \frac{A \Delta x + o(\Delta x)}{\Delta x} = A
			$$
		\end{itemize}
	\end{proof}
\end{theorem}

这也就说明,导数的记号 $\frac{\dif y}{\dif x}$ 可以看作一个整体,也可以看作 $y, x$ 的微分之商。因此导数也被称做微商。

\subsection{微分的运算法则}

微分的四则运算法则类似于求导法则:

\begin{itemize}
	\item $\dif (u + v) = \dif u + \dif v$;
	\item $\dif (u - v) = \dif u - \dif v$;
	\item $\dif (C u) = C \dif u$;
	\item $\dif (u \cdot v) = u \dif v + v \dif u$;
	\item $\dif \frac{u}{v} = \frac{v \dif u - u \dif v}{v^2}$;
\end{itemize}

\subsection{微分的形式不变性}

微分的形式不变性说的是,对于函数 $y = f(x)$,无论 $x$ 是自变量还是一个中间变量(即关于“真正”自变量的函数),微分总是有下面的形式:
$$
\dif y = f'(x) \dif x
$$
链式法则就是微分的形式不变性的体现。

\subsection{二阶与高阶微分}

\begin{definition}[函数的高阶微分]
	我们这样定义 $n$ 阶微分:
	$$
	\dif^n y = 
	\begin{cases}
		y & , n = 0 \\
		\dif (\dif^{n-1} y) & , n > 0
	\end{cases}
	$$
	那么我们可以得到:
	$$
	\begin{gathered}
		\dif y = y' \dif x \\
		\dif^2 y = \dif (y' \dif x) = y'' \dif x^2 \\
		\cdots \\
		\dif^n y = y^{(n)} \dif x^n
	\end{gathered}
	$$
	这里面出现 $\dif x$ 的高次幂的原因是,在求 $y' \dif x$ 微分时,$\dif x$ 被视作常数。因此 $\dif(y' \dif x) = \dif(y'') \dif x = y'' \dif x^2$。更高阶也同理。
\end{definition}

注意, \textbf{高阶微分不再具有形式不变性}。

\subsection{泰勒公式}

