\section{微分 II \& 泰勒公式}

\subsection{函数的微分 continue}

\begin{theorem}
	函数 $f(x)$ 在 $x$ 处可微的充要条件是函数 $f(x)$ 在 $x_0$ 处可导。

	\begin{proof}
		\begin{itemize}
			\item 充分性:取 $A = f'(x_0)$,那么:
			$$
			\lim_{\Delta x \to 0} \frac{f(x_0 + \Delta x) - f(x_0) - A \Delta x}{\Delta x} = \lim_{\Delta x \to 0} \frac{f(x + \Delta x) - f(x)}{\Delta x} - A = 0
			$$
			也就是说 $f(x + \Delta x) - f(x) = o(\Delta x)$。

			\item 必要性:可微即 $f(x) - f(x_0) = A \Delta x + o(\Delta x)$,那么:
			$$
			f'(x_0) = \lim_{\Delta x \to 0} \frac{f(x_0 + \Delta x) - f(x_0)}{\Delta x} = \lim_{\Delta x \to 0} \frac{A \Delta x + o(\Delta x)}{\Delta x} = A
			$$
		\end{itemize}
	\end{proof}
\end{theorem}

这也就说明,导数的记号 $\frac{\dif y}{\dif x}$ 可以看作一个整体,也可以看作 $y, x$ 的微分之商。因此导数也被称做微商。

\subsection{微分的运算法则}

微分的四则运算法则类似于求导法则:

\begin{itemize}
	\item $\dif (u + v) = \dif u + \dif v$;
	\item $\dif (u - v) = \dif u - \dif v$;
	\item $\dif (C u) = C \dif u$;
	\item $\dif (u \cdot v) = u \dif v + v \dif u$;
	\item $\dif \frac{u}{v} = \frac{v \dif u - u \dif v}{v^2}$;
\end{itemize}

\subsection{微分的形式不变性}

微分的形式不变性说的是,对于函数 $y = f(x)$,无论 $x$ 是自变量还是一个中间变量(即关于“真正”自变量的函数),微分总是有下面的形式:
$$
\dif y = f'(x) \dif x
$$
链式法则就是微分的形式不变性的体现。

\subsection{二阶与高阶微分}

\begin{definition}[函数的高阶微分]
	我们这样定义 $n$ 阶微分:
	$$
	\dif^n y = 
	\begin{cases}
		y & , n = 0 \\
		\dif (\dif^{n-1} y) & , n > 0
	\end{cases}
	$$
	那么我们可以得到:
	$$
	\begin{gathered}
		\dif y = y' \dif x \\
		\dif^2 y = \dif (y' \dif x) = y'' \dif x^2 \\
		\cdots \\
		\dif^n y = y^{(n)} \dif x^n
	\end{gathered}
	$$
	这里面出现 $\dif x$ 的高次幂的原因是,在求 $y' \dif x$ 微分时,$\dif x$ 被视作常数。因此 $\dif(y' \dif x) = \dif(y'') \dif x = y'' \dif x^2$。更高阶也同理。
\end{definition}

注意, \textbf{高阶微分不再具有形式不变性}。

\subsection{泰勒公式}

\begin{theorem}[Taylor 定理]
	设 $f(x)$ 在 $x_0$ 处有 $1 \sim n$ 阶导数,那么:
	$$
	f(x) = \sum_{i=0}^n \frac{f^{(i)}(x)}{i!} (x - x_0)^i + o((x - x_0)^n)
	$$
	其中前面的求和部分称为\textbf{Taylor 多项式},最后的一项称为\textbf{Peano 余项}。Taylor 定理也记作:
	$$
	f(x) = T_n(f, x_0; x) + o((x - x_0)^n)
	$$

	\begin{proof}
		为方便记 $T_k = T_k(f, x_0; x)$。

		对 $n$ 使用归纳法。首先根据微分的定义,$n = 1$ 时成立。然后设 $n = k$ 时成立,那么考虑 $n = k + 1$,我们有:
		$$
		\begin{aligned}
			T_{k+1}' & = \sum_{i=0}^{k+1} \frac{f^{(i)}(x_0)}{i!} \cdot i (x - x_0)^{i-1} \\
			& = \sum_{i=0}^k \frac{f^{(i+1)}(x_0)}{i!} (x - x_0)^i \\
			& = T_k(f', x_0; x)
		\end{aligned}
		$$
		然后,使用洛必达法则:
		$$
		\begin{aligned}
			& \lim_{x \to x_0} \frac{f(x) - T_{k+1}(x)}{(x - x_0)^{k+1}} \\
			= & \lim_{x \to x_0} \frac{f'(x) - T_{k}(f', x_0; x)}{(k+1) (x - x_0)^k} \\
			= & \lim_{x \to x_0} \frac{o(x^k)}{(k+1) (x - x_0)^k} \\
			= & 0
		\end{aligned}
		$$
		因此就证明了余项是 $o((x - x_0)^{k+1})$,归纳完成。
	\end{proof}
\end{theorem}

特别地,在 $x_0 = 0$ 处的 Taylor 展开式被称为\textbf{Maclaurin 展开式}。下面时一些常见的 Maclaurin 展开式:

$$
\begin{gathered}
	\E^x = \sum_{i=0}^n \frac{x^i}{i!} + o(x^n) \\
	\sin x = \sum_{i=1}^n \frac{(-1)^{i-1}}{(2i - 1)!} x^{2i - 1} + o(x^{2n})
\end{gathered}
$$

对于复合函数 $f(g(x))$,也可以直接把 $g(x)$ 代入 Taylor 多项式和 Peano 余项。