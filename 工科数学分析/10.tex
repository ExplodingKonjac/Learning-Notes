\section{求导法}

\subsection{对数求导法}

对于一系列幂指函数的连乘求导,可以先取对数后求导,然后使用复合函数的求导法则以化简运算。具体而言,若:
$$
f(x) = \prod_{i=1}^n u_i(x)^{v_i(x)}
$$
那么我们可以取对数:
$$
\ln f(x) = \sum_{i=1}^n v_i(x) \ln u_i(x)
$$
根据复合函数的求导法则:
$$
f'(x) = f(x) \cdot (\ln f(x))'
$$

\subsection{反函数的导数}

设 $y = f(x)$ 在 $(a, b)$ 上可导且单调,那么其存在反函数 $x = g(y)$。我们对 $x = g(f(x))$ y两边对 $x$ 求导得到:
$$
1 = g'(f(x)) \cdot f'(x) \Rightarrow g'(y) = \frac{1}{f'(g(y))}
$$
这就是反函数的求导法则。

\begin{example}[反三角函数的导数]
	求 $y = \arcsin x$ 的导数。

	\begin{solution}
		$\arcsin x\ (x \in [-1, 1])$ 的反函数是 $\sin x\ \ab(x \in \ab[-\frac{\pi}{2}, \frac{\pi}{2}])$,那么:
		$$
		(\arcsin x)' = \frac{1}{\cos(\arcsin x)} = \frac{1}{\sqrt{1 - x^2}}
		$$
	\end{solution}
\end{example}

于是现在我们可以求出所有常见初等函数的导数:

\begin{itemize}
	\item 反正弦:$(\arcsin x)' = \frac{1}{\sqrt{1 - x^2}}$;
	\item 反余弦:$(\arccos x)' = -\frac{1}{\sqrt{1 - x^2}}$;
	\item 反正切:$(\arctan x)' = \frac{1}{1 + x^2}$。
\end{itemize}

\subsection{隐函数的求导}

\begin{definition}
	若在某个区间上存在函数 $x \mapsto y$,那么:

	\begin{itemize}
		\item 若函数关系由 $y = f(x)$ 确定,那么其称为\textbf{显函数};
		\item 若函数关系由 $F(x, y) = 0$ 确定,那么其称为\textbf{隐函数}。
	\end{itemize}
\end{definition}

要求隐函数的导数,可以对关系式两边同时对 $x$ 求导,得到关系式 $A(x, y) \frac{\dd y}{\dd x} = B(x, y)$,然后求出 $\frac{\dd y}{\dd x} = \frac{B(x, y)}{A(x, y)}$。

\begin{example}
	若
	$$
	F(x, y) = y - x - \varepsilon \sin y = 0
	$$
	求 $y'(x)$。

	\begin{solution}
		两边同时对 $x$ 求导得到:
		$$
		\frac{\dd y}{\dd x} - 1 - \varepsilon \cos y \frac{\dd y}{\dd x} = 0
		$$
		解得
		$$
		\frac{\dd y}{\dd x} = \frac{1}{1 - \varepsilon \cos y}
		$$
	\end{solution}
\end{example}

\subsection{参数方程的求导}

\begin{definition}
	若在某个区间存在函数 $x \mapsto y$,且满足:
	$$
	\begin{cases}
		x = \varphi(t) \\
		y = \psi(t)
	\end{cases}
	$$
	那么称 $y(x)$ 为参数方程确定的函数。
\end{definition}

在特定条件下,参数方程确定的函数可以进行求导。具体而言,若 $\varphi(t)$ 存在连续单调的反函数 $\varphi^{-1}(x)$,且 $\varphi(t), \psi(t)$ 可导,那么
$$
\frac{\dd y}{\dd x} = \frac{\frac{\dd y}{\dd t}}{\frac{\dd x}{\dd t}} = \frac{\psi'(t)}{\varphi'(t)} = \frac{\psi'(\varphi^{-1}(x))}{\varphi'(\varphi^{-1}(x))}
$$

\begin{example}
	求参数曲线
	$$
	\begin{cases}
		x = a \cos^3 t \\
		y = a \sin^3 t
	\end{cases}\quad \ab(t \in \ab[0, \frac{\pi}{2}])
	$$
	的导数。

	\begin{solution}
		$$
		\begin{aligned}
			\frac{\dd y}{\dd x} & = \frac{3a \sin^2 t \cos t}{-3a \cos^2 t \sin t} \\
			& = -\tan t \\
			& = -\tan \arccos \sqrt[3]{\frac{x}{a}}
		\end{aligned}
		$$
	\end{solution}
\end{example}

\subsection{高阶导数}

\begin{definition}
	$f(x)$ 的 $n\ (n \in \mathbb{N})$ 阶导数记作 $f^{(n)}(x)$,定义为:
	$$
	f^{(n)}(x) = \begin{cases}
		f(x) & , n = 0 \\
		\ab(f^{(n-1)}(x))' & , n \ge 1
	\end{cases}
	$$
\end{definition}

对于高阶导数也有一些运算法则:

\begin{theorem}[高阶导数运算法则]
	\begin{itemize}
		\item 加减法:$(u \pm v)^{(n)} = u^{(n)} \pm v^{(n)}$;
		\item 乘常数:$(C u)^{(n)} = C u^{(n)}$;
		\item 乘法:$(u \cdot v)^{(n)} = \sum_{k=0}^n \binom{n}{k} u^{(k)} v^{(n - k)}$。
	\end{itemize}

	\begin{proof}
		只证明乘法法则。观察导数的乘法法则的组合意义:
		$$
		(uv)' = u'v + uv'
		$$
		相当于在 $u, v$ 中选择一个,使其的导数阶数 $+1$,并对所有选择求和。因此可得 $(uv)^{(n)}$ 就相当于进行 $n$ 次这样的操作,并对所有可能情况求和。于是可以直接得到:
		$$
		(uv)^{(n)} = \sum_{k=0}^n \binom{n}{k} u^{(k)} v^{(n - k)}
		$$
	\end{proof}
\end{theorem}