\section{不定积分 II}

\subsection{换元积分法}

\begin{theorem}[第一类换元法]
	设 $f(u)$ 在区间 $J$ 上有原函数 $G(u)$,且 $u=\varphi(x)$ 在区间 $I$ 上可导,$\{\varphi(x): x \in I\} \subseteq J$,那么:
	$$
	\int f(\varphi(x)) \varphi'(x) \dif x = G(\varphi(x)) + C
	$$
	或者记作下面的形式:
	$$
	\int f(\varphi(x)) \dif \varphi(x) = \left.\int f(u) \dif u\right|_{u=\varphi(x)}
	$$
\end{theorem}

\begin{example}
	求 $\displaystyle \int \sin 2x \dif x$。
	\begin{solution}
		$$
		\int \sin 2x \dif x = \frac{1}{2} \int \sin 2x \dif (2x) = -\frac{1}{2} \cos 2x + C
		$$
		或者:
		$$
		\int \sin 2x \dif x = 2 \int \sin x \cos x \dif x = 2 \int \sin x \dif (\sin x) = \sin^2 x + C_2
		$$
	\end{solution}
\end{example}

\begin{example}
	求 $\displaystyle \int \frac{\dif x}{x^2 - 8x + 25}$。
	\begin{solution}
		$$
		\begin{aligned}
			\int \frac{\dif x}{x^2 - 8x + 25} & = \int \frac{\dif x}{(x - 4)^2 + 9} \\
			& = \frac{1}{3} \int \frac{\frac{1}{3} \dif x}{\ab(\frac{x - 4}{3})^2 + 1} \\
			& = \frac{1}{3} \arctan \frac{x - 4}{3} + C
		\end{aligned}
		$$
	\end{solution}
\end{example}

\begin{theorem}[第二类换元法]
	设 $x = \psi(t)$ 单调且可导,并且 $\psi(t) \neq 0$,且 $f(\psi(t)) \psi'(t)$ 有原函数 $G(t)$,那么:
	$$
	\int f(x) \dif x = G(\psi^{-1}(t))
	$$
	或者记作下面的形式:
	$$
	\int f(x) \dif x = \left.\int f(\psi(t)) \dif \psi(t)\right|_{t = \psi^{-1}(x)}
	$$
\end{theorem}

\begin{example}
	求 $\displaystyle \int \frac{\dif x}{\sqrt{x^2 + a^2}}$。
	\begin{solution}
		$$
		\begin{aligned}
			\int \frac{\dif x}{\sqrt{x^2 + a^2}} & = \frac{\dif (a \tan t)}{\sqrt{a^2 \tan^2 t + a^2}} \\
			& = \int \frac{a \sec^2 t \dif t}{a \sec t} \\
			& = \int \sec t \dif t \\
			& = \ln |\sec t + \tan t| + C \\
			& = \ln \ab(\frac{x}{a} + \frac{\sqrt{x^2 + a^2}}{a}) + C
		\end{aligned}
		$$
	\end{solution}
\end{example}

\begin{example}
	求 $\displaystyle \int \frac{\dif x}{x(x^7 + 2)}$。
	\begin{solution}
		$$
		\begin{aligned}
			\int \frac{\dif x}{x(x^7 + 2)} & = \int \frac{\dif \frac{1}{t}}{\frac{1}{t} \ab(\frac{1}{t^7} + 2)} \\
			& = \int \frac{t^8}{1 + 2 t^7} \cdot \ab(-\frac{1}{t^2}) \dif t \\
			& = -\int \frac{t^6}{1 + 2 t^7} \dif t \\
			& = -\frac{1}{14} \ln |1 + 2 t^7| + C
		\end{aligned}
		$$
	\end{solution}
\end{example}

\subsection{分部积分法}

\begin{theorem}[分部积分法]
	设函数 $u(x), v(x)$ 可导,那么:
	$$
	\int u(x) v'(x) \dif x = u(x) v(x) - \int v(x) u'(x) \dif x
	$$
	或者记作下面的形式:
	$$
	\int u \dif v = u v - \int v \dif u
	$$
\end{theorem}

\begin{example}
	求 $\displaystyle x^2 \E^x \dif x$。
	\begin{solution}
		$$
		\begin{aligned}
			\int x^2 \E^x \dif x & = \int x^2 \dif \E^x \\
			& = x^2 \E^x - \int 2x \E^x \dif x \\
			& = x^2 \E^x - \int 2x \dif \E^x \\
			& = x^2 \E^x - \ab(2x \E^x - \int 2 \E^x) \\
			& = x^2 \E^x - 2x \E^x + 2\E^x + C
		\end{aligned}
		$$
	\end{solution}
\end{example}

\begin{example}
	求 $\displaystyle \sin \ln x \dif x$。
	\begin{solution}
		$$
		\begin{aligned}
			\int \sin \ln x \dif x & = x \sin \ln x - \int x \dif (\sin \ln x) \\
			& = x \sin x \ln x - \int \cos \ln x \dif x \\
			& = x \sin \ln x - \ab(x \cos \ln x - \int x \dif (\cos \ln x)) \\
			& = x \sin \ln x - x \cos \ln x - \int \sin \ln x \dif x \\
		\end{aligned}
		$$
		那么就可以解出来
		$$
		\int \sin \ln x \dif x = \frac{x \sin \ln x - x \cos \ln x}{2}
		$$
	\end{solution}
\end{example}

\begin{example}
	求 $\displaystyle \int \cos^n \dif x$。
	\begin{solution}
		$$
		\begin{aligned}
			\int \cos^n \dif x & = \int \cos^{n-1} x \dif \sin x \\
			& = \sin x \cos^{n-1} x + (n-1) \int \cos^{n-2} x \sin^2 x \dif x \\
			& = \sin x \cos^{n-1} x + (n-1) \int \cos^{n-2} x \dif x - (n-1) \int \cos^n x \dif x
		\end{aligned}
		$$
		记 $\displaystyle I_n = \int \cos^n \dif x$,那么:
		$$
		n I_n = \sin x \cos^{n-1} x + (n-1) I_{n-2}
		$$
		并且我们有 $\displaystyle I_1 = \int \cos x \dif x = \sin x + C$,因此我们可以递归求出 $I_n$。
	\end{solution}
\end{example}
