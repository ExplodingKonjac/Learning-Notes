\section{单调性、极值、凹凸性}

\subsection{单调性}

\begin{theorem}
	设函数 $f(x)$ 在 $[a, b]$ 上连续,在 $(a, b)$ 内可导,那么:

	\begin{itemize}
		\item $f(x)$ 在 $[a, b]$ 上单调不降的充要条件是 $\forall\,x \in (a, b): f'(x) \ge 0$;
		\item $f(x)$ 在 $[a, b]$ 上单调不升的充要条件是 $\forall\,x \in (a, b): f'(x) \le 0$。
	\end{itemize}

	\begin{proof}
		下面仅证明单调不降的情况。

		\begin{itemize}
			\item 必要性:
			$$
			f'(x) = f'_+(x) = \lim_{h \to 0^+} \frac{f(x + h) - f(x)}{h}
			$$
			根据极限的保号性得证。注意这里即使函数严格递增,也只能得到 $f'(x) \ge 0$。

			\item 充分性:根据 Larange 中值定理:
			$$
			\forall\,a \le x_1 < x_2 \le b: f(x_2) - f(x_1) = f'(\xi)(x_2 - x_1) > 0
			$$
		\end{itemize}
	\end{proof}
\end{theorem}

而关于严格单调的情况,我们也有定理:

\begin{theorem}
	若 $f(x)$ 在 $[a, b]$ 连续,在 $(a, b)$ 内可导,那么 $f(x)$ 严格单调递增/递减的充要条件是:

	\begin{enumerate}
		\item $\forall\,x \in (a, b): f'(x) \ge 0$(或 $\le 0$);
		\item 在 $(a, b)$ 的任意子开区间内,$f'(x)$ 不恒等于 $0$。
	\end{enumerate}

	\begin{proof}
		\begin{itemize}
			\item 必要性:显然 1. 成立。而对于 2. 使用反证,使用 Lagrange 中值定理得到函数在某个子区间内是常函数,矛盾,因此得证。

			\item 充分性:由 1. 成立得到 $f(x)$ 单调不降。而若 $f(x)$ 不是严格单调,则可以找到一个区间内 $f(x)$ 是常函数,与 2. 矛盾,因此得证。
		\end{itemize}
	\end{proof}
\end{theorem}

\subsection{极值点}

之前已经给出过极值点的定义。简而言之:函数某点值是某个邻域内的严格最大/最小值,那么这个值称为函数的极大/极小值。

对于极值点有若干个必要性判定方法:

\begin{itemize}
	\item 函数的单调区间的分界点必然是极值点;
	\item 若函数的驻点处二阶可导,并且 $f''(x_0) > 0$(或 $<0$),那么该点是极小值(或极大值)。 
\end{itemize}

注意,极值点不能推出两侧具有单调性,反例:

$$
f(x) = \begin{cases}
	x^2 \ab(1 + \sin \frac{1}{x}) & , x \neq 0 \\
	0 & , x = 0
\end{cases}
$$

\subsection{曲线的凹凸性}

\begin{definition}[凸函数]
	设 $f(x)$ 在区间 $I$ 上连续,如果 $f(x)$ 在 $I$ 上任意一条弦都在函数图像的上方,那么称 $f(x)$ 是下凸函数。形式化地:
	$$
	\forall\, x_1, x_2 \in I,\ \lambda \in (0, 1): f(\lambda x_1 + (1 - \lambda) x_2) \le \lambda f(x_1) + (1 - \lambda) f(x_2)
	$$
	若把 $\le$ 换成 $<$,那么称函数是严格下凸函数。

	把 $\le, <$ 换成 $\ge, >$,就得到了 $f(x)$ 是(严格)上凸函数的定义。
\end{definition}

对于函数下凸性有定理:

\begin{theorem}
	$f(x)$ 在区间 $I$ 下凸的充要条件是:
	$$
	\forall\, x_1 < x < x_2 \in I: \frac{f(x) - f(x_1)}{x - x_1} \le \frac{f(x_2) - f(x_1)}{x_2 - x_1} \le \frac{f(x_2) - f(x)}{x_2 - x}
	$$
	同理,把 $\le$ 换成 $<$,就得到了 $f(x)$ 在 $I$ 上严格下凸的充要条件。
\end{theorem}

凸性也可以与导函数结合考虑:

\begin{theorem}
	若 $f(x)$ 在 $(a, b)$ 可导,那么 $f(x)$ (严格)下凸的充要条件是 $f'(x)$ 在 $(a, b)$ 内(严格)单调增。

	\begin{proof}
		\begin{itemize}
			\item 必要性:对于 $x_1, x_2$,那么根据下凸函数基本性质有:
			$$
			\frac{f(x) - f(x_1)}{x - x_1} \le \frac{f(x_2) - f(x_1)}{x_2 - x_1} \le \frac{f(x_2) - f(x)}{x_2 - x}
			$$
			根据极限保号性就得到 $f'(x_1) \le f'(x_2)$。

			\item 充分性:使用下凸函数基本性质结合 Lagrange 中值定理证明。
		\end{itemize}
	\end{proof}
\end{theorem}

考察凹凸性地变化,我们的处函数拐点的定义:

\begin{definition}[曲线拐点]
	曲线上下凸性改变的点称为曲线地拐点。

	注意,\textbf{拐点真的是点} /oh,不是 $x=a$ 类似物。
\end{definition}

当曲线二阶可导的时候,可以用二阶导数的正负性来判定,而二阶不可导甚至一阶不可导的时候也可能有拐点。

\begin{theorem}[Jensen 不等式]
	若 $f(x)$ 在 $I$ 上下凸,那么
	$$
	\forall\,x_1, x_2, \dots, x_n \in I,\ \lambda_1, \lambda_2, \dots, \lambda_n > 0, \sum_{i=1}^n \lambda_i = 1: f\ab(\sum_{i=1}^n \lambda_i x_i) \le \sum_{i=1}^n \lambda_i f(x_i)
	$$
	当 $f(x)$ 严格下凸时,上式加强为小于号。

	\begin{proof}
		归纳法。假设求和到 $k$ 并且归一化之后结论成立(记 $u_i = \frac{\lambda_i}{\sum_{i=1}^k \lambda_I}$)。那么:
		$$
		\begin{aligned}
			f\ab(\sum_{i=1}^{k+1} \lambda_i x_i) & = f\ab((1 - \lambda_{k+1}) \sum_{i=1}^k u_i x_i + \lambda_{k+1} x_{x+1}) \\
			& \le (1 - \lambda_{k+1}) f\ab(\sum_{i=1}^k u_i x_i) + \lambda_{k+1} f(x_{k+1}) \\
			& \le (1 - \lambda_{k+1}) \sum_{i=1}^k u_i f(x_i) + \lambda_{k+1} f(x_{k+1}) \\
			& = \sum_{i=1}^{k+1} \lambda_i f(x_i)
		\end{aligned}
		$$
	\end{proof}
\end{theorem}